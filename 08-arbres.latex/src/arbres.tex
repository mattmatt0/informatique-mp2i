\newcommand\PATH{Lancez `make adapt avant de compiler!`}

\RequirePackage{fix-cm}
\documentclass{scrartcl}

% Fichier pour les dépendances

% Fontes Computer Modern
\usepackage{lmodern}

% Minted
\usepackage[cache=true,outputdir=\PATH/obj]{minted}

% Pour les symboles et outils mathématiques
\usepackage{amsmath}
\usepackage{amsfonts}
\usepackage{amssymb}
\usepackage{mathtools}
\usepackage{cancel}
\usepackage{mathrsfs}
\usepackage{stmaryrd}

% Pour les dessins / images
\usepackage{xcolor}
\usepackage{tikz}
\usepackage{graphicx}

% Tableaux et listes
\usepackage{tabularx}
\newcolumntype{Y}{>{\centering\arraybackslash}X}
\usepackage{booktabs}
\usepackage{multirow}
\usepackage{enumitem}
\usepackage{multicol}
\usepackage{adjustbox}

% Présentation de la page (bordures)
\usepackage[a4paper,top=1cm,bottom=1cm,includefoot,left=1.5cm,right=1.5cm,footskip=1cm]{geometry}
\usepackage{scrlayer-scrpage}
\rofoot*{\pagemark}
\cofoot*{}
\pagestyle{scrheadings}

% Autres
\usepackage{ulem} % Soulignage
\usepackage[french]{babel} % Babel, pour respecter les standards français
\frenchbsetup{StandardLists=true} % ... sauf pour les immondes listes à tirets
\usepackage[T1]{fontenc}

\usepackage{hyperref} % Pour les liens
\hypersetup{
    colorlinks=true,
    linkcolor=blue,
    filecolor=magenta,
    urlcolor=cyan,
}


%% Commandes personnalisées

% Code
\newenvironment{code}[1]
{\VerbatimEnvironment\begin{minted}[breaklines,frame=single,autogobble,linenos,bgcolor=codebg,tabsize=4,mathescape=true]{#1}}
{\end{minted}}
\definecolor{codebg}{gray}{1}
\newenvironment{algotext}
{\VerbatimEnvironment\begin{minted}[frame=single,autogobble,linenos,bgcolor=codebg,tabsize=4,escapeinside=||,mathescape=true]{text}}
{\end{minted}}


% Paragraphes spéciaux
\newcounter{compteurExos}
\setcounter{compteurExos}{0}
\newcommand{\plabel}[1]{{\parindent10pt \fontfamily{cmss}\selectfont\textbf{#1} \,}}

\newcommand{\prop}[1]{\plabel{Propriété :} \textsl{#1}}
\newcommand{\lemma}[1]{\plabel{Lemme :} \textsl{#1}}
\newcommand{\rem}{\plabel{Remarque :}}
\newcommand{\exemple}{\plabel{Exemple :}}
\newcommand{\exo}{\stepcounter{compteurExos}\plabel{Exercice \thecompteurExos :}}
\newcommand{\obs}{\plabel{Observation :} }

\newenvironment{demo}{\begin{itemize}[label=$\triangleright$]}{\end{itemize}}
\newcommand{\definition}{\parindent0pt }
\newcommand{\semidef}{\parindent0pt $\bullet$ \,}

% Outils
\newcommand{\corrpar}{\vspace{-20pt}}
\newcommand{\extraspace}{\vspace{5pt}}

% Configuration
\parskip5pt

% Annonce pour le développement:
\newcommand{\notecentrale}[1]{\begin{center}\textsl{#1}\end{center}}
\newcommand{\warnwip}{\notecentrale{Ce cours n'est pas encore bien relu}}
\newcommand{\warnwipnext}{\notecentrale{La suite de ce cours est encore en rédaction}\message}


\newcommand{\fpq}{\mathbb{F}_p(Q)}
\newcommand{\abins}{\mathscr{A}_B(\mathcal{S})}
\newcommand{\pb}[3]
{
	\textbf{#1}
	\left|\left|\begin{tabular}{l l}
		Entrée: & #2 \\
		\vspace{-10pt} & \\
		\hline
		\vspace{-10pt} & \\
		Sortie: & #3
	\end{tabular}\right.\right.
}
\newcommand{\pbinlist}[3]
{
	\textbf{#1} &  $\left|\left|\begin{tabular}{l l}
		Entrée: & #2\\ 
		\vspace{-10pt} & \\
		\hline
		\vspace{-10pt} & \\
		Sortie: & #3 
	\end{tabular}\right.\right.$ 
}
\newcommand{\val}{\text{val}}
\newcommand{\argmax}{\mathop{\text{argmax}}}
\newcommand{\argmin}{\mathop{\text{argmin}}}

\newcommand{\set}[1]{{\{#1\}}}
\newcommand{\intset}[1]{\llbracket #1 \rrbracket}
\newcommand{\intsete}[1]{\llbracket #1 \llbracket}
\newcommand{\inteset}[1]{\rrbracket #1 \rrbracket}
\newcommand{\intesete}[1]{\rrbracket #1 \llbracket}

\newcommand{\tq}{\, \big| \,}

\newcommand{\card}{\.\textmd{card}}
\newcommand{\Id}{\textsl{Id}}
\newcommand{\supp}{\text{supp}}

% Partie entière, arrondi supérieur

\DeclarePairedDelimiter\ceil{\lceil}{\rceil}
\DeclarePairedDelimiter\floor{\lfloor}{\rfloor}



\usepackage{lmodern}

\title{Structures de données arborescentes}
\author{}
\date{}

\begin{document}
	\maketitle
	\section{Motivations}
		\subsection{Un arbre pour un objet}
			Pour des éléments d'un ensemble construit par induction, 
			il est approprié d'utiliser une représentation par arbre puisque 
			ces objets ont intrinsèquement une structure arborescente

			\exemple Les expressions booléennes, arithmétiques, de type. 

		\subsection{Un arbre pour une collection d'objet}
			On cherche à stocker une collection d'objets sans multiplicité,
			et dont l'ordre relatif n'est pas significatif 
			(comme dans le cas d'un ensemble).

			On suppose que tous les éléments sont du type \texttt{elem}, 
			et qu'ils sont identifiés de manière unique par une clé, 
			c'est à dire une sous-partie permettant l'identification.
			
			On a alors les méthodes suivantes:
			\begin{multicols}{2}
			\begin{itemize}
				\item \texttt{creer\_ens\_vide : () -> ens}
				\item \texttt{est\_ens\_vide : ens -> bool}
				\item \texttt{appartient : ens$\times$elem -> bool}
				\item \texttt{ajoute\_elem : ens$\times$elem -> ens}
				\item \texttt{supprime\_elem : ens$\times$clé -> ens}
				\item \texttt{trouve\_elem : ens$\times$clé -> elem}
			\end{itemize}
			\end{multicols} 

			\rem On peut aussi imaginer des fonctions \texttt{ajoute\_elem} et \texttt{supprime\_elem}
			qui modifieraient directement l'ensemble donné en entrée, 
			plutôt que de renvoyer un nouvel ensemble. 

		\subsection{Implémentations}
			On peut stocker les éléments dans une liste. 
			On peut aussi associer chaque élément à une clé
			(pour un ensemble de caractères par exemple, leurs valeurs ascii),
			qui permet de les comparer rapidement. 
			Si l'on peut ordonner ces clés, 
			on peut alors classer les éléments par ordre croissant dans un tableau.

			\begin{center}
			\begin{tabular}{| c | c  | c |}
				\hline
				Opération & Complexité (Liste) & Complexité (Tableau ordonné) \\
				\hline
				\texttt{appartient} & $\theta(n)$ & $\theta(\log(n))$ (dichotomie) \\
				\hline
				\texttt{ajoute\_elem} & $\theta(1)$ & $\theta(n)$ \\
				\hline
				\texttt{supprime\_elem} & $\theta(n)$ & $\theta(n)$ \\
				\hline
				\texttt{trouve\_elem} & $\theta(n)$ & $\theta(\log(n))$ \\
				\hline
			\end{tabular}
			\end{center}

			\rem On voit selon le contexte qu'une certaine implémentation sera plus efficace qu'une autre:
			si on doit faire beaucoup d'insertions sans trop chercher d'éléments, le plus efficace sera la liste.
			Par contre, si on n'insère que rarement des éléments mais que 
			l'on est souvent ammené à chercher dans les éléments, le tableau trié sera à préférer.

			ll existe cela dit une structure qui permet d'avoir une complexité d'ajout / suppression et de recherche
			en $\theta(\log(n))$, sous certaines conditions: il s'agit des \textbf{arbres binaires de recherche (ABR)}.

	\section{Arbre minimal}
		\subsection{Ensemble $\mathscr{A}_B$}
			Soit $\mathcal{S}$ un ensemble.
			On définit l'ensemble des arbres binaires étiquetés par $\mathcal{S}$ 
			comme l'ensemble construit par induction à partir de ces deux règles:

			\begin{multicols}{2}
			\begin{itemize}
				\item $V\Big|_{\set{-}}^0$ 
				\item $N\Big|_{\mathcal{S}}^2$ 
			\end{itemize}
			\end{multicols}

			\exemple Soit $\mathcal{S} = \set{1,2,3,4}$. 

			\begin{center}
			\begin{tabular}{c | c | c}
				\begin{tikzpicture}
					\node (a) {$\times$};
				\end{tikzpicture}
				\quad & \quad
				\begin{tikzpicture}
					\node [circle,draw](a) {1}
						child {node (b) {$\times$}}
						child {node (b) {$\times$}};
				\end{tikzpicture} 
				\quad & \quad
				\begin{tikzpicture}[
					level 1/.style={sibling distance=10em}, 
					level 2/.style={sibling distance=5em}, 
					level 3/.style={sibling distance=2em}]
					\node [circle,draw](a) {4}
						child {node [circle,draw] (b) {2}
							child {node (c) {$\times$}}
							child {node (c) {$\times$}}}
						child {node [circle,draw] (b) {3}
							child {node [circle,draw] (c) {$1$}
								child {node (d) {$\times$}}
								child {node (d) {$\times$}}}
							child {node (c) {$\times$}}};
				\end{tikzpicture} \\
				$(V,\_)$ & $(N,1,(V,\_),(V,\_))$  & 
				$(N,4,(N,2,(V,\_),(V,\_)),(N,3,(N,1,(V,\_),(V,\_)),(V,\_)))$
			\end{tabular}
			\end{center}
			
		\subsection{Feuilles}
			\begin{center}\textsl{On fixe désormais $\mathcal{S}$ un ensemble d'étiquettes.
			On notera par ailleurs $N(x,g,d) = (N,x,g,d)$ et $V = (V,\_)$ pour alléger l'écriture.}\end{center}
			
			On dit qu'un arbre $t \in \abins$ est réduit à une feuille ssi il existe $x \in \mathcal{S}$
			tel que $t = N(x,V,V)$.

			\prop{Pour la relation d'ordre $\leq$ associée à la définition inductive de $\abins$, on a:}
			\begin{enumerate}
				\item \textsl{$V$ est le seul élément minimal de $(\abins, \leq)$.}
				\item \textsl{$t$ élément minimal de $\abins\setminus\set{V} \Leftrightarrow$ $t$ est réduit à une feuille.}
			\end{enumerate}
			\begin{demo}
				\item Le 1) découle du fait que $V$ soit le seul cas de base des règles d'induction de $\abins$.
				\item Soit $t$ minimal dans $\abins\setminus\set{V}$. 
				Il existe par définition $x \in \mathcal{S}$ et $(g,d) \in \abins^2$ tels que $t = N(x,g,d)$.
				Comme $g \leq t$ et $d \leq t$, et d'autre part $g \neq t$ et $d \neq t$,
				on en déduit que $g < t$ et $d < t$. 
				Par minimalité de $t$ dans $\abins\setminus\set{V}$, on en déduit que $g = d =V$,
				donc que $t$ est réduit à une feuille.
			\end{demo}

		\subsection{Chemins}
			Considérons l'alphabet $\Sigma = \set{0,1}$. 
			On définit par induction sur $\abins$ l'ensemble des chemins admissibles d'un arbre binaire $t$,
			noté $\text{ch}(t)$, par:
			\[
				\forall t \in \abins: \quad \text{ch}(t) = 
				\begin{cases}
					\varnothing \text{ si } t = V \\ 
					\set{\epsilon}\cup\set{0.\text{ch}(g)}\cup\set{1.\text{ch}(d)} \text{ si } t = N(x,g,d)
				\end{cases}
			\]

			Soit $t \in \abins$. Un \textbf{noeud} $\mathcal{N}$ de $t$ est un élément de $\text{ch}(t)$. 
			Sa profondeur, notée |$\mathcal{N}$| est alors sa longueur en tant que mot de $\Sigma^*$.
			La taille de $t$, notée $s(t)$, est alors le nombre de noeuds de $t$, 
			autrement dit $\card\text{ch}(t)$.

			\rem Un chemin admissible décrit la "position" d'un "noeud" dans l'arbre.

			\exo Donner une définition inductive de $s(t)$.

			
\end{document}
