\newcommand\PATH{/home/matthieu/Documents/Ecole/Informatique/cours/14-graphes}

\RequirePackage{fix-cm}
\documentclass{scrartcl}

% Fichier pour les dépendances

% Fontes Computer Modern
\usepackage{lmodern}

% Minted
\usepackage[cache=true,outputdir=\PATH/obj]{minted}

% Pour les symboles et outils mathématiques
\usepackage{amsmath}
\usepackage{amsfonts}
\usepackage{amssymb}
\usepackage{mathtools}
\usepackage{cancel}
\usepackage{mathrsfs}
\usepackage{stmaryrd}

% Pour les dessins / images
\usepackage{xcolor}
\usepackage{tikz}
\usepackage{graphicx}

% Tableaux et listes
\usepackage{tabularx}
\newcolumntype{Y}{>{\centering\arraybackslash}X}
\usepackage{booktabs}
\usepackage{multirow}
\usepackage{enumitem}
\usepackage{multicol}
\usepackage{adjustbox}

% Présentation de la page (bordures)
\usepackage[a4paper,top=1cm,bottom=1cm,includefoot,left=1.5cm,right=1.5cm,footskip=1cm]{geometry}
\usepackage{scrlayer-scrpage}
\rofoot*{\pagemark}
\cofoot*{}
\pagestyle{scrheadings}

% Autres
\usepackage{ulem} % Soulignage
\usepackage[french]{babel} % Babel, pour respecter les standards français
\frenchbsetup{StandardLists=true} % ... sauf pour les immondes listes à tirets
\usepackage[T1]{fontenc}

\usepackage{hyperref} % Pour les liens
\hypersetup{
    colorlinks=true,
    linkcolor=blue,
    filecolor=magenta,
    urlcolor=cyan,
}


%% Commandes personnalisées

% Code
\newenvironment{code}[1]
{\VerbatimEnvironment\begin{minted}[breaklines,frame=single,autogobble,linenos,bgcolor=codebg,tabsize=4,mathescape=true]{#1}}
{\end{minted}}
\definecolor{codebg}{gray}{1}
\newenvironment{algotext}
{\VerbatimEnvironment\begin{minted}[frame=single,autogobble,linenos,bgcolor=codebg,tabsize=4,escapeinside=||,mathescape=true]{text}}
{\end{minted}}


% Paragraphes spéciaux
\newcounter{compteurExos}
\setcounter{compteurExos}{0}
\newcommand{\plabel}[1]{{\parindent10pt \fontfamily{cmss}\selectfont\textbf{#1} \,}}

\newcommand{\prop}[1]{\plabel{Propriété :} \textsl{#1}}
\newcommand{\lemma}[1]{\plabel{Lemme :} \textsl{#1}}
\newcommand{\rem}{\plabel{Remarque :}}
\newcommand{\exemple}{\plabel{Exemple :}}
\newcommand{\exo}{\stepcounter{compteurExos}\plabel{Exercice \thecompteurExos :}}
\newcommand{\obs}{\plabel{Observation :} }

\newenvironment{demo}{\begin{itemize}[label=$\triangleright$]}{\end{itemize}}
\newcommand{\definition}{\parindent0pt }
\newcommand{\semidef}{\parindent0pt $\bullet$ \,}

% Outils
\newcommand{\corrpar}{\vspace{-20pt}}
\newcommand{\extraspace}{\vspace{5pt}}

% Configuration
\parskip5pt

% Annonce pour le développement:
\newcommand{\notecentrale}[1]{\begin{center}\textsl{#1}\end{center}}
\newcommand{\warnwip}{\notecentrale{Ce cours n'est pas encore bien relu}}
\newcommand{\warnwipnext}{\notecentrale{La suite de ce cours est encore en rédaction}\message}


\newcommand{\fpq}{\mathbb{F}_p(Q)}
\newcommand{\abins}{\mathscr{A}_B(\mathcal{S})}
\newcommand{\pb}[3]
{
	\textbf{#1}
	\left|\left|\begin{tabular}{l l}
		Entrée: & #2 \\
		\vspace{-10pt} & \\
		\hline
		\vspace{-10pt} & \\
		Sortie: & #3
	\end{tabular}\right.\right.
}
\newcommand{\pbinlist}[3]
{
	\textbf{#1} &  $\left|\left|\begin{tabular}{l l}
		Entrée: & #2\\ 
		\vspace{-10pt} & \\
		\hline
		\vspace{-10pt} & \\
		Sortie: & #3 
	\end{tabular}\right.\right.$ 
}
\newcommand{\val}{\text{val}}
\newcommand{\argmax}{\mathop{\text{argmax}}}
\newcommand{\argmin}{\mathop{\text{argmin}}}

\newcommand{\set}[1]{{\{#1\}}}
\newcommand{\intset}[1]{\llbracket #1 \rrbracket}
\newcommand{\intsete}[1]{\llbracket #1 \llbracket}
\newcommand{\inteset}[1]{\rrbracket #1 \rrbracket}
\newcommand{\intesete}[1]{\rrbracket #1 \llbracket}

\newcommand{\tq}{\, \big| \,}

\newcommand{\card}{\.\textmd{card}}
\newcommand{\Id}{\textsl{Id}}
\newcommand{\supp}{\text{supp}}

% Partie entière, arrondi supérieur

\DeclarePairedDelimiter\ceil{\lceil}{\rceil}
\DeclarePairedDelimiter\floor{\lfloor}{\rfloor}



\usepackage{lmodern}
\usepackage{bbm}

\title{Graphes}
\author{}
\date{}

\begin{document}
	\maketitle
	\section{Définitions}
		\subsection{Graphes}
			\definition Un graphe \textbf{non orienté} est la donnée 
			d'un couple $G = (V,E)$,
			où $V$ est un ensemble fini non vide 
			et $E \subset \set{\set{x,y} \tq (x,y) \in V^2}$.
			Un graphe \textbf{orienté} est la donnée d'un couple $H = (S,A)$,
			où $S$ est un ensemble fini non vide et 
			$A \in \mathcal{P}(S^2)$.

			Les éléments de $V$ et $A$ sont appelés \textbf{sommets du graphe}, 
			ceux de $E$ sont ses \textbf{arrêtes}, et ceux de $A$ ses \textbf{arcs}.

			Si $e = \set{x} \in E$ (avec $x \in V$), $e$ est une \textbf{boucle} sur $x$
			(idem pour $e = (x,x)$ pour $x \in S$). 
			Pour $(x,y) \in V^2$, on dit que $x$ et $y$ sont \textbf{voisins} ssi ${x,y} \in E$.
			Dans un graphe orienté, $x \in S$ est \textbf{successeur (resp. prédecesseur)de $y \in V$} ssi $(y,x)$ (resp. $(x,y)$) $\in A$.

			On appelle \textbf{voisinnage} de $x \in E$ l'ensemble de ses voisins.
			Le \textbf{degré} de $x$, noté $\deg x$, est le cardinal de ce voisinnage.

			Pour les graphes orientés, on distingue le \textbf{degré sortant} de $x$, noté $\deg^+ x$, le nombre de successeurs de $x$,
			du \textbf{degré entrant} de $x$, noté $\deg^- x$,le nombre de prédecesseurs de $x$.

			\notecentrale{On supposera par la suite que l'on travaille avec des graphes sans boucles}

			\prop{Soit $G = (V,E)$ un graphe. On a $\sum_{x \in V} \deg(x) = 2\card(E)$ }
			\begin{demo}
				\item On a:
				\[
					\sum_{x \in V} \deg(x)= \sum_{x\in V} \sum_{y\in V} \mathbbm{1}_\set{x,y} = \sum_{(x,y) \in V^2} \mathbbm{1}_\set{x,y}
					= 2\sum_{\set{x,y} \in V^2} \mathbbm{1}_\set{x,y} = 2\,\card(E).
				\]
				Car le graphe est sans boucle. Il faudrait sinon ajouter le nombre de boucles présentes dans le graphe.
			\end{demo}

		\subsection{Accessibilité, connexité}
			\notecentrale{On fixe $G = (V,E)$ un graphe non orienté, 
			et $H = (A,S)$ un graphe orienté.}
			
			\definition Soit $s = (s_i)\in V^{n+1}$. 
			On dit que $s$ est une \textbf{chaîne de $G$} ssi
			$\forall i \in \intsete{1,n}, \set{s_i, s_{i+1}} \in E$
			On dit alors que $s$ est une chaîne de longueur $n$ et qui relie $s_0$ \textbf{et} $s_n$.

			\definition Soit $s = (s_i)\in A^{n+1}$. 
			On dit que $s$ est un \textbf{chemin de $G$} ssi
			$\forall i \in \intsete{1,n}, \set{s_i, s_{i+1}} \in E$	
			On dit alors que $s$ est une chaîne de longueur $n$ et qui relie $s_0$ \textbf{à} $s_n$.

			On dit alors que $s_n$ est accessible depuis $s_0$.
			Par ailleurs, si $s_n = s_0$, on dit que $s$ est un \textbf{cycle} pour un graphe non-orienté, 
			ou un \textbf{circuit} dans un graphe orienté.

			Si tous les $(s_i)$ sont distincts, on dit que $s$ est \textbf{élémentaire}. 

			\rem Il y a toujours un nombre fini de chaînes élémentaires, mais si $G$ (resp. $H$) a des cycles (resp. des circuits),
			il y a un nombre infini de chaînes (il suffit de tourner en rond...).

			\exo Définir la relation entre les circuits/chemins d'un graphe, 
			qui met en relation deux circuits/chemins ssi ils relient les mêmes sommets.
			Est-ce une relation d'équivalence?

			\prop{La relation $\mathcal{R}$ définie sur $V^2$ par $x\mathcal{R}y$ ssi $x$ est accessible depuis $y$ est une relation d'équivalence.}
			\begin{demo}
				\item Soit $x \in V$. On a bien $x\mathcal{R}x$: la chaîne de longueur $n=0$ $s = (x)$ convient.
				\item Soit $(x,y) \in V^2$, tel que $x \mathcal{R}y$. 
					Alors par définition il existe $s = (s_0,...,s_n) \in V^{n+1}$ tel que $s_0 = x$ et $s_n = y$,
					et $\forall i \in \intsete{0,n}$, $\set{s_i,s_{i+1}} \in E$.
					Considérons $s' = (s_n,s_{n-1},...,s_1,s_0)$. $s'$ est une chaîne reliant $y$ et $x$.
					En effet, $s'_0 = s_n = y$ et $s'_n = s_0 = x$, et $\forall i \in \intsete{0,n}, 
					\set{s'_{i},s'_{i+1}} = \set{s_{n-i},\set{n-i-1} = \set{s_k,s_{k+1}}} \in E$ en posant $k = n-i-1 \in \intsete{0,n}$.
				\item Soit $(x,y,z) \in V^3$ tel que $x\mathcal{R}y$ et $y\mathcal{R}z$.
					Comme $x\mathcal{R}y$, il existe $s = (s_0,...,s_n) \in V^{n+1}$ une chaîne avec $s_0 = x$ et $s_n = y$.
					Comme $y\mathcal{R}x$, il existe $t = (t_0,...,t_m) \in V^{m+1}$ une chaîne avec $t_0 = y$ et $t_m = z$.
					Considérons $u = (s_0,...s_n,t_1,...t_m)$.
					$u$ est bien une chaîne car $\forall i \in \intsete{0,n+m}$, soit $i \in \intsete{0,n}$ et dans ce cas on a 
					$\set{u_i, u_{i+1}} = \set{s_i,s_{i+1}} \in A$, soit $i = n$ et on a $\set{u_i,u_{i+1}} = \set{t_0,t_1} \in E$,
					soit $i \in \inteset{n,n+m}$ et $\set{u_i,u_{i+1}} = \set{t_{i-n},t_{i-n+1}} \in E$.
					On a par ailleurs $u_0 = x$ et $u_{m+n} = z$, donc $x\mathcal{R}z$.
			\end{demo}

			\exo Définir une relation d'équivalence similaire pour $H$, où l'on doit avoir un chemin dans chaque sens entre deux points en relation.

			\definition \textbf{Une composante connexe} de $G$ est une classe d'équivalence pour la relation d'équivalence définie ci-dessus.
			Si $G$ n'admet qu'une composante connexe, on dit que $G$ est un graphe connexe. 
			Dans le cas de la relation d'équivalence sur les graphes orientés, on appelle les classes d'équivalences \textbf{composante fortement connexe}.

			\definition Soit $W \subset V$ avec $W \neq \varnothing$. 
			$W$ est \textbf{convexe} ssi $\forall(x,y) \in W^2$, il existe une chaîne reliant $x$ et $y$.

			\prop{$W$ est une composante connexe ssi $W$ est connexe minimal, 
			c'est à dire si $\forall W' \subset V\setminus\set{W}, W \subset W'$, $W'$ n'est pas connexe.}

			\definition Soit $G' = (V',E')$ un graphe.
			$G'$ est un \textbf{sous-graphe} ssi $V' \subset V$, $E' \subset E$.

			\definition Soit $V' \subset V$
			Le \textbf{graphe induit par $G$ sur $V'$} est $G' = (V', \set{\set{x,y}\in E \tq (x,y) \in V^2})$.

\end{document}
