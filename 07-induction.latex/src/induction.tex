\newcommand\PATH{Lancez `make adapt avant de compiler!`}
\documentclass{scrartcl}

% Fichier pour les dépendances

% Fontes Computer Modern
\usepackage{lmodern}

% Minted
\usepackage[cache=true,outputdir=\PATH/obj]{minted}

% Pour les symboles et outils mathématiques
\usepackage{amsmath}
\usepackage{amsfonts}
\usepackage{amssymb}
\usepackage{mathtools}
\usepackage{cancel}
\usepackage{mathrsfs}
\usepackage{stmaryrd}

% Pour les dessins / images
\usepackage{xcolor}
\usepackage{tikz}
\usepackage{graphicx}

% Tableaux et listes
\usepackage{tabularx}
\newcolumntype{Y}{>{\centering\arraybackslash}X}
\usepackage{booktabs}
\usepackage{multirow}
\usepackage{enumitem}
\usepackage{multicol}
\usepackage{adjustbox}

% Présentation de la page (bordures)
\usepackage[a4paper,top=1cm,bottom=1cm,includefoot,left=1.5cm,right=1.5cm,footskip=1cm]{geometry}
\usepackage{scrlayer-scrpage}
\rofoot*{\pagemark}
\cofoot*{}
\pagestyle{scrheadings}

% Autres
\usepackage{ulem} % Soulignage
\usepackage[french]{babel} % Babel, pour respecter les standards français
\frenchbsetup{StandardLists=true} % ... sauf pour les immondes listes à tirets
\usepackage[T1]{fontenc}

\usepackage{hyperref} % Pour les liens
\hypersetup{
    colorlinks=true,
    linkcolor=blue,
    filecolor=magenta,
    urlcolor=cyan,
}


%% Commandes personnalisées

% Code
\newenvironment{code}[1]
{\VerbatimEnvironment\begin{minted}[breaklines,frame=single,autogobble,linenos,bgcolor=codebg,tabsize=4,mathescape=true]{#1}}
{\end{minted}}
\definecolor{codebg}{gray}{1}
\newenvironment{algotext}
{\VerbatimEnvironment\begin{minted}[frame=single,autogobble,linenos,bgcolor=codebg,tabsize=4,escapeinside=||,mathescape=true]{text}}
{\end{minted}}


% Paragraphes spéciaux
\newcounter{compteurExos}
\setcounter{compteurExos}{0}
\newcommand{\plabel}[1]{{\parindent10pt \fontfamily{cmss}\selectfont\textbf{#1} \,}}

\newcommand{\prop}[1]{\plabel{Propriété :} \textsl{#1}}
\newcommand{\lemma}[1]{\plabel{Lemme :} \textsl{#1}}
\newcommand{\rem}{\plabel{Remarque :}}
\newcommand{\exemple}{\plabel{Exemple :}}
\newcommand{\exo}{\stepcounter{compteurExos}\plabel{Exercice \thecompteurExos :}}
\newcommand{\obs}{\plabel{Observation :} }

\newenvironment{demo}{\begin{itemize}[label=$\triangleright$]}{\end{itemize}}
\newcommand{\definition}{\parindent0pt }
\newcommand{\semidef}{\parindent0pt $\bullet$ \,}

% Outils
\newcommand{\corrpar}{\vspace{-20pt}}
\newcommand{\extraspace}{\vspace{5pt}}

% Configuration
\parskip5pt

% Annonce pour le développement:
\newcommand{\notecentrale}[1]{\begin{center}\textsl{#1}\end{center}}
\newcommand{\warnwip}{\notecentrale{Ce cours n'est pas encore bien relu}}
\newcommand{\warnwipnext}{\notecentrale{La suite de ce cours est encore en rédaction}\message}


\newcommand{\set}[1]{{\{#1\}}}
\newcommand{\intset}[1]{\llbracket #1 \rrbracket}
\newcommand{\intsete}[1]{\llbracket #1 \llbracket}
\newcommand{\inteset}[1]{\rrbracket #1 \rrbracket}
\newcommand{\intesete}[1]{\rrbracket #1 \llbracket}

\newcommand{\tq}{\, \big| \,}

\newcommand{\card}{\.\textmd{card}}
\newcommand{\Id}{\textsl{Id}}
\newcommand{\supp}{\text{supp}}

% Partie entière, arrondi supérieur

\DeclarePairedDelimiter\ceil{\lceil}{\rceil}
\DeclarePairedDelimiter\floor{\lfloor}{\rfloor}



\newcommand{\fpq}{\mathbb{F}_p(Q)}
\newcommand{\abins}{\mathscr{A}_B(\mathcal{S})}
\newcommand{\pb}[3]
{
	\textbf{#1}
	\left|\left|\begin{tabular}{l l}
		Entrée: & #2 \\
		\vspace{-10pt} & \\
		\hline
		\vspace{-10pt} & \\
		Sortie: & #3
	\end{tabular}\right.\right.
}
\newcommand{\pbinlist}[3]
{
	\textbf{#1} &  $\left|\left|\begin{tabular}{l l}
		Entrée: & #2\\ 
		\vspace{-10pt} & \\
		\hline
		\vspace{-10pt} & \\
		Sortie: & #3 
	\end{tabular}\right.\right.$ 
}
\newcommand{\val}{\text{val}}
\newcommand{\argmax}{\mathop{\text{argmax}}}
\newcommand{\argmin}{\mathop{\text{argmin}}}


\title{Relations}
\author{}
\date{}

\begin{document}
	\maketitle
	\section{Généralités}
		Considérons un ensemble non vide $X$. Pour $n\in\mathbb{N}$, on appelle relation $n$-aire sur $X$
		toute partie $\mathcal{R}$ de $X^n$. 

		\begin{center}\textsl{On ne s'intéressera dans ce chapitres qu'aux relations binaires}\end{center}

		Pour deux éléments $x,y$ de $X$, on note $x\mathcal{R}y$ et on dit que $x$ est en relation avec $y$ si et seulement si
		$(x,y)\in \mathcal{R}$.

		\begin{center}\textsl{On considèrera désormais que $X$ est un ensemble non vide et que $\mathcal{R}$ est une relation binaire sur $X$}\end{center}

		On dit que $\mathcal{R}$ est:
		\begin{itemize}
			\item réflexive ssi $\forall x \in X^2: x\mathcal{R}x$
			\item transitive ssi $x\mathcal{R}y$ et $y\mathcal{R}z \Rightarrow x\mathcal{R}z$
			\item symétrique ssi $x\mathcal{R}y \Rightarrow y\mathcal{R}x$
			\item antisymétrique ssi $x\mathcal{R}y$ et $y\mathcal{R}x \Rightarrow x = y$ 
		\end{itemize}

		On vérifie aisément que pour toute partie $Y$ de $X$, $\mathcal{R}\cap Y^2$ vérifie les mêmes propriétés que $X$ sur $Y$.

	\section{Composition de relations}
		Soient $\mathcal{R}$ et $\mathcal{R}'$ deux relations binaires sur $X$. 
		On définit la composée de $\mathcal{R}$ et $\mathcal{R}'$, notée $\mathcal{R} \circ \mathcal{R}'$ par:
		\[
			\mathcal{R}\circ\mathcal{R}' = \left\{(x,y) \in X^2 \; \big| \; \exists z \in X: x\mathcal{R}z \text{ et } z\mathcal{R}'y \right\}
		\]
		
		\paragraph{Propriété} \textsl{$\circ$ est associatif et possède pour neutre $\Delta_X = \left\{(x,x) \big| x\in X\right\}$}
		\begin{demo}
			\item [$\triangleright$] Fastidieux mais sans difficulté...
		\end{demo}

		On définit par récurrence $\mathcal{R}^n = \underbrace{\mathcal{R}\circ\mathcal{R}\circ...\circ\mathcal{R}}_{n \text{ fois}}$.
		On conviendra que $\mathcal{R}^0 = \Delta_X$. 

		\paragraph{Propriété} \textsl{Soient $\mathcal{R}$ et $\mathcal{R'}$ deux relations binaires symétriques de $X$. 
		On a $x\left(\mathcal{R}\circ\mathcal{R'}\right)y \Leftrightarrow y\left(\mathcal{R'}\circ\mathcal{R}\right)x$.}
		\begin{demo}
			\item Soit $(x,y)\in X^2$ tel que $x\left(\mathcal{R}\circ\mathcal{R}'\right)y$.\\
			Il existe alors $z \in X$ tel que $x\mathcal{R}z$ et $z\mathcal{R'}y$. 
			Comme $\mathcal{R'}$ et $\mathcal{R}$ est symétrique, on a aussi $z\mathcal{R}x$ et $y\mathcal{R'}z$,
			d'où $y\left(\mathcal{R'}\circ\mathcal{R}\right)x$
		\end{demo}

		\paragraph{Remarque} En particulier, si $\mathcal{R}$ est symétrique, $\mathcal{R}^n$ l'est aussi pour tout entier naturel $n$.

	\section{Fonctions}
		On dit que $\mathcal{R}$ définit une fonction ssi $x\mathcal{R}y$ et $x\mathcal{R}y' \Rightarrow y = y'$

		\paragraph{Propriété} \textsl{Si $\mathcal{R}$ définit une fonction $f$ et $\mathcal{R}'$ définit une fonction $g$, 
			alors $\mathcal{R}\circ\mathcal{R}'$ définit la fonction $g\circ f$.}
		\begin{demo}
			\item Soit $(x,y,y') \in X^3$, 
			tel que $x\mathcal{R}\circ\mathcal{R'}y$ et $x\mathcal{R}\circ\mathcal{R'}y'$. \\
			Par définition, il existe $z \in X$ tel que $x\mathcal{R}z$ et $z\mathcal{R}'y$, 
			et $z' \in X$ tel que $x\mathcal{R}z'$ et $z'\mathcal{R}'y'$.\\
			Comme $x\mathcal{R}z$ et $\mathcal{R}z'$, on a $z=z'=f(x)$ car $\mathcal{R}$ définit la fonction $f$. \\
			Alors on a $f(x)\mathcal{R}'y$ et $f(x)\mathcal{R}'y'$ d'où $y=y'=g\circ f(x)$ car $\mathcal{R}'$ définit la fonction $g$.
		\end{demo}

	\section{Étude des propriétés de relations binaires}
		\subsection{Composition de relations binaires}
			Comme pour tout couple d'entiers naturels $(m,n)$, 
			on a $\mathcal{R}^n\circ\mathcal{R}^m = \mathcal{R}^m\circ\mathcal{R}^n$ (cela découle simplement de l'associativité de $\circ$), 
			on peut définir la fermeture (ou clôture) transitive de la relation binaire $\mathcal{R}$ par:
			\[
				\mathcal{R_T} = \bigcup\limits_{n\geq 1} \mathcal{R}^n
			\]
		
			\paragraph{Lemme} \textsl{Soit $\mathcal{R}'$ une relation binaire transitive sur $X$ contenant $\mathcal{R}$. 
			Alors, pour tout entier naturel $n \geq 1$, $\mathcal{R}^n \subset R'$}	
			\begin{demo}
				\item Par récurrence sur $n$. 
				C'est vrai pour $n=1$ car $\mathcal{R}\subset\mathcal{R}'$.
				Supposons que ce soit vrai pour $n\in\mathbb{N}^*$ quelconque.
				Soit $(x,y)\in \mathcal{R}^{n+1}$. 
				Alors il existe $z \in X$ tel que $x\mathcal{R}^nz$ et $z\mathcal{R}^ny$.
				Comme par hypothèse $\mathcal{R}^n \subset \mathcal{R}'$, on a $z\mathcal{R}'y$, 
				et comme $\mathcal{R} \subset \mathcal{R}'$, $x\mathcal{R}'z$. Par transitivité, on a $x\mathcal{R}'y$,
				ce qui prouve l'inclusion et donc la proposition au rang $n+1$.
			\end{demo}

			\paragraph{Propriété} \textsl{$\mathcal{R_T}$ est la plus petite (au sens de l'inclusion) relation binaire transitive sur $X$ contenant $R$}
			\begin{demo}
				\item Montrons que $\mathcal{R_T}$ est transitive:
				Soient $x$, $y$ et $z$ dans $X$ tels que $x\mathcal{R_T}y$ et $y\mathcal{R_T}z$.
				Alors il existe $(m,n)\in\mathbb{N}^2$ tels que $x\mathcal{R}^my$ et $y\mathcal{R}^nz$.
				On a donc $x\mathcal{R}^m\circ\mathcal{R}^nz$ i.e $x\mathcal{R}^{m+n}z$, d'où $x\mathcal{R_T}z$.
				\item Montrons que $\mathcal{R_T}$ est minimale:
				Considérons $\mathcal{R}'$ une relation binaire transitive contenant $\mathcal{R}$. Soit $(x,y) \in \mathcal{R_T}$.
				Il existe donc $n\in\mathbb{N}^*$ tel que $x\mathcal{R}^ny$. D'après le lemme, $\mathcal{R}^n\subset\mathcal{R}'$,
				donc $(x,y)\in\mathcal{R}'$.
			\end{demo}

		\subsection{Relations d'équivalence}
			On dit que $\mathcal{R}$ est une relation d'équivalence sur $X$ ssi elle est réflexive, transitive et symétrique.
			On note classe d'équivalence de $a\in X$ l'ensemble $\dot{a} = \left\{x\in X \big| x\mathcal{R}a\right\}$. On dit
			alors que $a$ est un représentant de $\dot{a}$

			\paragraph{Propriété} \textsl{Soient $x$ et $y$ deux éléments de $X$. 
				Soit $y\in\dot{x}$ et alors $\dot{x} = \dot{y}$, soit $\dot{x}\cap\dot{y} = \varnothing$}
			\begin{demo}
				\item On a forcément $y \in \dot{x}$ ou $y\not\in\dot{x}$. Dans le premier cas,
				$y$ est en relation avec $x$, donc par transitivité tout élément en relation avec $y$ sera en relation avec
				$x$, d'où $\dot{y} \subset \dot{x}$. L'autre inclusion découle de la symétrie de $\mathcal{R}$. Dans le second
				cas, si $\dot{x}\cap\dot{y} \neq \varnothing$, il existerait $a\in X$ tel que $a\mathcal{R}y$ et $a\mathcal{R}x$.
				Par symétrie et transitivité, on aurait alors $y\mathcal{R}x$, ce qui contredit $y\not\in\dot{x}$.
			\end{demo}

			L'ensemble des classes d'équivalences de $X$ par $\mathcal{R}$ est appelé ensemble quotient de $X$ par $\mathcal{R}$,
			et est noté $X/R$

			\paragraph{Corrolaire} \textsl{L'ensemble des classes d'équivalence par $\mathcal{R}$ forme une partition de $X$}
			\begin{demo}
				\item Notons $H = \bigcup\limits_{x\in X} \dot{x}$. $H \subset X$ en tant qu'union de parties
				de $X$, et $X \subset H$ car tout élément de $X$ appartient à sa propre classe d'équivalence. 
				Cela prouve que l'ensemble des classes d'équivalence forme un recouvrement de $X$.
				Prenons alors deux classes d'équivalences $\dot{x}$ et $\dot{y}$. 
				D'après le lemme, on a soit $\dot{x}=\dot{y}$, soit $\dot{x}\cap\dot{y} = \varnothing$, 
				ce qui prouve que ce recouvrement est bien disjoint.
			\end{demo}

			Soient $X$ et $Y$ deux ensembles non vides. Soit $\mathcal{R}$ une relation d'équivalence sur $X$.
			On dit que $f\in\mathcal{F}(X,Y)$ passe au quotient si elle est constante sur les classes d'équivalence,
			c'est à dire si $x\mathcal{R}y \Rightarrow f(x)=f(y)$.

			On peut alors définir $\bar{f}\in\mathcal{F}(X/R,Y)$ qui à une classe d'équivalence $\dot{x}$ associe
			l'image d'un de ses représentants par $f$.

			\paragraph{Remarque} Quelque part, $\dot{f}$ "gagne" en injectivité par rapport à $f$, puisqu'il y a "moins" d'éléments
			ayant la même image.
			\paragraph{Remarque} On a $f(X) = \bar{f}(X/R)$
		
		\subsection{Relations d'ordre}
			On dit que $\mathcal{R}$ est une relation d'ordre sur $X$ ssi elle est réflexive, transitive et antisymétrique.
			\begin{center}\textsl{On notera désormais une telle relations $\leq$}\end{center}
			Un ensemble $(X,\leq)$ muni d'une telle relation est dit ordonné. Pour une partie $Y$ de $X$,
			on note $(Y,\leq)$ l'ensemble ordonné induit (c'est à dire $Y$ muni de la relation $\leq$ restreite à $Y$).

			On dit que $\mathcal{R}$ est une relation d'ordre totale sur $Y$ si et seulement si pour tout couple
			$(x,y)$ d'éléments de $Y$, on a $x\mathcal{R}y$ ou $y\mathcal{x}$.

			On note $x < y$ ssi $x \leq y$ et $x \neq y$, $x \geq y$ ssi $y \leq x$ et $x > y$ ssi $y < x$.
\end{document}
