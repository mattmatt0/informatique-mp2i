\newcommand\PATH{Lancez `make adapt avant de compiler!`}

\RequirePackage{fix-cm}

\documentclass{scrartcl}

% Fichier pour les dépendances

% Fontes Computer Modern
\usepackage{lmodern}

% Minted
\usepackage[cache=true,outputdir=\PATH/obj]{minted}

% Pour les symboles et outils mathématiques
\usepackage{amsmath}
\usepackage{amsfonts}
\usepackage{amssymb}
\usepackage{mathtools}
\usepackage{cancel}
\usepackage{mathrsfs}
\usepackage{stmaryrd}

% Pour les dessins / images
\usepackage{xcolor}
\usepackage{tikz}
\usepackage{graphicx}

% Tableaux et listes
\usepackage{tabularx}
\newcolumntype{Y}{>{\centering\arraybackslash}X}
\usepackage{booktabs}
\usepackage{multirow}
\usepackage{enumitem}
\usepackage{multicol}
\usepackage{adjustbox}

% Présentation de la page (bordures)
\usepackage[a4paper,top=1cm,bottom=1cm,includefoot,left=1.5cm,right=1.5cm,footskip=1cm]{geometry}
\usepackage{scrlayer-scrpage}
\rofoot*{\pagemark}
\cofoot*{}
\pagestyle{scrheadings}

% Autres
\usepackage{ulem} % Soulignage
\usepackage[french]{babel} % Babel, pour respecter les standards français
\frenchbsetup{StandardLists=true} % ... sauf pour les immondes listes à tirets
\usepackage[T1]{fontenc}

\usepackage{hyperref} % Pour les liens
\hypersetup{
    colorlinks=true,
    linkcolor=blue,
    filecolor=magenta,
    urlcolor=cyan,
}


%% Commandes personnalisées

% Code
\newenvironment{code}[1]
{\VerbatimEnvironment\begin{minted}[breaklines,frame=single,autogobble,linenos,bgcolor=codebg,tabsize=4,mathescape=true]{#1}}
{\end{minted}}
\definecolor{codebg}{gray}{1}
\newenvironment{algotext}
{\VerbatimEnvironment\begin{minted}[frame=single,autogobble,linenos,bgcolor=codebg,tabsize=4,escapeinside=||,mathescape=true]{text}}
{\end{minted}}


% Paragraphes spéciaux
\newcounter{compteurExos}
\setcounter{compteurExos}{0}
\newcommand{\plabel}[1]{{\parindent10pt \fontfamily{cmss}\selectfont\textbf{#1} \,}}

\newcommand{\prop}[1]{\plabel{Propriété :} \textsl{#1}}
\newcommand{\lemma}[1]{\plabel{Lemme :} \textsl{#1}}
\newcommand{\rem}{\plabel{Remarque :}}
\newcommand{\exemple}{\plabel{Exemple :}}
\newcommand{\exo}{\stepcounter{compteurExos}\plabel{Exercice \thecompteurExos :}}
\newcommand{\obs}{\plabel{Observation :} }

\newenvironment{demo}{\begin{itemize}[label=$\triangleright$]}{\end{itemize}}
\newcommand{\definition}{\parindent0pt }
\newcommand{\semidef}{\parindent0pt $\bullet$ \,}

% Outils
\newcommand{\corrpar}{\vspace{-20pt}}
\newcommand{\extraspace}{\vspace{5pt}}

% Configuration
\parskip5pt

% Annonce pour le développement:
\newcommand{\notecentrale}[1]{\begin{center}\textsl{#1}\end{center}}
\newcommand{\warnwip}{\notecentrale{Ce cours n'est pas encore bien relu}}
\newcommand{\warnwipnext}{\notecentrale{La suite de ce cours est encore en rédaction}\message}


\newcommand{\algofor}[3]{\textbf{Pour } #1 \textbf{ allant de } #2 \textbf{à} #3 :\\}
\newcommand{\algoreturn}[1]{\textbf{Renvoyer } #1 \\}
\newcommand{\algotrue}{\texttt{Vrai}}
\newcommand{\algofalse}{\texttt{Faux}}
\newcommand{\algotry}{\textbf{Essayer} : \\}
\newcommand{\algocatch}[1]{\textbf{Rattrappr} #1 \\}
\newcommand{\algoif}[1]{\textbf{Si } #1 : \\}
\newcommand{\algoelif}[1]{\textbf{Sinon si} #1 \textbf{alors} }
\newcommand{\tab}{.\quad}

\newcommand{\set}[1]{{\{#1\}}}
\newcommand{\intset}[1]{\llbracket #1 \rrbracket}
\newcommand{\intsete}[1]{\llbracket #1 \llbracket}
\newcommand{\inteset}[1]{\rrbracket #1 \rrbracket}
\newcommand{\intesete}[1]{\rrbracket #1 \llbracket}

\newcommand{\tq}{\, \big| \,}

\newcommand{\card}{\.\textmd{card}}
\newcommand{\Id}{\textsl{Id}}
\newcommand{\supp}{\text{supp}}

% Partie entière, arrondi supérieur

\DeclarePairedDelimiter\ceil{\lceil}{\rceil}
\DeclarePairedDelimiter\floor{\lfloor}{\rfloor}



\newcommand{\fpq}{\mathbb{F}_p(Q)}
\newcommand{\abins}{\mathscr{A}_B(\mathcal{S})}
\newcommand{\pb}[3]
{
	\textbf{#1}
	\left|\left|\begin{tabular}{l l}
		Entrée: & #2 \\
		\vspace{-10pt} & \\
		\hline
		\vspace{-10pt} & \\
		Sortie: & #3
	\end{tabular}\right.\right.
}
\newcommand{\pbinlist}[3]
{
	\textbf{#1} &  $\left|\left|\begin{tabular}{l l}
		Entrée: & #2\\ 
		\vspace{-10pt} & \\
		\hline
		\vspace{-10pt} & \\
		Sortie: & #3 
	\end{tabular}\right.\right.$ 
}
\newcommand{\val}{\text{val}}
\newcommand{\argmax}{\mathop{\text{argmax}}}
\newcommand{\argmin}{\mathop{\text{argmin}}}


\usepackage{lmodern}

\title{Le problème SAT}
\author{}
\date{}

\begin{document}
	\maketitle
		\begin{center}\textsl{Dans ce chapitre, on s’intéresse à la satisfiabilité des formules (sur un ensemble de variables fini, i.e. $\card (Q) \in \mathbb{N}$).
		En effet, une formule propositionnelle peut modéliser un problème concret dont une solution serait donnée par un environnement satisfaisant la formule.}\end{center}

	\section{Définitions}
		On s’intéresse au problème de décision de savoir si une formule est satisfiable, sans demander par quel environnement.
		\[
			\pb{SAT}{$A \in \fpq$}{oui si $A$ est satifsiable, non sinon.}
		\]

		\rem Puisque la satisfiabilité d’une formule ne dépend que de sa classe, 
		on peut résoudre le problème sur une formule ayant une forme particulière, 
		mais attention au coût de transformation. 

		Ce problème a plusieurs variantes:
		\begin{center}
			\begin{tabular}{r l}
				\pbinlist{FND-SAT}{$A \in \fpq$ sous FND.}{oui si $A$ est satisfiable, non sinon.} \\[20pt]
				\pbinlist{FNC-SAT}{$A \in \fpq$ sous FNC.}{oui si $A$ est satisfiable, non sinon.} \\[20pt]
				\pbinlist{3-SAT}{$A \in \fpq$ sous FNC avec des clauses de 3 littéraux seulement.}{oui si $A$ est satisfiable, non sinon.} \\[20pt]
				\pbinlist{2-SAT}{$A \in \fpq$ sous FNC avec des clauses de 2 littéraux seulement.}{oui si $A$ est satisfiable, non sinon.} 
			\end{tabular}
		\end{center}

	\section{Réductions}
		Soient $X$ et $Y$ deux problèmes de décisions.
		On note $X^+$ (resp. $X^{-}$) les instances positives (resp. négatives) de $X$, 
		i.e. celles pour lesquelles la solution est vrai (resp. faux). 
		On définit de même $Y^+$ et $Y^{-}$.

		\definition On dit que \textbf{$Y$ se réduit à $X$ en temps polynomial} s'il existe une transformation $\varphi$,
		calculable en temps polynomial, qui transforme toute instance de $Y$ en une instance de $X$ de sorte que:
		\[
			\forall \mathcal{I}, \quad \varphi(\mathcal{I}) \in X^+ \Leftrightarrow \mathcal{I} \in Y^+
		\]
		
		On a alors, pour les problèmes définis précédemment:
		\begin{itemize}
			\item \textbf{SAT} est plus dur que \textbf{FND-SAT} et \textbf{FNC-SAT}:
			\begin{itemize}
				\item \textbf{FNC-SAT} se réduit en temps polynomial à \textbf{SAT} (pour $\varphi = \Id$).
				\item \textbf{FND-SAT} se réduit en temps polynomial à \textbf{SAT} (pour $\varphi = \Id$).
			\end{itemize}
			\item \textbf{FNC-SAT} est plus dur que \textbf{2-SAT} et \textbf{3-SAT}:
			\begin{itemize}
				\item \textbf{2-SAT} se réduit en temps polynomial à \textbf{FNC-SAT} (pour $\varphi = \Id$).
				\item \textbf{3-SAT} se réduit en temps polynomial à \textbf{FNC-SAT} (pour $\varphi = \Id$).
			\end{itemize}
		\end{itemize}

		\prop{\textbf{FNC-SAT} se réduit en temps polynomial à \textbf{3-SAT}.}
		\begin{demo}
			\item On transforme chaque clause de taille inférieure à 3 en clause de taille 3:
			\begin{center}\begin{tabular}{c c c}
				$a \equiv (a \vee a \vee a)$ & \quad & $(a \vee b) \equiv (a \vee a \vee b)$
			\end{tabular}\end{center}
			\item On transforme chaque clause de taille supérieure à 3 en conjonction de clauses de taille 3 en ajoutant des nouvelles variables:
			\[
				(a \vee b \vee c \vee d) \longrightarrow (a \vee b \vee z) \wedge ((\neg z) \vee c \vee d) 
			\]
			\[
				(a \vee b \vee c \vee d \vee e) \longrightarrow (a \vee b \vee z_1) \wedge ((\neg z_1) \vee c \vee z_2) \wedge ((\neg z_2) \vee d \vee e) 
			\]
			\item Toutes ces transformations sont en temps polynomial et préservent la satisfiablilité.
		\end{demo}

		\exo Soit $C = \ell_1 \vee \ell_2 \vee \ell_3 \vee ... \vee \ell_k$ une clause de $\fpq$ de taille $k > 3$.
		Soit $z$ une nouvelle variable (i.e $z \not\in Q$). 
		On pose alors $C_1 = (\ell_1 \vee \ell_2 \vee z)$ et $C_2 = ((\neg z) \vee \ell_3 \vee ... \vee \ell_k)$.
		Montrer les deux propositions suivantes:
		\begin{itemize}
			\item $\forall \rho \in \mathbb{B}^Q$, $[C]^\rho = V \Rightarrow 
				(\exists \rho' \in \mathbb{B}^{Q\cup\set{z}}, \rho'\big|_Q = \rho \text{ et } [C_1\wedge C_2]^{\rho'} = V)$
			\item $\forall \rho' \in \mathbb{B}^{Q\cup\set{z}}, [C_1\wedge C_2]^{\rho'} = V \Rightarrow [C]^{\rho'} = V$
		\end{itemize}

		\rem \textbf{FNC-SAT} ne se réduit pas en temps polynomial à \textbf{2-SAT}. 
		On verra que \textbf{2-SAT} se résout en temps polynomial (cf. chapitre sur les graphes), 
		que \textbf{3-SAT} est NP-complet (théorème de Cook, au programme de seconde année),
		et qu'à moins que P = NP, ces deux problèmes ne sont pas équivalents.

		\rem \textbf{FNC-SAT} ne se réduit pas non plus en temps polynomial à \textbf{FND-SAT}:
		La transformation d'une forme à l'autre se fait en temps exponentiel et peut aussi augmenter la taille de l'entrée de manière exponentielle.
		On verra que \textbf{FND-SAT} se résout en temps polynomial (juste après).
		À moins que P = NP, ces deux problèmes ne sont pas équivalents.

	\section{Modéliser une FND}
		On rappelle qu'une formule sous FND est une disjonction de conjonction de littéraux. D'un point de vue sémantique, c'est à dire dans $\fpq$,
		\begin{itemize}
			\item L'ordre des littéraux au sein d'une conjonction n'a pas d'importance, ni leur multiplicité:
				une conjonction est satisfaite par un environnement propositionnel ssi il satisfait l’ensemble de ses termes.
			\item L'ordre des conjonctions n'a pas non plus d'importance au sein de la disjonction, ni leur multiplicité:
				une conjonction est satisfaite par un environnement propositionnel ssi il satisfait l'un de ses termes.
			\item Pour modéliser les conjonctions de littéraux, il est intéressant d'utiliser un \textbf{ensemble de littéraux} ou un \textbf{couple d'ensembles de variables}:
				d'une part celles apparaissant dans les littéraux positifs, d'autre part celles apparaissant dans les littéraux négatifs.
			\item Pour modéliser une FND, il faut alors utiliser un ensemble de conjonctions. 
				Quant à la structure de données à employer, une liste de littéraux convient.
		\end{itemize}
	
	\section{\textbf{FND-SAT} et \textbf{FND-SAT}}
		On suppose que $Q = \set{q_1,q_2,...,q_N}$. Avec la modélisation décrite précédemment, les deux problèmes deviennent:
		\[
			\pb{FND-SAT}{$((\ell_{i,j})_{i\in\intset{1,n_j}})_{j\in\intset{1,n}}$ une famille de littéraux de $Q$}{oui si $A = \bigvee_{j=1}^n\bigwedge_{i=1}^{n_j}\ell_{i,j}$ est satisfiable, non sinon.}
		\]
		\[
			\pb{FNC-SAT}{$((\ell_{i,j})_{i\in\intset{1,n_j}})_{j\in\intset{1,n}}$ une famille de littéraux de $Q$}{oui si $A = \bigwedge_{j=1}^n\bigvee_{i=1}^{n_j}\ell_{i,j}$ est satisfiable, non sinon.}
		\]
		\exemple Prenons $(a\wedge b \wedge (\neg c)) \vee ((\neg b) \wedge (\neg c)) \vee (a \wedge c)$.
		Mise sous forme d'ensemble d'ensembles de littéraux, la formule devient:
		\[ \set{\set{a,b,(\neg c)},\set{(\neg b,\neg c)},\set{a,c}} \]
		Soit encore, sous forme d'ensemble de couples:
		\[ \set{(\set{a,b},\set{c}),(\varnothing, \set{b,c}), (\set{a,c},\varnothing)} \]

		\exemple Prenons $(a\wedge b \wedge (\neg c)) \vee (a \wedge c \wedge a) \vee ((\neg b) \wedge (\neg c))$.
		Mise sous forme d'ensemble d'ensembles de littéraux, la formule devient:
		\[ \set{\set{a,b,\neg c},\set{(\neg b,\neg c)},\set{a,c}} \]
		Soit encore, sous forme d'ensemble de couples:
		\[ \set{(\set{a,b},\set{c}),(\varnothing, \set{b,c}), (\set{a,c},\varnothing)} \]

		\exemple Prenons $(a\wedge b \wedge (\neg a)) \vee (a \wedge c \wedge (\neg a)) \vee ((\neg b) \wedge (\neg c))$.
		Mise sous forme d'ensemble d'ensembles de littéraux, la formule devient:
		\[ \set{\set{a,b,\neg a},\set{(a, \neg a,\neg c)},\set{\neg b, \neg c}} \]
		Soit encore, sous forme d'ensemble de couples:
		\[ \set{(\set{a,b},\set{a}),(\set{a}, \set{a,c}), (\varnothing,\set{b,c})} \]

	\section{Algorithme pour FND-SAT}
		Considérons $l = (l_{1,1} \wedge ...\wedge l_{1,n_1}) \vee ... \vee (l_{k,1} \wedge ... \wedge l_{k,n_k})$,
		contenant $N$ variables propositionnelles $q_1,...,q_N$.

		\parindent0pt\algofor{$i$}{1}{$k$}
		\tab	Créer $T$ un tableau indicé par $\intset{1,N}$ initialisé à -1 \\
		\tab	\algotry 
		\tab	\tab	\algofor{$j$}{1}{$n_i$}
		\tab	\tab	\tab	\algoif{$\ell_{i,j}$ = $q_p$}
		\tab	\tab	\tab	\tab	\textbf{alors Si} $T[p]$ = -1, \textbf{alors} $T[p] \leftarrow 1$ \\
		\tab	\tab	\tab	\tab \algoelif{$T[p]$ = 0} déclencher "\textit{conj. non sat.}" \\
		\tab	\tab	\tab	\algoif{$\ell_{i,j} = \neg q_p$}
		\tab	\tab	\tab	\tab	\textbf{alors Si} $T[p]$ = -1, \textbf{alors} $T[p] \leftarrow 0$ \\
		\tab	\tab	\tab	\tab	\algoelif{$T[p] = 1$} déclencher "\textit{conj. non sat.}" \\
		\tab	\tab	\algoreturn{\algotrue} \\
		\tab	\algocatch{"\textit{conj. non sat.}"}
		\algoreturn{\algofalse}
	\section{Modélisation}
		\subsection{Exemple du solitaire}
			\plabel{État inital du jeu:} Sur chaque case d'un damier carré de $m\times m$ cases, 
			il y a une pierre bleue ($b$), ou bien une pierre rouge ($r$) ou rien.
			
			\plabel{But du jeu:} Enlever des pierres de manière à ce que:
			\begin{itemize}
				\item Sur chaque colonne toutes les pierres soient de la même couleur.
				\item Sur chaque ligne, il y ait au moins une pierre.
			\end{itemize}
			
			\exo \begin{itemize}
				\item Formaliser le problème de décision consistant à savoir si une partie peut être gagnée.
				\item Réduire ce problème à \textbf{3-SAT}.
				\item Réduire \textbf{3-SAT} à ce problème (bonus).
			\end{itemize}
			
	\section{Algorithme de Quine}
		\subsection{Substitution}
			Soit $\sigma\in \mathcal{F}(Q,\fpq)$.
			On appelle \textbf{support de $\sigma$}, noté $\supp(\sigma)$ l'ensemble des variables que $\sigma$ n'envoie pas sur la formule réduite à elle-même.
			\[
				\supp(\sigma) = \set{q \in Q \tq \sigma(q) \neq q}
			\]

			\exemple Pour $\sigma = \left(\begin{tabular}{c c c}
				$Q=\set{x,y}$ & $\rightarrow$ & $\mathbb{F}_p(\set{x,y})$ \\
				$x$ & $\mapsto$ & $x$ \\
				$y$ & $\mapsto$ & $\neg x$
			\end{tabular}\right)$, $\supp(\sigma) = \set{y}$.

			\definition Soit $\sigma \in \mathcal{F}(Q,\fpq)$. On dit que $\sigma$ est une \textbf{substitution} ssi $\card(\supp(\sigma)) \in \mathbb{N}$.

			\definition Soit $\sigma \in \mathcal{F}(Q,\fpq)$ une substitution. 
			On appelle \textbf{application de la substitution $\sigma$} la fonction suivante:
			\[
				\bullet[\sigma] = 
				\left(\begin{tabular}{c c c}
					$\fpq$ & $\rightarrow$ & $\fpq$ \\
					$\bot$ & $\mapsto$ & $\bot$ \\
					$\top$ & $\mapsto$ & $\top$ \\
					$q$ & $\mapsto$ & $\sigma(q)$ \\
					$\neg A$ & $\mapsto$ & $\neg (A[\sigma])$ \\
					$A \alpha B$ & $\mapsto$ & $A[\sigma] \alpha B[\sigma]$ où $\alpha \in \set{\wedge,\vee,\rightarrow,\leftrightarrow}$
				\end{tabular}\right)
			\]
			Soit $\sigma_1$ et $\sigma_2$ deux substitutions. 
			On note $\sigma_1\cdot\sigma_2$ et on appelle \textbf{composée de $\sigma_1$ et $\sigma_2$} la fonction
			\[\sigma=\left(\begin{tabular}{c c c }
			$Q$ & $\rightarrow$ & $\fpq$ \\
			$q$ & $\mapsto$ & $\sigma_2(q)[\sigma_1]$
			\end{tabular}\right)\]

			\prop{La composée de substitutions est associative.}
			\begin{demo}
				\item En exercice.
			\end{demo}
			
			\rem Attention: une substition n'est pas forcément symétrisable! 
			Ainsi, l'ensemble des substitutions muni de la composition est juste un monoïde, mais pas un groupe. 

		\subsection{Règles de simplification}
			On a les règles suivantes:
			
			\begin{multicols}{3}
			\begin{itemize}
				\item $\neg \top \equiv \bot$ 
				\item $\neg \bot \equiv \top$ 
				\item $A \wedge \top \equiv \top \wedge A \equiv A$ 
				\item $A \wedge \bot \equiv \bot \wedge A \equiv \bot$ 
				\item $A \vee \top \equiv \top \vee A \equiv \top$ 
				\item $A \rightarrow \bot \equiv \neg A$ 
				\item $A \rightarrow \top \equiv A$ 
				\item $A \vee \bot \equiv \bot \vee A \equiv A$ 
				\item $\bot \rightarrow A \equiv T$ 
				\item $\top \rightarrow A \equiv A$ 
				\item $A \leftrightarrow \top \equiv A$ 
				\item $A \leftrightarrow \bot \equiv \neg A$ 
			\end{itemize}
			\end{multicols}		

			On remarque que l'application de chacune de ses règles amène à une réduction stricte de la taille de la formule, 
			soit en enlevant un opérateur, soit en enlevant une variable, soit en enlevant un symbole $\bot$ ou $\top$.

			\prop{Soit $(A,B,A',B') \in \fpq^4$. Pour $\alpha \in \set{\wedge,\vee,\rightarrow,\leftrightarrow}$, 
				si $A \equiv A'$ et $B \equiv B'$, alors $A \alpha B \equiv A' \alpha B'$.}
			\begin{demo}
				\item Il suffit de revenir à la définition de l'équivalence. 
				Si $A \equiv A'$ alors pour tout environnement propositionnel $\rho$, $[A]^\rho = [A']^\rho$. 
				Comme $[A \alpha B]^\rho$ s'écrit $f([A]^\rho,[B]^\rho) = f([A']^\rho,[B']^\rho)$, on en déduit que 
				$[A \alpha B]^\rho = [A' \alpha B']^\rho$. Cela étant vrai pour tout $\rho$, on a bien $A\alpha B \equiv A' \alpha B'$.
			\end{demo}

			\exo Simplifier les expressions suivantes: 
			\begin{itemize}
				\item $((a \rightarrow b) \wedge (b \rightarrow c)) \rightarrow (a \rightarrow c)$
				\item $((s \rightarrow (b \vee t)) \wedge (((b \vee a) \rightarrow (r \wedge m)) \wedge (\neg r))) \rightarrow (s \rightarrow t)$
			\end{itemize}

\end{document}
