\newcommand\PATH{/home/matthieu/Documents/Ecole/Informatique/cours/1-relations-d-ordre.latex/obj}
\documentclass{scrartcl}

\usepackage{amsmath}
\usepackage{amsfonts}
\usepackage{amssymb}
\usepackage{mathtools}
\usepackage[cache=false,outputdir=\PATH]{minted}
\usepackage{xcolor}
\usepackage{tikz}
\usepackage{tabularx}
\usepackage{booktabs}
\usepackage{ltablex}
\usepackage{cancel}
\usepackage{graphicx}
\usepackage{mathrsfs}
\usepackage{ulem}

\usepackage[a4paper,
top=1.5cm,
bottom=1.5cm,
includefoot,
left=3cm,
right=3cm,
footskip=1cm]{geometry}
\usepackage{scrlayer-scrpage}
\rofoot*{\pagemark}
\cofoot*{}
\pagestyle{scrheadings}
		
\definecolor{codebg}{gray}{0.9}

\DeclarePairedDelimiter\ceil{\lceil}{\rceil}
\DeclarePairedDelimiter\floor{\lfloor}{\rfloor}

\newenvironment{code}[1]
{\VerbatimEnvironment\begin{minted}[autogobble,linenos,bgcolor=codebg,tabsize=4,mathescape=true]{#1}}
{\end{minted}}



\title{Ordres et induction}
\author{}
\date{}

\begin{document}
	\maketitle
	\section{Ordre}
		\subsection{Élements particuliers}
			Soit $x\in X$. On dit que:
			\begin{itemize}
				\item $x$ est un majorant de $Y$ ssi $\forall y \in Y, y\leq x$.
				\item $x$ est un minorant de $Y$ ssi $\forall y \in Y, x \leq y$.
			\end{itemize}
			Soit $y\in Y$. On dit que:
			\begin{itemize}
				\item $y$ est le plus grand élément de $Y$ ssi c'est un majorant de $Y$.
				\item $y$ est le plus petit élément de $Y$ ssi c'est un minorant de $Y$.
				\item $y$ est minimal ssi il n'existe pas de $y'$ dans $Y$ tel que $y'\leq y$
				\item $y$ est maximal ssi il n'existe pas de $y'$ dans $Y$ tel que $y\leq y'$
			\end{itemize}
			
			\paragraph{Remarque} Si le plus grand élément et le plus petit élément sont forcément uniques
			à cause de l'antisymétrie de la relation, les éléments minimaux et maximaux, eux, ne le sont
			pas toujours !

			\paragraph{Remarque} Cela dit, si l'ordre est total, il n'y a qu'un seul élément minimal, le plus
			petit élément (il en va de même pour les élément maximaux).

			On note Min($Y$) l'ensemble des minorants de $Y$, et Maj($Y$) l'ensemble de ses majorants.
			Le plus grand élément de Min($Y$), s'il existe, est appelé borne inférieure de $Y$. De la même façon,
			le plus petit élément de Max($Y$), s'il existe, est appelé borne supérieure de $Y$. 

		\subsection{Ordre bien fondé}
			On considère ici deux ensembles ordonnés, $(X,\leq)$ et $(Y,\preceq)$.
			%Par ailleurs, on note $x \geq y \Leftrightarrow y \leq x$, $x \succeq y \Leftrightarrow y \preceq x$,
			%$x > y \Leftrightarrow y < x$ et enfin $$

			Soit $f \in \mathcal{F}(X,Y)$. 
			\begin{itemize}
				\item $f$ est croissante ssi $\forall (x,x') \in X^2, x \leq x' \Rightarrow f(x) \preceq f(x')$
				\item $f$ est décroissante ssi $\forall (x,x') \in X^2, x \leq x' \Rightarrow f(x) \succeq f(x')$
				\item $f$ est strictement croissante ssi $\forall (x,x') \in X^2, x < x' \Rightarrow f(x) \prec f(x')$
				\item $f$ est strictement décroissante ssi $\forall (x,x') \in X^2, x < x' \Rightarrow f(x) \succ f(x')$
			\end{itemize}

			\paragraph{Remarque} On étend ces définitions aux suites de $X^\mathbb{N}$, 
			vues comme des fonctions de $(\mathbb{N},\leq)$ dans $(X,\leq)$.

			\paragraph{Exemple} La fonction identité Id de $(\mathbb{N}^*,\leq)$ dans $(\mathbb{N}^*,|)$ est strictement croissante
			
			\paragraph{Propriété} \textsl{Si $\leq$ est une relation d'ordre totale sur $X$ (autrement dit si $(X,\leq)$ est un ordre total),
				alors toute fonction strictement croissante de $(X,\leq)$ dans $(Y,\preceq)$ est injectif.}
			\begin{labeling}{$\triangleright$}
				\item [$\triangleright$] Soient $x$ et $y$ dans $X$ tels que $f(x) = f(y)$.
				On a donc en particulier $f(x) \leq f(y)$ et $f(y) \leq f(x)$. 
				Le fait que l'ordre sur $X$ soit total nous permet de conclure
				en utilisant la contraposée de la définition de la croissance stricte de $f$ que:
				\[
					\begin{cases}
						f(x) \leq f(y) \Rightarrow x \leq y \\
						f(y) \leq f(x) \Rightarrow y \leq q
					\end{cases}
					\Leftrightarrow \left(f(x) = f(y) \Rightarrow x=y \right)
				\]
				Donc $f$ est injective
			\end{labeling}

			\paragraph{Remarque} Attention à ce que l'ordre soit bien total. 
				Par exemple, la fonction de $(\mathbb{N}^*,|)$ dans $(\mathbb{N}^*,\leq)$ 
				qui associe à un entier naturel son nombre de diviseurs est bien strictement croissante,
				mais certainement pas injective !

			On dit qu'un ordre $\leq$ sur $X$ est bien fondé ssi toute partie non vide de $X$ admet
			un élément minimal.

			\paragraph{Propriété} \textsl{Un ordre est bien fondé ssi il n'existe pas une suite infinie strictement décroissante.}
			\begin{labeling}{$\Rightarrow$ :}
				\item [$\Rightarrow$ :] Considérons $(X,\leq)$ un ensemble bien fondé,
				et supposons par l'absurde l'existence d'une suite $(u_n)_{n\in\mathbb{N}}\in X^\mathbb{N}$ strictement décroissante.
				On note $A = \left\{u_n \big| n \in \mathbb{N}\right\}$.
				$A$ est une partie non vide de $X$, et admet donc un élément minimal $a_0$.
				Il existe alors par définition de $A$ $n_0\in\mathbb{N}$ tel que $a_0 = u_{n_0}$.
				Par stricte décroissance de $(u_n)$, on aurait alors  $u_{n_0+1} < u_{n_0} = a_0$, 
				ce qui contredit la minimalité de $a_0$. 
				Donc il n'existe pas de suite d'élements de $X$ strictement décroissante.
				\item [$\Leftarrow$ :] Montrons la contraposée. 
				Supposons que $(X,\leq)$ n'est pas bien fondé. 
				Il existe donc $A\subset X \neq \varnothing$ n'admettant pas d'élément minimal.
				Soit $a_0$ l'un des éléments de $A$. 
				$a_0$ n'est pas minimal dans $A$, donc il existe $a_1 \in A$ tel que $a_1 < a_0$.
				On peut alors construire selon ce procédé une suite $(a_n)_{n\in\mathbb{N}}$ infinie
				d'éléments de $X$ strictement décroissante.
			\end{labeling}
		\subsection{Ordre produit, ordre lexicographique}
			Considérons une famille finie $(X_i,\leq_i)_{i\in [1..n]}$ d'ensembles ordonnés.
			On note $Y = \prod_{i=1}^n X_i$.
			
			\paragraph{Propriété} \textsl{La relation $\mathcal{R}$ définie sur $Y$ par 
				$(a_1,a_2,...,a_n)\mathcal{R}(b_1,b_2,...,b_n) \Leftrightarrow \forall i \in [1..n], a_i\leq_i b_i$
				est une relation d'ordre, appelé ordre produit des $\leq_i$ et 
				noté $\leq_1\times\leq_2\times...\times\leq_n$ ou $\prod_{i=1}^n\leq_i$.}
			\begin{labeling}{$\triangleright$}
				\item [$\triangleright$] $\mathcal{R}$ est réflexive car on a bien, pour tout $a = (a_1,a_2,...,a_n)$ dans $Y$
					$a\mathcal{R}a$: en effet, comme tous les $a_i$ sont égaux, $\forall i \in [1..n], a_i \leq a_i$.
				\item [$\triangleright$] $\mathcal{R}$ est transitive: considérons trois éléments $a,b$ et $c$ de $Y$,
					tels que $a\mathcal{R}b$ et $b\mathcal{R}c$. Alors pour tout $i \in [1..n]$, on a $a_i\leq_i b_i$
					et $b_i \leq_i c_i$, donc par transitivité de $\leq_i$, $a_i\leq_i c_i$, d'où $a\mathcal{R}c$.
				\item [$\triangleright$] $\mathcal{R}$ est antisymétrique car si $a\mathcal{R}b$ et $b\mathcal{R}a$,
					alors pour tout $i\in [1..n]$, on a $a_i \leq_i b_i$ et $b_i\leq_i a_i$, c'est à dire
					$a_i = b_i$, donc $a=b$.
			\end{labeling}

			\paragraph{Propriété} \textsl{La relation $\mathcal{R}$ définie sur $Y$ par 
				$a\mathcal{R}b \Leftrightarrow a=b$ ou $\exists j \in [1..n], \forall i \in [1..j[, a_i = b_i$ et
				$a_j <_j b_j$
				est une relation d'ordre, appelé ordre lexicographique sur Y}
			\begin{labeling}{$\triangleright$}
				\item [$\triangleright$] $\mathcal{R}$ est réflexive par définition.
				\item [$\triangleright$] $\mathcal{R}$ est transitive: considérons trois éléments $a,b$ et $c$ de $Y$,
					tels que $a\mathcal{R}b$ et $b\mathcal{R}c$. Si $a=b$ ou $b=c$, on a bien $a\mathcal{R}c$.
					Considérons maintenant que ce ne soit pas le cas. Comme $a\mathcal{R}b$, il existe$(j,j')\in[1..n]^2$ tel que 
					$\forall i \in [1..j[$, $a_i = b_i$, $\forall i \in [1..j'[$, $b_i=c_i$ ainsi que $a_j <_j b_j$ et $b_{j'} \leq c_{j'}$.
					En prenant $J = \min(j,j')$, on a alors, pour tout $i \in [1..J[$, $a_i = c_i$, et $a_J <_J c_J$ 
					(car soit $a_J < b_J \leq_J c_J$, soit $a_J \leq b_J < c_J$ car $J = j$ ou $J = j'$).
					Ainsi $a\mathcal{R}c$
				\item [$\triangleright$] Montrons $\mathcal{R}$ est antisymétrique par l'absurde.
					Supposons qu'il existe $(a,b)\in Y^2$ tel que $a\mathcal{R}b$ et $b\mathcal{R}a$, mais $a\neq b$.
					Alors il existe $(j,j')\in[1..n]^2$ vérifiant $\forall i \in [1..j[ a_i <_i b_i$ et $\forall i \in [1..j'[ b_i <_i a_i$,
					donc en particulier pour $i=1$, $a_1 < b_1$ et $a_1 > b_1$: on arrive à une absurdité.
					Donc $\mathcal{R}$ est bien antisymétrique.
			\end{labeling}

			On peut étendre cet définition à des $n$-uplets de taille différente. Considérons un ensemble $(\Sigma,\leq)$.
			On note $\Sigma^*$ l'ensemble des $n$-uplets d'éléments de $\Sigma$. 
			$\Sigma$ est alors appelé alphabet et $\Sigma^*$ désigne l'ensemble des mots de $\Sigma$.
			Le mot vide (autrement dit le 0-uplet) est noté $\epsilon$.
			Pour un mot $u$ de $\Sigma$, on note $|u|$ la taille de ce mot.

			\paragraph{Propriété} \textsl{La relation $\mathcal{R}$ définie sur $\Sigma^*$ par $\forall (u,v)\in(\Sigma^*)^2$
			$u\mathcal{R}v \Leftrightarrow \exists j \in \mathbb{N}: $($\forall i \in [1..j[, u_i = v_i$ et $j=|u|$) ou 
			($j < |u|$ et $u_j < v_j$) est un relation d'ordre.}
			\begin{labeling}{$\triangleright$}
				\item [$\triangleright$] Introduisons le caractère nul 0 et l'ensemble $\mathcal{A} = \Sigma \cup \{0\}$.
					On introduit la relation d'ordre $\preceq$ sur $\mathcal{A}$, telle que $\preceq\cap\Sigma^2 = \leq$, et 
					que $\forall a \in \mathcal{A}, 0 \preceq a$.
					Considérons alors la fonction $f_n$ qui à un mot $(u_1,u_2,...,u_m)$ de $\Sigma$ de longueur $m \leq n$ associe
					le mot $(u_1,u_2,...,u_m,0,0,...)$ de $\mathcal{A}^n$.
				\item [$\triangleright$] Montrons que $u \mathcal{R}v \Rightarrow f(u) \preceq v$ (où $\preceq$ désigne l'ordre lexicographique) 
					et que $f$ est injective:
					Pour le premier point, si $u \mathcal{R} v$, alors il existe $j \in [1..n] : \forall i \in [1..j[, u_i = v_i$ et $j=|u|$,
					auquel cas $\forall i \in [1..n] f_n(u)_i = f_n(v_i)$ et $f_n(u)_j = 0 \preceq f_n(v)_j$ (ce qui arrive aussi dans l'autre cas)
					d'où $f_n(u) \preceq f_n(v)$.
					Pour le second point, si $f_n(u) = f_n(v)$, alors $\forall i \in [1..n]$, $f_n(u)_i = f_n(v)_i$.
					Il existe alors un $j$ à partir duquel $f(u)_i = f(v)_i = 0$. Alors $|u|=|v|=j$, 
					et $\forall i \in [1..j], u_i=f(u)_i = f(v)_i = v_i$ d'où $u=v$.
					On en déduit par ailleurs que $u\mathcal{R}v$ et $u\neq v \Rightarrow f(u) \prec f(v)$
				\item [$\triangleright$] $\mathcal{R}$ est alors réflexive: en effet si $u\mathcal{R}v$ et $v\mathcal{R}u$,
					en notant $n=\max(|u|,|v|)$, $f(u) \preceq f(v)$ et $f(v) \preceq f(u)$ d'où $f(u)=f(v)$ et par injectivité 
					$u=v$. Par ailleurs, si $u\mathcal{R}v$ et $v\mathcal{R}w$, alors $f(u) \preceq f(v) \preceq f(w)$, d'où 
					$f(u) \preceq f(w)$. Alors par contraposée on a bien $u\mathcal{R}w$. 
			\end{labeling}


\end{document}
