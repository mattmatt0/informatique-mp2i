\newcommand\PATH{/home/matthieu/Documents/Ecole/Informatique/cours/01-formalisme-des-preuves.latex}

\RequirePackage{fix-cm}
\documentclass{scrartcl}

\usepackage{tgbonum}
% Fichier pour les dépendances

% Fontes Computer Modern
\usepackage{lmodern}

% Minted
\usepackage[cache=true,outputdir=\PATH/obj]{minted}

% Pour les symboles et outils mathématiques
\usepackage{amsmath}
\usepackage{amsfonts}
\usepackage{amssymb}
\usepackage{mathtools}
\usepackage{cancel}
\usepackage{mathrsfs}
\usepackage{stmaryrd}

% Pour les dessins / images
\usepackage{xcolor}
\usepackage{tikz}
\usepackage{graphicx}

% Tableaux et listes
\usepackage{tabularx}
\newcolumntype{Y}{>{\centering\arraybackslash}X}
\usepackage{booktabs}
\usepackage{multirow}
\usepackage{enumitem}
\usepackage{multicol}
\usepackage{adjustbox}

% Présentation de la page (bordures)
\usepackage[a4paper,top=1cm,bottom=1cm,includefoot,left=1.5cm,right=1.5cm,footskip=1cm]{geometry}
\usepackage{scrlayer-scrpage}
\rofoot*{\pagemark}
\cofoot*{}
\pagestyle{scrheadings}

% Autres
\usepackage{ulem} % Soulignage
\usepackage[french]{babel} % Babel, pour respecter les standards français
\frenchbsetup{StandardLists=true} % ... sauf pour les immondes listes à tirets
\usepackage[T1]{fontenc}

\usepackage{hyperref} % Pour les liens
\hypersetup{
    colorlinks=true,
    linkcolor=blue,
    filecolor=magenta,
    urlcolor=cyan,
}


%% Commandes personnalisées

% Code
\newenvironment{code}[1]
{\VerbatimEnvironment\begin{minted}[breaklines,frame=single,autogobble,linenos,bgcolor=codebg,tabsize=4,mathescape=true]{#1}}
{\end{minted}}
\definecolor{codebg}{gray}{1}
\newenvironment{algotext}
{\VerbatimEnvironment\begin{minted}[frame=single,autogobble,linenos,bgcolor=codebg,tabsize=4,escapeinside=||,mathescape=true]{text}}
{\end{minted}}


% Paragraphes spéciaux
\newcounter{compteurExos}
\setcounter{compteurExos}{0}
\newcommand{\plabel}[1]{{\parindent10pt \fontfamily{cmss}\selectfont\textbf{#1} \,}}

\newcommand{\prop}[1]{\plabel{Propriété :} \textsl{#1}}
\newcommand{\lemma}[1]{\plabel{Lemme :} \textsl{#1}}
\newcommand{\rem}{\plabel{Remarque :}}
\newcommand{\exemple}{\plabel{Exemple :}}
\newcommand{\exo}{\stepcounter{compteurExos}\plabel{Exercice \thecompteurExos :}}
\newcommand{\obs}{\plabel{Observation :} }

\newenvironment{demo}{\begin{itemize}[label=$\triangleright$]}{\end{itemize}}
\newcommand{\definition}{\parindent0pt }
\newcommand{\semidef}{\parindent0pt $\bullet$ \,}

% Outils
\newcommand{\corrpar}{\vspace{-20pt}}
\newcommand{\extraspace}{\vspace{5pt}}

% Configuration
\parskip5pt

% Annonce pour le développement:
\newcommand{\notecentrale}[1]{\begin{center}\textsl{#1}\end{center}}
\newcommand{\warnwip}{\notecentrale{Ce cours n'est pas encore bien relu}}
\newcommand{\warnwipnext}{\notecentrale{La suite de ce cours est encore en rédaction}\message}


\newcommand{\set}[1]{{\{#1\}}}
\newcommand{\intset}[1]{\llbracket #1 \rrbracket}
\newcommand{\intsete}[1]{\llbracket #1 \llbracket}
\newcommand{\inteset}[1]{\rrbracket #1 \rrbracket}
\newcommand{\intesete}[1]{\rrbracket #1 \llbracket}

\newcommand{\tq}{\, \big| \,}

\newcommand{\card}{\.\textmd{card}}
\newcommand{\Id}{\textsl{Id}}
\newcommand{\supp}{\text{supp}}

% Partie entière, arrondi supérieur

\DeclarePairedDelimiter\ceil{\lceil}{\rceil}
\DeclarePairedDelimiter\floor{\lfloor}{\rfloor}




\title{Schémas de preuves}
\author{}
\date{}

\begin{document}
	\maketitle
	\section{Raisonnements usuels}
		\subsection{Implications / Équivalences}
			Le principe est d'utiliser la transitivité de l'implication.
			Ainsi pour prouver que $A$ implique $C$, on commencera par supposer $A$ vraie,
			puis on rappelera que $A$ implique $B$, et donc que $B$ est vraie. 
			Ensuite, on dira que $B$ implique $C$, donc comme $B$ est vraie, $C$ l'est aussi.

			Le raisonnement par équivalence est le même procédé, sauf que l'on utilise des équivalences
			à la place des implications.

		\subsection{Absurde}
			Le but ici est de montrer que quelque chose est faux.
			Pour ce faire, on commence par supposer l'assertion à infirmer, 
			et en utilisant certains raisonnements, on en déduit qu'une assertion que l'on sait vraie est fausse,
			ou réciproquement. On dit alors qu'il y a contradiction, ou absurdité,
			et on conclut en disant que l'assertion initialement supposée est fausse.

			\rem C'est un cas particulier du raisonnement par contraposée: si $A$ implique quelque chose de faux,
			cela signifie que quelque chose de vrai implique non $A$. 

		\subsection{Récurrence}
			La récurrence permet de montrer une propriété $P$ dépendant d'un entier naturel $n$ 
			(on note la propriété au rang $n$ $P(n)$ ou $P_n$) 
			sur une partie de $\mathbb{N}$.
			Le cas général est la récurrence forte, 
			qui permet de démontrer la propriété sur $\mathbb{N}$.
			Elle consiste en:
			\begin{enumerate}
				\item Démontrer $P(0)$ (initialisation)
				\item Fixer un entier naturel $n$
				\item Supposer, pour tout $k \in \intset{0,n}$, que $P(k)$ est vraie
				\item En déduire que $P(n+1)$ est vraie (hérédité).
			\end{enumerate}

			\rem La récurrence forte est le \textsl{cas général}. 
			On utilisera plus souvent la récurrence simple, 
			qui suppose uniquement $P(n)$ et non $P(k)$ pour tout $k \in \intset{0,n}$.
			Il existe aussi la récurrence finie 
			(le $n$ pour lequel on suppose $P(n)$ est compris dans un certain intervalle $\intset{a,b}$ d'entiers fini, 
			et on prouve $P(a)$ à l'initialisation plutôt que $P(0)$), 
			ainsi que la récurrence descendante 
			(une récurrence finie, mais on prouve $P(b)$ à l'initialisation
			puis $P(n-1)$ à l'hérédité, jusqu'à $P(a)$).

			\rem La récurrence peut être encore généralisée à l'induction, qui sera évoquée ultérieurement dans le programme.

		\subsection{Analyse-synthèse}
			Le raisonnement par analyse synthèse permet de trouver toutes les solutions à un problème.
			On commence par considérer une solution quelconque d'un problème donné. 
			En raisonnant sur les propriétés qu'elle doit alors avoir, on en déduit qu'elle doit avoir une forme donnée: 
			c'est l'\textbf{analyse}. Ensuite, on considère un objet ayant cette forme, et on montre qu'alors il est solution:
			c'est la \textbf{synthèse}.

		\subsection{Disjonction des cas}
			La disjonction des cas consiste à partitionner l'ensemble des objets vérifiant les hypothèses afin d'en faire
			émerger de nouvelles. On montre alors la propriété sur chacune des parties.

			\rem Attention: on parle bien de \underline{partition}: les objets traités dans chaque cas sont différents,
			et une fois réunis, il reforment l'ensemble de départ!

			\exemple Montrons que la somme de deux entiers de même parité est paire par disjonction de cas.
			Soient $m$ et $n$ deux entiers naturels de même parité.
			\begin{itemize}
				\item Soit $m$ et $n$ sont tous les deux pairs, et alors il existe deux entiers $p$ et $q$
				tels que $m=2p$ et $n=2q$. Alors $m+n = 2p+2q = 2(p+q)$ est un entier pair.
				\item Soit $m$ et $n$ sont tous les deux impairs, et alors il existe deux entiers $p$ et $q$
				tels que $m=2p+1$ et $n=2q+1$. Alors $m+n = 2p+1+2q+1 = 2(p+q+1)$ est un entier pair.
			\end{itemize}

	\section{Preuves sur une fonction}
		\subsection{Boucles}
		\subsection{Terminaison d'une fonction}
		\subsection{Correction d'une fonction}
\end{document}
