\newcommand\PATH{Lancez `make adapt avant de compiler!`}

\RequirePackage{fix-cm}
\documentclass{scrartcl}

% Fichier pour les dépendances

% Fontes Computer Modern
\usepackage{lmodern}

% Minted
\usepackage[cache=true,outputdir=\PATH/obj]{minted}

% Pour les symboles et outils mathématiques
\usepackage{amsmath}
\usepackage{amsfonts}
\usepackage{amssymb}
\usepackage{mathtools}
\usepackage{cancel}
\usepackage{mathrsfs}
\usepackage{stmaryrd}

% Pour les dessins / images
\usepackage{xcolor}
\usepackage{tikz}
\usepackage{graphicx}

% Tableaux et listes
\usepackage{tabularx}
\newcolumntype{Y}{>{\centering\arraybackslash}X}
\usepackage{booktabs}
\usepackage{multirow}
\usepackage{enumitem}
\usepackage{multicol}
\usepackage{adjustbox}

% Présentation de la page (bordures)
\usepackage[a4paper,top=1cm,bottom=1cm,includefoot,left=1.5cm,right=1.5cm,footskip=1cm]{geometry}
\usepackage{scrlayer-scrpage}
\rofoot*{\pagemark}
\cofoot*{}
\pagestyle{scrheadings}

% Autres
\usepackage{ulem} % Soulignage
\usepackage[french]{babel} % Babel, pour respecter les standards français
\frenchbsetup{StandardLists=true} % ... sauf pour les immondes listes à tirets
\usepackage[T1]{fontenc}

\usepackage{hyperref} % Pour les liens
\hypersetup{
    colorlinks=true,
    linkcolor=blue,
    filecolor=magenta,
    urlcolor=cyan,
}


%% Commandes personnalisées

% Code
\newenvironment{code}[1]
{\VerbatimEnvironment\begin{minted}[breaklines,frame=single,autogobble,linenos,bgcolor=codebg,tabsize=4,mathescape=true]{#1}}
{\end{minted}}
\definecolor{codebg}{gray}{1}
\newenvironment{algotext}
{\VerbatimEnvironment\begin{minted}[frame=single,autogobble,linenos,bgcolor=codebg,tabsize=4,escapeinside=||,mathescape=true]{text}}
{\end{minted}}


% Paragraphes spéciaux
\newcounter{compteurExos}
\setcounter{compteurExos}{0}
\newcommand{\plabel}[1]{{\parindent10pt \fontfamily{cmss}\selectfont\textbf{#1} \,}}

\newcommand{\prop}[1]{\plabel{Propriété :} \textsl{#1}}
\newcommand{\lemma}[1]{\plabel{Lemme :} \textsl{#1}}
\newcommand{\rem}{\plabel{Remarque :}}
\newcommand{\exemple}{\plabel{Exemple :}}
\newcommand{\exo}{\stepcounter{compteurExos}\plabel{Exercice \thecompteurExos :}}
\newcommand{\obs}{\plabel{Observation :} }

\newenvironment{demo}{\begin{itemize}[label=$\triangleright$]}{\end{itemize}}
\newcommand{\definition}{\parindent0pt }
\newcommand{\semidef}{\parindent0pt $\bullet$ \,}

% Outils
\newcommand{\corrpar}{\vspace{-20pt}}
\newcommand{\extraspace}{\vspace{5pt}}

% Configuration
\parskip5pt

% Annonce pour le développement:
\newcommand{\notecentrale}[1]{\begin{center}\textsl{#1}\end{center}}
\newcommand{\warnwip}{\notecentrale{Ce cours n'est pas encore bien relu}}
\newcommand{\warnwipnext}{\notecentrale{La suite de ce cours est encore en rédaction}\message}


\newcommand{\fpq}{\mathbb{F}_p(Q)}
\newcommand{\abins}{\mathscr{A}_B(\mathcal{S})}
\newcommand{\pb}[3]
{
	\textbf{#1}
	\left|\left|\begin{tabular}{l l}
		Entrée: & #2 \\
		\vspace{-10pt} & \\
		\hline
		\vspace{-10pt} & \\
		Sortie: & #3
	\end{tabular}\right.\right.
}
\newcommand{\pbinlist}[3]
{
	\textbf{#1} &  $\left|\left|\begin{tabular}{l l}
		Entrée: & #2\\ 
		\vspace{-10pt} & \\
		\hline
		\vspace{-10pt} & \\
		Sortie: & #3 
	\end{tabular}\right.\right.$ 
}
\newcommand{\val}{\text{val}}
\newcommand{\argmax}{\mathop{\text{argmax}}}
\newcommand{\argmin}{\mathop{\text{argmin}}}

\newcommand{\set}[1]{{\{#1\}}}
\newcommand{\intset}[1]{\llbracket #1 \rrbracket}
\newcommand{\intsete}[1]{\llbracket #1 \llbracket}
\newcommand{\inteset}[1]{\rrbracket #1 \rrbracket}
\newcommand{\intesete}[1]{\rrbracket #1 \llbracket}

\newcommand{\tq}{\, \big| \,}

\newcommand{\card}{\.\textmd{card}}
\newcommand{\Id}{\textsl{Id}}
\newcommand{\supp}{\text{supp}}

% Partie entière, arrondi supérieur

\DeclarePairedDelimiter\ceil{\lceil}{\rceil}
\DeclarePairedDelimiter\floor{\lfloor}{\rfloor}



\usepackage{lmodern}
\usepackage{algorithmicx}

\title{Algorithmique du texte}
\author{}
\date{}

\begin{document}
	\maketitle
	\definition On appelle \textbf{texte} une suite finie de caractères (ce que l'on a appelé mot jusqu'à présent).
	L'algorithmique du texte consiste à résoudre des problèmes sur des textes, qui peuvent en réalité modéliser des informations diverses:
	\begin{itemize}
		\item des textes à proprement parler;
		\item une séquence ADN;
		\item de la musique;
		\item des images...
	\end{itemize}
	Par exemple,
	\begin{itemize}
		\item les recherches de similarité, dont notamment:
		\begin{itemize}
			\item la recherche du plus long sous-mot commun;
			\item la recherche du plus long facteur commun;
			\item la distance d'édition / alignement;
		\end{itemize}
		\item la recherche de motif;
		\item la compression;
		\item l'encodage par facteurs...
	\end{itemize}
	
	\section{Rappels}
		On fixe $\Sigma$ un ensemble, appelé \textbf{alphabet}, dont les élément sont appelés \textbf{caractères}.
		On définit alors l'ensemble $\Sigma*$ des textes sur $\Sigma$ et celui des textes non vides $\Sigma^+$ par:
		\[
			\Sigma^* = \bigcup_{n \in \mathbb{N}} \Sigma^n \quad \quad \Sigma^+ = \bigcup_{n\in \mathbb{N}^*} \Sigma^n
		\]
		On note $\varepsilon$ le texte vide, c'est à dire le seul 0-uplet de $\Sigma^*$.
		On munit alors $\Sigma^*$ de la \textbf{concaténation} définie ainsi:
		\[
			\forall (u = (u_i)_{i \in \intset{1,n}},(v_j)_{j \in \intset{1,m}}) \in (\Sigma^*)^2, u.v = u_1u_2...u_{n-1}u_nv_1v_2...v_{m-1}v_m 
		\]
		On vérifie alors que $(\Sigma^*,.)$ est un monoïde.
		
		On appelle \textbf{sous-texte} d'un texte $u$ de $\Sigma^*$ toute suite extraite de $u$.
		Par ailleurs, on dit que $v \in \Sigma^*$ est un \textbf{facteur} de $u = u_1...u_n$ ssi il existe $(i,j)\in\intset{1,n}^2$ tel que $v = u_i...u_j$.
		Dans le cas où $i=1$, on dit que $v$ est un \textbf{préfixe} de $u$. Si $j=n$, alors c'est un \textbf{suffixe} de $u$.

	\section{Plus long facteur commun}
		\subsection{Description}
		\[
			\pb{\begin{minipage}{0.2\textwidth}\begin{center}Plus long facteur commun\end{center}\end{minipage}}
			{$u \in \Sigma^*$ de longueur $n$ et $v \in \Sigma^*$ de longueur $m$}
			{\begin{minipage}{0.6\textwidth} max$\set{|f| | f \text{ facteur de $u$ et de $v$} }$ i.e \\ 
			\begin{tabular}{l}
				$\exists (i,j) \in \intset{1,n}^2, f=(u_k)_{k \in \intset{i,j}}$ ou encore \\ 
				$\exists (p,q) \in \intset{1,m}^2, f=(v_k)_{k\in\intset{p,q}}$
			\end{tabular} \end{minipage}
			}
		\]
		On abgrègera \textbf{Plus long facteur commun} en \textbf{PLFC}.

		\subsection{Résolution}
			On pose pour $(i,j) \in \intset{0,n} \times \intset{0,m}$:
			\[
				A_{i,j} = \max\set{|s| \tq s \text{ est un suffixe de } u_1...u_i, v_1...v_j} \leq \min (i,j)
			\]
			On a immédiatement, pour tout $(i,j) \in \intset{1,n} \times \intset{1,m}$:
			\begin{itemize}
				\item $i=0$ ou $j=0 \rightarrow A_{i,j} = 0$
				\item $A_{i,j} = A_{i-1,j-1} + 1$ si $u_i = v_j$, 0 sinon.
			\end{itemize}

			D'ou \textbf{PLFC}$(u,v) = \max_{(i,j) \in \intset{0,n} \times \intset{0,m}} A_{i,j}$

	\section{Recherche de motif}
		\subsection{Définitions}
			\[
				\pb{Trouve motif}
				{$t \in \Sigma^*$ un texte de longueur $n$, $x \in \Sigma^*$ un motif de longueur $m$ (avec $m \leq n$)}
				{$\underbrace{\set{i \in \intset{1,n} \tq (t_{i+k})_{k \in \intsete{0,m} = x}}}_\text{l'ensemble des indices de début des occurences de $x$ dans $t$}$}
			\]
			\exo{Proposer un algorithme de résolution naïve pour ce problème.}

		\subsection{Algorithme de }
\end{document}
