\newcommand\PATH{Lancez `make adapt avant de compiler!`}

\RequirePackage{fix-cm}
\documentclass{scrartcl}

% Fichier pour les dépendances

% Fontes Computer Modern
\usepackage{lmodern}

% Minted
\usepackage[cache=true,outputdir=\PATH/obj]{minted}

% Pour les symboles et outils mathématiques
\usepackage{amsmath}
\usepackage{amsfonts}
\usepackage{amssymb}
\usepackage{mathtools}
\usepackage{cancel}
\usepackage{mathrsfs}
\usepackage{stmaryrd}

% Pour les dessins / images
\usepackage{xcolor}
\usepackage{tikz}
\usepackage{graphicx}

% Tableaux et listes
\usepackage{tabularx}
\newcolumntype{Y}{>{\centering\arraybackslash}X}
\usepackage{booktabs}
\usepackage{multirow}
\usepackage{enumitem}
\usepackage{multicol}
\usepackage{adjustbox}

% Présentation de la page (bordures)
\usepackage[a4paper,top=1cm,bottom=1cm,includefoot,left=1.5cm,right=1.5cm,footskip=1cm]{geometry}
\usepackage{scrlayer-scrpage}
\rofoot*{\pagemark}
\cofoot*{}
\pagestyle{scrheadings}

% Autres
\usepackage{ulem} % Soulignage
\usepackage[french]{babel} % Babel, pour respecter les standards français
\frenchbsetup{StandardLists=true} % ... sauf pour les immondes listes à tirets
\usepackage[T1]{fontenc}

\usepackage{hyperref} % Pour les liens
\hypersetup{
    colorlinks=true,
    linkcolor=blue,
    filecolor=magenta,
    urlcolor=cyan,
}


%% Commandes personnalisées

% Code
\newenvironment{code}[1]
{\VerbatimEnvironment\begin{minted}[breaklines,frame=single,autogobble,linenos,bgcolor=codebg,tabsize=4,mathescape=true]{#1}}
{\end{minted}}
\definecolor{codebg}{gray}{1}
\newenvironment{algotext}
{\VerbatimEnvironment\begin{minted}[frame=single,autogobble,linenos,bgcolor=codebg,tabsize=4,escapeinside=||,mathescape=true]{text}}
{\end{minted}}


% Paragraphes spéciaux
\newcounter{compteurExos}
\setcounter{compteurExos}{0}
\newcommand{\plabel}[1]{{\parindent10pt \fontfamily{cmss}\selectfont\textbf{#1} \,}}

\newcommand{\prop}[1]{\plabel{Propriété :} \textsl{#1}}
\newcommand{\lemma}[1]{\plabel{Lemme :} \textsl{#1}}
\newcommand{\rem}{\plabel{Remarque :}}
\newcommand{\exemple}{\plabel{Exemple :}}
\newcommand{\exo}{\stepcounter{compteurExos}\plabel{Exercice \thecompteurExos :}}
\newcommand{\obs}{\plabel{Observation :} }

\newenvironment{demo}{\begin{itemize}[label=$\triangleright$]}{\end{itemize}}
\newcommand{\definition}{\parindent0pt }
\newcommand{\semidef}{\parindent0pt $\bullet$ \,}

% Outils
\newcommand{\corrpar}{\vspace{-20pt}}
\newcommand{\extraspace}{\vspace{5pt}}

% Configuration
\parskip5pt

% Annonce pour le développement:
\newcommand{\notecentrale}[1]{\begin{center}\textsl{#1}\end{center}}
\newcommand{\warnwip}{\notecentrale{Ce cours n'est pas encore bien relu}}
\newcommand{\warnwipnext}{\notecentrale{La suite de ce cours est encore en rédaction}\message}


\newcommand{\set}[1]{{\{#1\}}}
\newcommand{\intset}[1]{\llbracket #1 \rrbracket}
\newcommand{\intsete}[1]{\llbracket #1 \llbracket}
\newcommand{\inteset}[1]{\rrbracket #1 \rrbracket}
\newcommand{\intesete}[1]{\rrbracket #1 \llbracket}

\newcommand{\tq}{\, \big| \,}

\newcommand{\card}{\.\textmd{card}}
\newcommand{\Id}{\textsl{Id}}
\newcommand{\supp}{\text{supp}}

% Partie entière, arrondi supérieur

\DeclarePairedDelimiter\ceil{\lceil}{\rceil}
\DeclarePairedDelimiter\floor{\lfloor}{\rfloor}



\newcommand{\fpq}{\mathbb{F}_p(Q)}
\newcommand{\abins}{\mathscr{A}_B(\mathcal{S})}
\newcommand{\pb}[3]
{
	\textbf{#1}
	\left|\left|\begin{tabular}{l l}
		Entrée: & #2 \\
		\vspace{-10pt} & \\
		\hline
		\vspace{-10pt} & \\
		Sortie: & #3
	\end{tabular}\right.\right.
}
\newcommand{\pbinlist}[3]
{
	\textbf{#1} &  $\left|\left|\begin{tabular}{l l}
		Entrée: & #2\\ 
		\vspace{-10pt} & \\
		\hline
		\vspace{-10pt} & \\
		Sortie: & #3 
	\end{tabular}\right.\right.$ 
}
\newcommand{\val}{\text{val}}
\newcommand{\argmax}{\mathop{\text{argmax}}}
\newcommand{\argmin}{\mathop{\text{argmin}}}


\title{Notions fondamentales}
\author{}
\date{}

\begin{document}
	\maketitle
	\section{Exemples de domaines informatiques.}
		L'informatique ne se résume pas à l'utilisation d'ordinateurs. On peut citer Alan Turing,
		qui disait que 
		"l'informatique n'est pas plus la science des ordinateurs que l'astronomie n'est la science des téléscopes".
		C'est une discipline très vaste, avec beaucoup de domaines, dont notamment:
		\begin{multicols}{2}
		\begin{itemize}
			\item Les bases de données
			\item Les réseaux
			\item Les systèmes embarqués
			\item La robotique
			\item La sécurité des systèmes d'information
			\item La cryptographie
			\item Le calcul scientifique
			\item L'intelligence artificielle
			\item L'interface utilisateur
			\item Le développement web
			\item La recherche opérationnelle et l'optimisation
			\item La vérification de programmes
			\item La logique
		\end{itemize}
		\end{multicols}
	
	\section{Problèmes}
		Un \textbf{problème} est la donnée d'un format de \textbf{paramètres d'entrée} (qui décrit quelles données on a entrées),
		et de la \textbf{sortie attendue} en fonction de ces paramètres.

		\exemple 
		\[
			\pb{Rectangle}
			{Un entier $a \in \mathbb{N}$, un entier $b \in \mathbb{N}$.}
			{La surface d'un rectangle de longueur $a$ et de largeur $b$.}
		\]\[
			\pb{Parité}
			{$n\in\mathbb{N}$.}
			{\texttt{Vrai} si $n$ est pair, \texttt{Faux} sinon.}
		\]
		
		On appelle \textbf{problème de décision} un problème dans lequel 
		la sortie attendue est \texttt{Vrai} ou \texttt{Faux}, et
		\textbf{problème d'optimisation} un problème dans lequel la sortie attendue minimise ou maximise
		une fonction donnée sur un ensemble donné.

		\exemple \textbf{Parité} est un problème de décision. Les deux problèmes qui suivent sont des problèmes d'optimisation:
		\[
			\pb{\begin{tabular}{c}Plus court \\ chemin\end{tabular}}
			{\begin{tabular}{l}Un graphe $G$ (pondéré), un point de départ $A \in G$\\ et un point d'arrivée $B \in G$.\end{tabular}}
			{Un chemin de longueur minimale de $A$ à $B$ dans $G$}
		\]\[
			\pb{Sac à dos}
			{$n\in \mathbb{N}$, $P_{max} \in \mathbb{R}^+, (p_i)_{i\in\intset{1,n}} \in (\mathbb{R}^+)^n$, $(v_i)_{i \in \intset{1,n}}$}
			{\begin{tabular}{l}$(x_i^*)_{i \in \intset{1,n}} \in \set{0,1}^n$ tel que $\sum_{i=1}^n x_i^*p_i \leq P_{max}$  \\
			et maximisant 
			$f = \left(
			\begin{tabular}{c c c}
				$\set{0,1}^n$ & $\rightarrow$ & $\mathbb{R}$ \\ 
				$(x_1,...,x_n)$ & $\mapsto$ & $\sum_{i=1}^n x_iv_i$
			\end{tabular}\right)$\end{tabular}}
		\]

		Pour un problème donné, une \textbf{instance} est une famille de paramètres répondant au type des entrées du problème.

		\plabel{Exemples:}
		\begin{itemize}
			\item 18 est une instance positive (de solution \texttt{Vrai}) de \textbf{Parité}.
			\item 17 est une instance négative (de solution \texttt{Faux}) de \textbf{Parité}.
			\item (7,8) est une instance de \textbf{Rectangle} de solution 56.
			\item (1,(1,2),(3,15)) est un instance de \textbf{Sac à dos} de solution (1,0).
		\end{itemize}

	\section{Algorithmes}
		Un \textbf{algorithme} est une méthode systématique à base d'\textbf{opérations} qui permet de résoudre toutes les instances d'un problème.

		\exemple Comptage binaire sur les doigts:
		\begin{algotext}
Commencer les doigts levés.
Si le pouce est baissé
	Lever le pouce
Sinon
	Baisser les doigts levés consécutifs au pouce
	Lever le doigt suivant
	Baisser le pouce
		\end{algotext}

		En programmation, une \textbf{fonction} peut être vue come un rassemblement d'instructions, 
		qui s'appliquent sur des données que l'on peut faire varier avec ce que l'on appelle argument.
		Si on attend généralement d'une fonction qu'elle donne/retourne un résultat en sortie, 
		mais elle peut aussi avoir des \textbf{effets de bord}: affichages, modification de la mémoire...

		\rem Quitte à décomposer les problèmes difficiles en sous-problèmes, on essaiera d'avoir un fonction par problème.
		Pour que ce découpage par fonctions soit lisible dans le code, 
		on munit chaque fonction dans celui-ci d'une spécification qui contient:
		\begin{itemize}
			\item La signature de la fonction, contenant:
			\begin{itemize}
				\item Le nombre et le type des arguments
				\item Le type de la sortie
			\end{itemize}
			\item Les hypothèses sur les arguments
			\item La valeur de la sortie pour les arguments et éventuellement la description des effets de bord.
		\end{itemize}

		\exemple
		\begin{code}{c}
			int surface_rectangle(int longueur, int largeur)
			{
				// hyp: longueur >= 0 && largeur >= 0
				/* Renvoie la surface d'un rectangle de longueur `longueur` et de largeur `largeur` */
				return longueur * largeur;
			}
		\end{code}
	\section{Langage de programmation}
		\definition Un langage de programmation est une manière conventionnelle et formelle
		de formuler des algorithmes intelligibles par les humains et exécutables par une machine.

		\begin{tabular}{| c | c | c |}
			\hline
			 & Langue naturelle & Langage de programmation \\
			 \hline
			 Alphabet & \Verb|abc...zABC...Z 0123456789 .;:"()| & \Verb|+@{}#&...| (table ASCII) \\
			 \hline 
			 Mots & Mots présents dans le dictionnaire, noms propres & Mots clés, identificateurs \\
			 \hline
			 Grammaire & Former des mots/phrases selon les règles de la grammaire & Syntaxe \\
			 \hline
			 Sémantique & Sens de la phrase & Sémantique \\
			 \hline
		\end{tabular}
	
			
\end{document}
