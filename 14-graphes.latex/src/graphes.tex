\newcommand\PATH{Lancez `make adapt avant de compiler!`}

\RequirePackage{fix-cm}
\documentclass{scrartcl}

% Fichier pour les dépendances

% Fontes Computer Modern
\usepackage{lmodern}

% Minted
\usepackage[cache=true,outputdir=\PATH/obj]{minted}

% Pour les symboles et outils mathématiques
\usepackage{amsmath}
\usepackage{amsfonts}
\usepackage{amssymb}
\usepackage{mathtools}
\usepackage{cancel}
\usepackage{mathrsfs}
\usepackage{stmaryrd}

% Pour les dessins / images
\usepackage{xcolor}
\usepackage{tikz}
\usepackage{graphicx}

% Tableaux et listes
\usepackage{tabularx}
\newcolumntype{Y}{>{\centering\arraybackslash}X}
\usepackage{booktabs}
\usepackage{multirow}
\usepackage{enumitem}
\usepackage{multicol}
\usepackage{adjustbox}

% Présentation de la page (bordures)
\usepackage[a4paper,top=1cm,bottom=1cm,includefoot,left=1.5cm,right=1.5cm,footskip=1cm]{geometry}
\usepackage{scrlayer-scrpage}
\rofoot*{\pagemark}
\cofoot*{}
\pagestyle{scrheadings}

% Autres
\usepackage{ulem} % Soulignage
\usepackage[french]{babel} % Babel, pour respecter les standards français
\frenchbsetup{StandardLists=true} % ... sauf pour les immondes listes à tirets
\usepackage[T1]{fontenc}

\usepackage{hyperref} % Pour les liens
\hypersetup{
    colorlinks=true,
    linkcolor=blue,
    filecolor=magenta,
    urlcolor=cyan,
}


%% Commandes personnalisées

% Code
\newenvironment{code}[1]
{\VerbatimEnvironment\begin{minted}[breaklines,frame=single,autogobble,linenos,bgcolor=codebg,tabsize=4,mathescape=true]{#1}}
{\end{minted}}
\definecolor{codebg}{gray}{1}
\newenvironment{algotext}
{\VerbatimEnvironment\begin{minted}[frame=single,autogobble,linenos,bgcolor=codebg,tabsize=4,escapeinside=||,mathescape=true]{text}}
{\end{minted}}


% Paragraphes spéciaux
\newcounter{compteurExos}
\setcounter{compteurExos}{0}
\newcommand{\plabel}[1]{{\parindent10pt \fontfamily{cmss}\selectfont\textbf{#1} \,}}

\newcommand{\prop}[1]{\plabel{Propriété :} \textsl{#1}}
\newcommand{\lemma}[1]{\plabel{Lemme :} \textsl{#1}}
\newcommand{\rem}{\plabel{Remarque :}}
\newcommand{\exemple}{\plabel{Exemple :}}
\newcommand{\exo}{\stepcounter{compteurExos}\plabel{Exercice \thecompteurExos :}}
\newcommand{\obs}{\plabel{Observation :} }

\newenvironment{demo}{\begin{itemize}[label=$\triangleright$]}{\end{itemize}}
\newcommand{\definition}{\parindent0pt }
\newcommand{\semidef}{\parindent0pt $\bullet$ \,}

% Outils
\newcommand{\corrpar}{\vspace{-20pt}}
\newcommand{\extraspace}{\vspace{5pt}}

% Configuration
\parskip5pt

% Annonce pour le développement:
\newcommand{\notecentrale}[1]{\begin{center}\textsl{#1}\end{center}}
\newcommand{\warnwip}{\notecentrale{Ce cours n'est pas encore bien relu}}
\newcommand{\warnwipnext}{\notecentrale{La suite de ce cours est encore en rédaction}\message}


\newcommand{\fpq}{\mathbb{F}_p(Q)}
\newcommand{\abins}{\mathscr{A}_B(\mathcal{S})}
\newcommand{\pb}[3]
{
	\textbf{#1}
	\left|\left|\begin{tabular}{l l}
		Entrée: & #2 \\
		\vspace{-10pt} & \\
		\hline
		\vspace{-10pt} & \\
		Sortie: & #3
	\end{tabular}\right.\right.
}
\newcommand{\pbinlist}[3]
{
	\textbf{#1} &  $\left|\left|\begin{tabular}{l l}
		Entrée: & #2\\ 
		\vspace{-10pt} & \\
		\hline
		\vspace{-10pt} & \\
		Sortie: & #3 
	\end{tabular}\right.\right.$ 
}
\newcommand{\val}{\text{val}}
\newcommand{\argmax}{\mathop{\text{argmax}}}
\newcommand{\argmin}{\mathop{\text{argmin}}}

\newcommand{\set}[1]{{\{#1\}}}
\newcommand{\intset}[1]{\llbracket #1 \rrbracket}
\newcommand{\intsete}[1]{\llbracket #1 \llbracket}
\newcommand{\inteset}[1]{\rrbracket #1 \rrbracket}
\newcommand{\intesete}[1]{\rrbracket #1 \llbracket}

\newcommand{\tq}{\, \big| \,}

\newcommand{\card}{\.\textmd{card}}
\newcommand{\Id}{\textsl{Id}}
\newcommand{\supp}{\text{supp}}

% Partie entière, arrondi supérieur

\DeclarePairedDelimiter\ceil{\lceil}{\rceil}
\DeclarePairedDelimiter\floor{\lfloor}{\rfloor}



\usepackage{lmodern}

\title{Graphes}
\author{}
\date{}

\begin{document}
	\maketitle
	\section{Définitions}
		\subsection{Graphes}
			\definition Un graphe \textbf{non orienté} est la donnée 
			d'un couple $G = (V,E)$,
			où $V$ est un ensemble fini non vide 
			et $E \subset \set{\set{x,y} \tq (x,y) \in V^2}$.
			Un graphe \textbf{orienté} est la donnée d'un couple $H = (S,A)$,
			où $S$ est un ensemble fini non vide et 
			$A \in \mathcal{P}(S^2)$.

			Les éléments de $V$ et $A$ sont appelés \textbf{sommets du graphe}, 
			On parlera, pour désigner les éléments de $E$ dans un graphe non-orienté d'\textbf{arêtes} du graphe,
			et pour un graphe orienté d'\textbf{arcs}.

			Si $e = \set{x} \in E$ (avec $x \in V$), $e$ est une \textbf{boucle} sur $x$
			(idem pour $e = (x,x)$ pour $x \in S$). 
			Pour $(x,y) \in V^2$, on dit que $x$ et $y$ sont \textbf{voisins} ssi $\set{x,y} \in E$.
			Cette notion se précise dans un graphe orienté: 
			on dit que $x \in S$ est un \textbf{successeur} (resp. \textbf{prédecesseur}) de $y$
			ssi $(y,x) \in A$ (resp. $(x,y)\in A$). 
			
			On appelle \textbf{voisinnage} du sommet $x$ l'ensemble $\mathscr{V}(x)$ de ses voisins.
			Le \textbf{degré} de $x$, noté $\deg x$, est le cardinal de ce voisinnage.

			Pour les graphes orientés, on distingue le \textbf{degré sortant} de $x$, noté $\deg^+ x$, le nombre de successeurs de $x$,
			du \textbf{degré entrant} de $x$, noté $\deg^- x$,le nombre de prédecesseurs de $x$.

			\notecentrale{On supposera par la suite que l'on travaille avec des graphes sans boucles.}

			\prop{Soit $G = (V,E)$ un graphe. On a $\sum_{x \in V} \deg(x) = 2\card(E)$ }
			\begin{demo}
				\item On a:
				\[
					\sum_{x \in V} \deg(x)= \sum_{x\in V} \sum_{y\in V} \mathbbm{1}_\set{x,y} = \sum_{(x,y) \in V^2} \mathbbm{1}_\set{x,y}
					= 2\sum_{\set{x,y} \in V^2} \mathbbm{1}_\set{x,y} = 2\,\card(E).
				\]
				Car le graphe est sans boucle. Il faudrait sinon ajouter le nombre de boucles présentes dans le graphe.
			\end{demo}

		\subsection{Accessibilité, connexité}
			\notecentrale{On fixe $G = (V,E)$ un graphe non orienté, 
			et $H = (A,S)$ un graphe orienté.}
			
			\definition Soit $s = (s_i)\in V^{n+1}$. 
			On dit que $s$ est une \textbf{chaîne de $G$} ssi
			$\forall i \in \intsete{1,n}, \set{s_i, s_{i+1}} \in E$
			On dit alors que $s$ est une chaîne de longueur $n$ et qui relie $s_0$ \textbf{et} $s_n$.

			\definition Soit $s = (s_i)\in A^{n+1}$. 
			On dit que $s$ est un \textbf{chemin de $G$} ssi
			$\forall i \in \intsete{1,n}, \set{s_i, s_{i+1}} \in E$	
			On dit alors que $s$ est une chaîne de longueur $n$ et qui relie $s_0$ \textbf{à} $s_n$.

			On dit alors que $s_n$ est \textbf{accessible} depuis $s_0$.
			Par ailleurs, si $s_n = s_0$, on dit que $s$ est un \textbf{cycle} pour un graphe non-orienté, 
			ou un \textbf{circuit} dans un graphe orienté.

			Si tous les $(s_i)$ sont distincts, on dit que $s$ est \textbf{élémentaire}. 

			\rem Il y a toujours un nombre fini de chaînes élémentaires, mais si $G$ (resp. $H$) a des cycles (resp. des circuits),
			il y a un nombre infini de chaînes (il suffit de tourner en rond...).

			\exo Définir la relation entre les circuits/chemins d'un graphe, 
			qui met en relation deux circuits/chemins ssi ils relient les mêmes sommets.
			Est-ce une relation d'équivalence?

			\prop{La relation $\leftrightarrow $ définie sur $V^2$ par $x\leftrightarrow y$ ssi $x$ est accessible depuis $y$ est une relation d'équivalence.}
			\begin{demo}
				\item Soit $x \in V$. On a bien $x\leftrightarrow x$: la chaîne de longueur $n=0$ $s = (x)$ convient.
				\item Soit $(x,y) \in V^2$, tel que $x \leftrightarrow y$. 
					Alors par définition il existe $s = (s_0,...,s_n) \in V^{n+1}$ tel que $s_0 = x$ et $s_n = y$,
					et $\forall i \in \intsete{0,n}$, $\set{s_i,s_{i+1}} \in E$.
					Considérons $s' = (s_n,s_{n-1},...,s_1,s_0)$. $s'$ est une chaîne reliant $y$ et $x$.
					En effet, $s'_0 = s_n = y$ et $s'_n = s_0 = x$, et $\forall i \in \intsete{0,n}, 
					\set{s'_{i},s'_{i+1}} = \set{s_{n-i},\set{n-i-1} = \set{s_k,s_{k+1}}} \in E$ en posant $k = n-i-1 \in \intsete{0,n}$.
				\item Soit $(x,y,z) \in V^3$ tel que $x\leftrightarrow y$ et $y\leftrightarrow z$.
					Comme $x\leftrightarrow y$, il existe $s = (s_0,...,s_n) \in V^{n+1}$ une chaîne avec $s_0 = x$ et $s_n = y$.
					Comme $y\leftrightarrow x$, il existe $t = (t_0,...,t_m) \in V^{m+1}$ une chaîne avec $t_0 = y$ et $t_m = z$.
					Considérons $u = (s_0,...s_n,t_1,...t_m)$.
					$u$ est bien une chaîne car $\forall i \in \intsete{0,n+m}$, soit $i \in \intsete{0,n}$ et dans ce cas on a 
					$\set{u_i, u_{i+1}} = \set{s_i,s_{i+1}} \in A$, soit $i = n$ et on a $\set{u_i,u_{i+1}} = \set{t_0,t_1} \in E$,
					soit $i \in \inteset{n,n+m}$ et $\set{u_i,u_{i+1}} = \set{t_{i-n},t_{i-n+1}} \in E$.
					On a par ailleurs $u_0 = x$ et $u_{m+n} = z$, donc $x\leftrightarrow z$.
			\end{demo}

			\exo Définir une relation d'équivalence similaire pour $H$, où l'on doit avoir un chemin dans chaque sens entre deux points en relation.

			\definition \textbf{Une composante connexe} de $G$ est une classe d'équivalence pour la relation d'équivalence définie ci-dessus.
			Si $G$ n'admet qu'une composante connexe, on dit que $G$ est un graphe connexe. 
			Dans le cas de la relation d'équivalence sur les graphes orientés, on appelle les classes d'équivalences \textbf{composante fortement connexe}.

			\definition Soit $W \subset V$ avec $W \neq \varnothing$. 
			$W$ est \textbf{connexe} ssi $\forall(x,y) \in W^2$, il existe une chaîne reliant $x$ et $y$.

			\prop{$W$ est une composante connexe ssi $W$ est connexe minimal, 
			c'est à dire si $\forall W' \subset V\setminus\set{W}, W \subset W'$, $W'$ n'est pas connexe.}

			\definition Soit $G' = (V',E')$ un graphe.
			$G'$ est un \textbf{sous-graphe} ssi $V' \subset V$, $E' \subset E$.

			\definition Soit $V' \subset V$
			Le \textbf{graphe induit par $G$ sur $V'$} est $G' = (V', \set{\set{x,y}\in E \tq (x,y) \in V^2})$.

			\prop{Un ensemble de sommets est connexe ssi le graphe qu'il induit est connexe.}
			\begin{demo}
				\item En exercice.
			\end{demo}
	
		\subsection{Types de graphes}
			\definition Un graphe non-orienté (resp. orienté) est dit \textbf{acyclique} s'il ne contient aucun cycle élémentaire (resp. aucun circuit).

			\definition Un \textbf{arbre} est un graphe connexe acyclique. (cf TD pour caractérisation).

			\definition Un graphe acyclique décomposé en ses composantes connexes (qui sont donc des arbres), est appelé \textbf{forêt}.

			\definition Un graphe non-orienté $(V,E)$ est dit \textbf{biparti} ssi il existe une partition $\set{W_1,W_2}$ de $V$ telle que toutes les arêtes de $E$
			aient une extrémité dans $W_1$ et l'autre dans $W_2$.



	\section{Parcours}
		\subsection{Définitions}
			On définit, pour $W \subset V$ la bordure de $W$ par:
				\[ 
					\mathcal{B}(W) = 
					\set{ y \in V\setminus W \tq \exists x \in W, \set{x,y} \in E}
				\]
			Dans le cadre d'un graphe orienté, pour $T \subset S$:
				\[ 
					\mathcal{B}(T) = 
					\set{ y \in V\setminus T \tq \forall x \in T, y \in \mathscr{V}(x) }
				\]

			On dit que $L\in V^n$ (ou $S^n$ pour un graphe orienté) est un parcours ssi 
			$\forall i \in \intset{1,n}, L_i \in \mathcal{B}(\set{L_j \tq j \in \intsete{1,i}})$ ou $\mathcal{B}(\set{L_j \tq j \in \intsete{1,i}})$ est vide.
			
			
			\begin{center}	
			\begin{tikzpicture}
				\node[circle,draw] (a) at (0,0) {A};
				\node[circle,draw] (b) at (1,1) {B};
				\node[circle,draw] (c) at (3,0) {C};
				\node[circle,draw] (d) at (1,-1) {D};
				\node[circle,draw] (e) at (2,-1) {E};
				\node[circle,draw] (f) at (4,-1) {F};
				\node[circle,draw] (g) at (5,1) {G};
				\draw[-stealth] (a) -- (b);
				\draw[-stealth] (a) -- (d);
				\draw[-stealth] (b) -- (c);
				\draw[-stealth] (d) -- (e);
				\draw[-stealth] (e) -- (c);
				\draw[-stealth] (g) -- (c);
				\draw[-stealth] (c) -- (f);
				\draw[-stealth] (f) -- (g);
			\end{tikzpicture} \\
			Un parcours du graphe orienté ci-dessus est $L = [F,G,C,E,A,B,D]$
			\end{center}
			

			De plus, on dit que $L_i$ est un \textbf{point de regénération du parcours} 
			ssi $\mathcal{B}(\set{L_j \tq j \in \intsete{1,i}}) = \varnothing$. 

			Enfin, en notant $\mathcal{R}$ l'ensemble des points de régénération de $L$,
			on dit que $F = (V,P)$ est une forêt d'arborescences associée au parcours $L$ ssi $F$ respecte les trois propriétés suivantes:
			\begin{enumerate}
				\item $\forall i \in \intset{1,n}, L_i \in \mathcal{R}$ ou $\exists j \in \intsete{1,i}, L_i \in \mathscr{V}(L_j), (L_j,L_i) \in P$;
				\item $\forall u \in V \setminus R, \exists! w \in V, (w,u) \in P$ (on dit alors que $w$ est le \textbf{père} de $u$);
				\item $F = (V,P)$. $P$ est minimal parmi les ensemble vérifiant 1.
			\end{enumerate}

			\begin{center}
			\begin{tikzpicture}
				\node[circle,draw] (a) at (0,0) {A};
				\node[circle,draw] (b) at (2,0) {B};
				\node[circle,draw] (f) at (2,-2) {F};
				\node[circle,draw] (e) at (0,-2) {E};

				\node[circle,draw] (g) at (4,-2) {G};
				\node[circle,draw] (h) at (5,-1) {H};
				\node[circle,draw] (i) at (6,-2) {I};

				\node[circle,draw] (c) at (8,0) {C};
				\node[circle,draw] (d) at (10,0) {D};
				\node[circle,draw] (k) at (11,-1) {K};
				\node[circle,draw] (j) at (9,-2) {J};
				\node[circle,draw] (l) at (11,-2) {L};

				\draw (a) -- (b); 
				\draw (a) -- (e); 
				\draw (b) -- (e); 
				\draw (b) -- (f); 
				\draw (e) -- (f); 

				\draw (g) -- (h); 
				\draw (h) -- (i); 
				
				\draw (c) -- (d); 
				\draw (d) -- (k); 
				\draw (k) -- (j); 
				\draw (k) -- (l); 
				\draw (j) -- (l); 
			\end{tikzpicture}\\
			$L = [\underbrace{A,B,E,F}_{r_1=1},\underbrace{G,H,I}_{r_2 = 5},\underbrace{C,D,K,J,L}_{r_3 = 8}]$.
			\end{center}


			\prop{Soit $G = (V,E)$ un graphe non-orienté. Soit $W \subset V$. Si $\mathcal{B}(W) = \varnothing$, 
			alors il n'existe aucune chaîne reliant un sommet de $W$ et un sommet de $V \setminus W$.}
			\begin{demo}
				\item Par l'absurde, supposons qu'il existe une chaîne $\gamma$ de lngueur $l$ telle que $\gamma_0 \in W$ et $\gamma_l \in V\setminus W$.
				On a $\gamma_0 \neq \gamma_l$ donc $l > 0$.
				On peut alors définir $i_0 = \min\set{i \in \intset{1,l} \tq \gamma_i \not\in W}$.
				Par définition de $i_0$, $\gamma_{i_0-1} \in W$.
				Par définition, une chaîne $\set{\gamma_{i_0},\gamma_{i_0}} \in E$, autrement dit $\gamma_{i_0} \in \mathscr{V}(\gamma_{i_0-1})$.
				Donc $\gamma_{i_0} \in \mathcal{B}(W)$: absurde.
			\end{demo}

			\prop{Soit $G = (V,E)$ un graphe non-orienté, et $L = (L_i)_{i \in \intset{1,n}}$ un parcours de $G$.
			Si l'ensemble des points de regénération s'écrit $R = \set{L_{r_k} | k \in \intset{1,K}}$ avec $(r_k)$ croissant,
			alors $G$ admet $K$ composantes connexes, à savoir les $(C_k)_{k \in \intset{1,K}}$ 
			définis par $C_k = \set{L_i \tq i \in \intsete{r_k,r_k+1}}$ avec $r_{K+1} = n+1$.}
			\begin{demo}
				\item Soit $k \in \intset{1,K}$. Montrons que $C_k$ est connexe maximal.
				\item Si $C_k = V$, il est trivialement connexe. Sinon, soit $u\in V\setminus C_k$. MQ $\hat{C_k} = C_k\cup\set{u}$  n'est pas connexe.
				Par définition d'un parcours. Il existe $i_u \in \intset{1,n}$ tq $u = L_{i_u}$.
				Comme $u \not\in C_k, i_u \not\in \intsete{r_k,r_{k+1}}$
				Si $i_u < r_k$, on pose $W = \set{L_i \tq i \in \intsete{1,r_k}}$. Ainsi, $L{i_u} \in W$ et $\mathcal{B} = \varnothing$ car 
				$L_{i_k}$ est point de regénération. D'après le lemme, il n'existe aucun chemin reliant $L_{i_u}$ et $L_{i_k} \not \in W$
				donc $\hat{C_k}$ n'est pas connexe.
				Autre cas...
				Ainsi $C_k$ est maximal.

				\item Par l'absurde, on suppose que $C_k$ n'est pas connexe.
				On peut donc dire qu'il existe $i \in \intsete{r_k,r_k+1}$ tel que $L_{r_k} \not\leftrightarrow$
				On considère alors $i_0 = \min\set{j \in \intsete{r_k, r_{k+1}} \tq L_j \not\leftrightarrow L_{r_k}}$.
				Par minimalité de $i_0$, $\forall j \in \intesete{r_k,i_0}, L_{r_k} \leftrightarrow L_j$.
				donc $\forall j \in \intsete{r_k,i_0}, L_j \not\leftrightarrow L_{i_0}$ 
				(sinon on aurait $L_{i_0} \leftrightarrow L_j \leftrightarrow L_{r_k}$).
				De plus, puisque $L_{r_k} \in R, \mathcal{B}(\set{L_j \tq j \in \intsete{1,r_k}}) = \varnothing $
				Donc d'après le lemme, $\forall j \in \intsete{1,r_k}, L_{i_0} \not\leftrightarrow L_j$.
				Donc $L_{i_0} \not\in \mathcal{B}(\set{L_j \tq j \in \intsete{1,i_0}})$.
				Par définition d'un parcours on a nécessairement
				$\mathcal{B}(...) = \varnothing$ donc $L_{i_0} \in R$: 
				Absurde, car $L_{r_k}$ et $L_{r_{k+1}}$ sont 2 points de régénération consécutifs.
			\end{demo}

			\definition Soit $L = (L_i)_{i \in \intset{1,n}}$ un parcours de $G$.
			Pour $k \in \intset{0,n}$, on appelle \textbf{sommet ouvert à l'étape $k$} (dans L)
			un sommet de $O_k = \set{L_j \tq j \in \intsete{1,k} \text{ et } \mathscr{V}(L_j) \not\subset \set{L_i \tq i \in \intset{1,k}}}$.

			$L$ est un \textbf{parcours en largeur} (resp. profondeur) de $G$ ssi
			$\forall k \in \intset{1,n}$, $L_k$ est un point de régénération 
			ou bien $L_k \in \mathscr{V}(L_{i_0})$ où $i_0 = \min\set{i\in\intset{1,n} \tq L_i \in O_k}$ (resp. max).	
			Autrement, chaque sommet du parocurs est point de régérération ou bien voisin du premier (resp.dernier) sommet ouvert.

			\rem Lorsque l'on s'intéresse à la forêt d'arborescence associée à un parcours en largeur / profondeur, 
			on choisira comme paire d'un sommet $L_k$ le premier / dernier sommet ouvert à l'étape $k$.
			Autrement dit, la forêt justifiera que le parcours est en largeur / profondeur.

			\exemple Un parcours en largeur du graphe ci-dessous est $\underline{A}BCDE\underline{F}HG$, 
			et un parcours en profondeur de $\underline{A}CDEB\underline{F}HG$.
			\begin{center} \begin{tikzpicture}
				\node[circle,draw] (a) at (0,1) {A};
				\node[circle,draw] (b) at (0,0) {B};
				\node[circle,draw] (c) at (1,0) {C};
				\node[circle,draw] (d) at (2,2) {D};
				\node[circle,draw] (e) at (3,1) {E};
				\node[circle,draw] (f) at (2,0) {F};
				\node[circle,draw] (g) at (2,-1) {G};
				\node[circle,draw] (h) at (3,0) {H};
				\draw[-stealth] (a) -- (b);
				\draw[-stealth] (b) -- (c);
				\draw[-stealth] (a) -- (c);
				\draw[-stealth] (c) -- (d);
				\draw[-stealth] (d) -- (e);
				\draw[-stealth] (f) -- (h);
				\draw[-stealth] (f) -- (g);
				\draw[-stealth] (h) -- (e);
				\draw[-stealth] (f) -- (c);
				\draw[-stealth] (h) -- (g);
			\end{tikzpicture} \end{center}

	\section{Algorithmes de parcours}
		\subsection{Détection de composantes connexes.}
			$G = (V,E)$ avec $V = \intset{1,n}$ un graphe non orienté.
			L'algorithme consiste à associer à chaque sommet du graphe le numéro de la composante connexe associée.
			\begin{itemize}
				\item $O$ contient les éléments de la composante connexe actuelle que l'on est en train de traiter
					(les éléments "\textbf{O}uverts" afin de les étudier);
				\item $n_c$ contient le numéro de la composante connexe que l'on est en train d'explorer;
				\item $F$ contient tous les sommets traités (les sommets \textbf{F}ermés).
				\item $T_{res}$ contient les numéros de parties de chaque sommet.
			\end{itemize}
			\begin{algotext}
				|$n$| <- nb_sommets(|$G$|)
				|$O$| <- ensemble de sommets, initialement vide
				|$F$| <- ensemble de sommets, initialement vide
				|$T_{res}$| <- tableau indicé par |$\intset{1,n}$|, initialisé à -1.
				|$n_c$| <- 0 
				racine <- 1
				|$T_{res}$|[racine] <- |$n_c$|
				|$O$|.ajouter(racine)
				|$i$| <- 0
				|$\textsl{Invariants:}$|
					|- $ \textsl{les éléments de $F$ sont dans $n_c$ composantes connexes différentes;}$|
					|- $ \textsl{il y a $i$ éléments différents de -1 dans $T$; ce sont les éléments de $F$;}$|
					|- $ \textsl{les éléments portant le même numéro dans $T$ sont accessibles entre eux.}$|
				Tant que |$i < n$|:
					Si |$O$ est vide|:
						racine <- numéro d'un sommet |$S$| tel que |$T_{res}[s] = -1$|
						|$n_c$| <- |$n_c$| + 1
						|$O$|.ajouter(racine)
						|$T_{res}$[racine]| <- |$n_c$|
					|$u$| <- un sommet |$\texttt{\underline{extrait}}$| de |$O$|
					|Pour tout $v$ voisin de $u$|
					|	Si $T_{res}[v]$ = -1:|
					|		$O$.ajouter($v$)|
					|		$T_{res}[v]$ <- $n_c$|
					|$F$.ajouter($u$)|
					|$i$++|
				Renvoyer |$T_{res}$|
			\end{algotext}
		\subsection{Détection de graphe biparti.}
			\plabel{Algorithme pour décider si un graphe est biparti : } Considérons un graphe $G$.
			On parcourt récursivement (à partir d'un point de régénération, puis ses successeurs...)
			les composantes connexes graphe en associant à chaque sommet une couleur (ici représentée par 0 ou 1).
			Si jamais il y a conflit entre deux associations de couleur 
			(un sommet qui s'est vu associer la couleur 0 auquel on essaierait d'associer la couleur 1 par exemple),
			le graphe n'est pas biparti.
			Si par contre on arrive à parcourir tout le graphe sans conflit, le graphe est bien biparti.

			Dans le cas où $G=(V,E)$ avec $V = \intset{1,n}$:
			\begin{itemize}
				\item $T$ est un tableau contenant la couleur du sommet $i$, -1 si aucune couleur ne lui a été assignée;
				\item $O$ désigne l'ensemble des sommets déjà colorés desquels on colorie les voisins.
			\end{itemize}
			\begin{algotext}
				|$n$| <- nb_sommets(|$G$|)
				|$T$| <- tableau d'entiers indicé par |$\intset{1,n}$| initialisé à -1
				|$O$| <- |$\set{1}$|
				couleur <- 0
				|$T[1]$| <- couleur
				|$\textsl{Invariants: }$|
					|- $ \textsl{le sous graphe induit par $G$ sur $V' = \set{i \in \intset{1,n} \tq T[i] \neq -1}$ est biparti;}$|
					|- $ \textsl{il y a $i$ éléments différents de -1 dans $T$.}$|
				Pour |$i$| allant de |$1$| à |$n$|:
					Si |$O$| est vide:
						|$u$| <- élément de |$\intset{1,n}$| tel que |$T[u] = -1$|
						|$T[u]$| = couleur
					Sinon:
						|$u$| <- élément extrait de |$O$|
					couleur = 1 - |$T[u]$|
					Pour |$v$| voisin de |$u$|:
						Si |$T[v] = -1$|
							|$T[v]$| <- couleur
							ajouter |$v$| à |$O$|
						Sinon:
							Si |$T[v] \neq$| couleur:
								Renvoyer Faux
				Renvoyer Vrai
			\end{algotext}

		\subsection{Plus court chemin en nombre d'arcs.}
			\definition Dans un graphe non-orienté, on appelle \textbf{distance entre les sommets $a$ et $b$}, notée $\dist(a,b)$,
			la longueur inimale d'une chaîne reliant $a$ et $b$ (et $+\infty$ s'il n'en existe pas).
			On généralise cette notion aux graphes orientés, mais il se s'agit alors plus vraiment d'une distance: 
			la distance de $a$ à $b$ n'est pas forcément égale à la distance de $b$ à $a$! 

			Pour cet algorithme, on considère $G = (S,A)$ graphe orienté avec $S = \intset{1,n}$ et $(a,b)\in S^2$.
			On cherche la distance en nombre d'arcs de $a$ à $b$. 
			On parcourt récursivement le graphe à partir de $a$, et dès que l'on rencontre un sommet, 
			on lui assigne sa distance à $a$ - alors minimale - jusqu'à atteindre $b$.
			Pour garder une trace du chemin optimal, on stocke pour chaque élément son prédecesseur.
			\begin{itemize}
				\item $O_1$ représente la liste des sommets à traîter, $O_2$ la liste de leurs successeurs.
				\item \texttt{prof} est le "numéro de la génération": le premier sommet dont on part représente la génération 0,
				ses successeurs la génération 1, et ainsi de suite.
				\item $D$ contient les distances des différents sommets du graphe par rapport au point $a$.
				\item $P[i]$ est le numéro du point précédent le sommet numéro $i$ dans l'éventuel plus court chemin.
				\item $F$ est l'ensemble des sommets traités par l'algorithme dont les successeurs ont aussi été traités.
			\end{itemize}

			\begin{algotext}
				|$n$| <- nb_sommets(|$G$|)
				|$O_1$ <- pile d'entiers initialisée vide|
				|$O_2$ <- pile d'entiers initialisée vide|
				|$F$ <- ensemble initialisé vide|
				|$D$ <- tableau d'entiers indicé par $\intset{1,n}$ initialisé à $+\infty$|
				|$P$ <- tableau d'entiers indicé par $\intset{1,n}$ initialisé à $+\infty$|
				prof <- 0
				|$O_1$|.ajouter(|$a$|)	
				|$D[a]$| <- 0
				|$P[a]$| <- |$a$|

				|$\textsl{Invariants: }$|
					|- $ \textsl{pour tout sommet $f$ de $F$, $d(a,f) = D[a] \leq \mathtt{prof}$ ;}$|
					|- $ \textsl{$F$ est inclus dans la composante connexe $W$ contenant $a$;}$|
					|- $ \textsl{$O_2$ est la bordure de $F$.}$|
				Tant que |$O_1$| et |$O_2$| ne sont pas vides:
					Si |$O_1$| est vide:
						Transvaser |$O_2$| dans |$O_1$|
						prof = prof + 1
					|$u$| <- |$O_1$|.extraire_sommet
					Pour |$v$| successeur de |$u$|:
						Si |$D[v] = +\infty$|:
							|$D[v]$| <- |$D[u] + 1$|
							|$P[v]$| <- |$u$|
							|$O_2$.ajouter($v$)|
					|$F$.ajouter($u$)|
				Renvoyer |$D[b]$|
			\end{algotext}
			
	\section{Plus court chemin}
		\definition Pour un graphe orienté $G = (V,E)$, on dit que $w$ est une \textbf{pondération des arêtes}		
		ssi $w \in \mathcal{F}(E,\mathbb{R})$.

		On dit alors que $G$ est un \textbf{graphe pondéré}.

		\definition Soit $H = (S,A,w)$ un graphe orienté.
		Si $\gamma = (\gamma_i)_{i\in\intset{0,n}}$ est une chaîne ou un chemin de $G$,
		sa longueur est $\text{long}(\gamma) = \sum_{i=0}^{n+1}w(\gamma_i,\gamma_{i+1})$.

		On définit alors la distance entre deux sommets de $S$ par
		\[
			d(a,b) = \min\set{\text{long}(\gamma) \tq \gamma \text{ est un chemin reliant $a$ et $b$}}
		\]

		\rem Ces définition se généralisent naturellement aux graphes non-orientés.
		
		\subsection{Algorithme de Dijkstra}
			Considérons $G = (S,A,w)$ orienté, pondéré \textbf{positivement}, où $S = \intset{1,n}$, $s \in S$ un sommet de départ. 
			\begin{itemize}
				\item 
			\end{itemize}
			\begin{algotext}
				|$n$| <- nombre de sommets de |$G$|
				|$D$| <- tableau indicé par |$\intset{1,n}$|, initialisé à |$+\infty$|
				|$P$| <- tableau indicé par |$\intset{1,n}$|, initialisé à -1.
				|$D[S] = 0$|
				|$P[S] = s$|

				|$O$| <- |$\set{0}$|
				|$F$| <- |$\varnothing$|
				Tant que |$O$| est non vide :
					|$u$| <- extraire de |$O$| l'élément minimisant |$D$| 		// (|$u = \argmin\set{D[i] \tq i \in O}$|)
					Pour |$v \in \text{succ}(u)$|:
						Si |$D[v] = +\infty$|:
							|$O$|.ajouter(|$v$|)
						Si |$D[u] + w(u,v) < D[v]$|:
							|$D[v]$| <- |$D[u] + w(u,v)$| 
							|$P[v]$| <- |$u$|
					|$F$|.ajouter(|$u$|)
				renvoyer |$D$|
			\end{algotext}

			\prop{Soit $G = (S,A,w)$ un graphe orienté pondéré positivement. Soit $s \in S$. 
			Le tableau $D$ retourné par l'algorithme ci-dessus est tel que $\forall u \in S,  D[u] = d(s,u)$.}
		\subsection{Démonstration}
			\definition Pour $X \subset S$, et $(u,v) \in S^2$, on note $C_X(u,v)$ l'ensemble des chemins élémentaires de $u$ à $v$ dans $G$,
			dont tous les sommets sauf éventuellement $v$ sont dans $X$ et $d_X(u,v)$ la longueur minimale d'un chemin dans $C_X(u,v)$.

			\plabel{Invariant : } \textsl{Soit $(G,S)$ une entrée de l'algorithme. On admet que l'appel termine. 
			Soit $K$ le nombre de tours de boucle \texttt{tant que} effectués.
			Pour $k \in \intset{0,K}$, on note $O_k, F_k, D_k$ l'état des ensembles $O$ et $F$ et le tableau $D$ à la fin du $k$-ième tour de boucle.}
			\begin{enumerate}[label=(\alph*)]
				\item Pour tout $u \in S$, pour tout $k \in \intset{0,K}, D_k[u] = d_{D_k}{s,u}$;
				\item $\forall u \in F_k, D_k[u] = d_(s,u)$;
				\item $O_k \cap F_k \neq \varnothing$ et $O_k\cup F_k = \set{i \in \intset{1,n} D[i] < +\infty}$.
			\end{enumerate}

			\begin{demo}
				\item On procède par récurrence finie:
				\item Au rang $k=0$, on a:
				\begin{itemize}
					\item $F_0 \neq \varnothing$ donc $(b)$ est vraie;
					\item On a aussi $O_k \cap F_k = \varnothing$.
					De plus $O_0 = \set{s}$ donc $O_0\cup F_0 = \set{s}$, 
					or $s$ est le seul sommet $i$ pour lequel $D_0[i] \neq +\infty$, donc on a bien (c).
					\item $\forall u \in S$, $C_{F_0}(s,u) = \set{(s)}$ si $u=s$, $\varnothing$ sinon.
					donc $d_{F_0}(s,u) = 0$ si $u = s$, $+\infty$ sinon
					On a donc bien (a)
				\end{itemize}
				\item Soit $k \in \intsete{0,K}$. On suppose $P_k$. On veut montrer $P_{k+1}$.
				Soit $x \in O_k$ le sommet extrait de $O$ à l'étape $k+1$.
				On a alors $F_{k+1} = F_k \cup {x}$.
				De plus, $D_k[x] = \min\set{D_k[i] \tq i \in O_k}$.
				$D_{k+1}[u] = D_k[u]$ si $u \not\in\text{succ}(x)$, $\min\set{D_k{x} + w(x,u),D_k[u]}$ sinon.
				$F_k \subset F_{k+1}$, donc $\forall u \in S$, $D_{F_k}(s,u) \geq d_{F_{k+1}}(s,u)$.
				$O_{k+1} = (O_k \cup \set{u \in \text{succ}(x) \tq D_k[u] = +_\infty})\setminus\set{x}$
				\begin{itemize}
					\item On remarque que nécessairement $D_k[s] = 0$, $s \in F_k$ (car $F_1 = \set[{s}$ et $(F_i)$ croissante), donc $d_{F_k}(s,s) \leq \text{long}(s) = 0$,
					soit $d_{F_{k+1}}(s,s) = 0 = D_{k+1}[s]$.
					\item Soit $u \in S \setminus{s}$.
					Soit $c = (c_i)_{i \in \intset{0,l}}\in C_{F_{k+1}}(s,u)$ (nécessairement $c_0 = s$ et $c_l = u$).
					On note alors $p = c_{l-1}$. Aini $c = \overbrace{s..p}^{\widetilde{c}} u$.

					\plabel{Cas 1: $p \neq x$ : } Alors $p \in F_{k+1} \setminus \set{x} = F_k$.
					[On remarque que $\forall y \in F_k, d_{F_k}(s,y) = D_k[y] = d(S,y)$.
					Donc $\forall y \in F_k, \exists c \in C_{F_k}(s,y)$ de longueur $(d,x,y)$.] ($\theta$)
					Ici en particulier, il existe $\widetilde{b} \in C_{F_k}(s,p)$ de longueur $d(s,p)$.
					Ainsi, le chemin $d = \widetilde{b}.u \in C_{F_k}(s,u)$, 
					et $\text{long}(b) = \text{long}(\widetilde{b}) + w(p,u) = d(s,p) + w(p,u) \leq \underbrace{\text{long}(\widetilde{c}) + w(p,u)}_{\text{long}(c)}$.
					donc $d_{F_k}(s,u) \leq \text{long}(b) \leq \text{long}(c)$ d'où long$(c) \geq D_k[u]$.

					\plabel{Cas 2: $p = x$ :}
					... long$(c) \geq D_k[x] + w(x,u)$. 
				\end{itemize}	
			\end{demo}
\end{document}

