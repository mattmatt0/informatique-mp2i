\newcommand\PATH{Lancez `make adapt avant de compiler!`}

\RequirePackage{fix-cm}
\documentclass{scrartcl}

% Fichier pour les dépendances

% Fontes Computer Modern
\usepackage{lmodern}

% Minted
\usepackage[cache=true,outputdir=\PATH/obj]{minted}

% Pour les symboles et outils mathématiques
\usepackage{amsmath}
\usepackage{amsfonts}
\usepackage{amssymb}
\usepackage{mathtools}
\usepackage{cancel}
\usepackage{mathrsfs}
\usepackage{stmaryrd}

% Pour les dessins / images
\usepackage{xcolor}
\usepackage{tikz}
\usepackage{graphicx}

% Tableaux et listes
\usepackage{tabularx}
\newcolumntype{Y}{>{\centering\arraybackslash}X}
\usepackage{booktabs}
\usepackage{multirow}
\usepackage{enumitem}
\usepackage{multicol}
\usepackage{adjustbox}

% Présentation de la page (bordures)
\usepackage[a4paper,top=1cm,bottom=1cm,includefoot,left=1.5cm,right=1.5cm,footskip=1cm]{geometry}
\usepackage{scrlayer-scrpage}
\rofoot*{\pagemark}
\cofoot*{}
\pagestyle{scrheadings}

% Autres
\usepackage{ulem} % Soulignage
\usepackage[french]{babel} % Babel, pour respecter les standards français
\frenchbsetup{StandardLists=true} % ... sauf pour les immondes listes à tirets
\usepackage[T1]{fontenc}

\usepackage{hyperref} % Pour les liens
\hypersetup{
    colorlinks=true,
    linkcolor=blue,
    filecolor=magenta,
    urlcolor=cyan,
}


%% Commandes personnalisées

% Code
\newenvironment{code}[1]
{\VerbatimEnvironment\begin{minted}[breaklines,frame=single,autogobble,linenos,bgcolor=codebg,tabsize=4,mathescape=true]{#1}}
{\end{minted}}
\definecolor{codebg}{gray}{1}
\newenvironment{algotext}
{\VerbatimEnvironment\begin{minted}[frame=single,autogobble,linenos,bgcolor=codebg,tabsize=4,escapeinside=||,mathescape=true]{text}}
{\end{minted}}


% Paragraphes spéciaux
\newcounter{compteurExos}
\setcounter{compteurExos}{0}
\newcommand{\plabel}[1]{{\parindent10pt \fontfamily{cmss}\selectfont\textbf{#1} \,}}

\newcommand{\prop}[1]{\plabel{Propriété :} \textsl{#1}}
\newcommand{\lemma}[1]{\plabel{Lemme :} \textsl{#1}}
\newcommand{\rem}{\plabel{Remarque :}}
\newcommand{\exemple}{\plabel{Exemple :}}
\newcommand{\exo}{\stepcounter{compteurExos}\plabel{Exercice \thecompteurExos :}}
\newcommand{\obs}{\plabel{Observation :} }

\newenvironment{demo}{\begin{itemize}[label=$\triangleright$]}{\end{itemize}}
\newcommand{\definition}{\parindent0pt }
\newcommand{\semidef}{\parindent0pt $\bullet$ \,}

% Outils
\newcommand{\corrpar}{\vspace{-20pt}}
\newcommand{\extraspace}{\vspace{5pt}}

% Configuration
\parskip5pt

% Annonce pour le développement:
\newcommand{\notecentrale}[1]{\begin{center}\textsl{#1}\end{center}}
\newcommand{\warnwip}{\notecentrale{Ce cours n'est pas encore bien relu}}
\newcommand{\warnwipnext}{\notecentrale{La suite de ce cours est encore en rédaction}\message}


\newcommand{\fpq}{\mathbb{F}_p(Q)}
\newcommand{\abins}{\mathscr{A}_B(\mathcal{S})}
\newcommand{\pb}[3]
{
	\textbf{#1}
	\left|\left|\begin{tabular}{l l}
		Entrée: & #2 \\
		\vspace{-10pt} & \\
		\hline
		\vspace{-10pt} & \\
		Sortie: & #3
	\end{tabular}\right.\right.
}
\newcommand{\pbinlist}[3]
{
	\textbf{#1} &  $\left|\left|\begin{tabular}{l l}
		Entrée: & #2\\ 
		\vspace{-10pt} & \\
		\hline
		\vspace{-10pt} & \\
		Sortie: & #3 
	\end{tabular}\right.\right.$ 
}
\newcommand{\val}{\text{val}}
\newcommand{\argmax}{\mathop{\text{argmax}}}
\newcommand{\argmin}{\mathop{\text{argmin}}}

\newcommand{\set}[1]{{\{#1\}}}
\newcommand{\intset}[1]{\llbracket #1 \rrbracket}
\newcommand{\intsete}[1]{\llbracket #1 \llbracket}
\newcommand{\inteset}[1]{\rrbracket #1 \rrbracket}
\newcommand{\intesete}[1]{\rrbracket #1 \llbracket}

\newcommand{\tq}{\, \big| \,}

\newcommand{\card}{\.\textmd{card}}
\newcommand{\Id}{\textsl{Id}}
\newcommand{\supp}{\text{supp}}

% Partie entière, arrondi supérieur

\DeclarePairedDelimiter\ceil{\lceil}{\rceil}
\DeclarePairedDelimiter\floor{\lfloor}{\rfloor}



\usepackage{lmodern}

\title{Bases de données}
\author{}
\date{}

\begin{document}
	\maketitle
	\section{Introduction}
		L'objet des bases de données (abrégé BD) est la gestion d'informations,
		qui peuvent se révéler nombreuses et variées dans des applications réelles.

		Il s'agit de trois choses:
		\begin{enumerate}
			\item Stocker les informations de manière persistente,
			\item Permettre l'ajout, la suppression et la modification de ces informations,
			\item Rendre facilement (de manière aisée pour des non-spécialistes) et efficacement accessible ces informations.
		\end{enumerate}

		On entend par facilement "de manière aisée pour des non-spécialistes". 
		On utilise pour cela une langage d'interface simple.

		Les systèmes de gestion de bases de données (abrégé SGBD) assurent le stockage physique des informations,
		et les opérations de création / ajout / modification et recherche, 
		pourvu qu'elles soient formulées dans un langage adéquat, appelé \textbf{langage de requête}.
		Pour cela, les informations doivent être organisées dans une base de données, 
		que l'on peut définir comme un ensemble d'informations structurées.

		La structure d'une base de données est définie par ses \textbf{tables}, et les \textbf{colonnes} de ces tables.
		Chaque table est identifiée par son nom, et contient un nombre fixé de colonnes.
		Chaque colonne est identifiée par son nom, et fixe un \textbf{domaine} de valeurs possibles:
		entiers, chaînes de caractères, ou flottants.
		Ce sont les lignes de ces tables, appelées \textbf{enregistrements}, qui contiennent ces informations.

		\rem La définition de base de données n'interdit pas la multiplicité des lignes. 
		En pratique, on s'interdira de stocker deux fois la même ligne dans le même tableau.

		Si l'on fixe l'ordre des $n$ colonnes d'une table, et que l'on note 
		$(D_i)_{i\in\intset{1,n}}$ les domaines de ces colonnes, chaque enregistrement est un $n$-uplet
		de $D_1\times...\times D_n$.

		Le nombre de lignes n'est pas limité, et leur ordre n'est pas significatif, de sorte qu'une table puisse être vue 
		comme un sous-ensemble de $D_1\times...\times D_n$, que l'on appelle aussi une relation sur $D_1\times...\times D_n$.
		
		\rem Techniquement, il ne s'agit pas réellement d'un sous-ensemble à cause de la multiplicité possible, mais plutôt d'un multi-ensemble.
		Mais on évitera toujours de telles redondances via l'identification d'une clé (voir plus loin).

		La gestion d'une base de données consiste donc en ces trois points:
		\begin{enumerate}
			\item la conception de la base de données, c'est à dire décider de la structure la plus adéquate pour une application donnée ;
			\item l'acquisition des données et leur enregistrement dans la base ;
			\item la valorisation des données, c'est à dire l'extraction d'informations pertinentes.
		\end{enumerate}

	\section{Le modèle entité/association (E/A)}
		Le modèle entité-association est un outil qui permet de représenter la structure d'une base de données. 
		Il est utile en phase de conception, ou pour visualiser une base de données existante.

		\exemple On cherche à enregistrer un groupe de personnes, l'endroit où elles habitent, la capitale du pays d'où elles viennent et leur âge.

		\begin{center}
		\begin{tabular}{|c|c|c|c|c|}
			\hline
			Nom & Prénom & Pays de résidence & Capitale & Date de naissance \\
			\hline
			DURAND & Paul & Pays-Bas & Amsterdam & 26/09/1981 \\
			\hline
			WILES & Andrew & Royaume-Uni & Londres & 11/04/1953 \\
			\hline
			... & ... & ... & ... & ... \\
			\hline
		\end{tabular}
		\end{center}

		\rem On a ici une redondance entre le pays de résidence et sa capitale: cela pose des problèmes d'espace occupé,
		mais surtout si on doit faire une correction (imaginons qu'un pays change de capitale!), il faut s'assurer que
		le changement soit fait partout. Il serait ici plus adapté de stocker dans une autre table les différents pays
		et leur capitale, et ne stocker alors que le pays dans la table précédente.

		\rem Il faut aussi éviter les doublons, qui pourront poser des problèmes de dénombrement, de cohérence et de mise à jour.
		Pour cela, il faut identifier ce qui caractérise un enregistrement, c'est à dire la \textbf{clé} de cet enregistrement,
		et imposer que deux enregistrement aient deux clés différentes.

		La clé d'une table est donc un sous-ensemble de ses colonnes, telle que deux lignes de la table ne coïncident pas sur ces colonnes.
		Il y a parfois plusieurs clés possibles. Celle qu'on indique au SGBD est appelée clé primaire.

		Une bonne clé est:
		\begin{itemize}
			\item composée de données stables dans le temps, peu sujettes à des modifications ;
			\item composée de données dont on dispose pour chaque enregistrement ;
			\item la plus petite possible.
		\end{itemize}
		
		Si parmi les données aucune ne satisfait ces contraintes, on ajoutera un identifiant artificiel qui servira de clé.
	
		\begin{center}	
		\begin{tikzpicture}
		\node[anchor=north west] (tableau1) at (0,0) {
			\begin{tabular}[t]{|c|}
				\hline
				\texttt{Personne : str}\\
				\hline
				\textcolor{orange}{\texttt{\textbf{id}: int}} \\
				\texttt{Nom : str} \\
				\texttt{Prénom : str } \\
				\texttt{Date de naissance : int} \\
				\texttt{Ville de naissance : int} \\
				\hline
			\end{tabular}
		};
%		0*
%			Loge à
%		1..1
		\node[anchor=north west] (tableau2) at (12,0) {
		\begin{tabular}[t]{|c|}
			\hline
			\texttt{Pays}\\
			\hline
			\textcolor{orange}{\texttt{\textbf{Nom} : string}} \\
			\texttt{Capitale : string} \\
			\texttt{Nombre d'habitants : int} \\
			\hline
		\end{tabular}};
		\draw[->] (5.6,-0.4) -- (12,-0.4) node[pos=0.1,above]{1..*} node[pos=0.5,above] {Habite} node[pos=0.5,below] {
			\begin{tabular}{|c|}
				\hline
				\texttt{Habite} \\
				\hline
				\textcolor{orange}{\textbf{\texttt{Date début : int}}} \\
				\texttt{Date fin : int} \\
				\hline
			\end{tabular}} node[pos=0.9,above] {1..1}; 
		\end{tikzpicture} \\
		\textsc{Représentation de deux entités et d'une association entre elles}. Les lignes en orange désignent les clés.
		\end{center}

		Dans le modèle entité-association, on différencie les tables qui modélisent les objets de celles qui modélisent les liens entre ces objets.
		Les premiers sont appelés entités, et les seconds associations.

		On définit un type d'entité par la donnée de son nom, ses attributs avec leurs domaines, et sa clé.
		On définit une association binaire par son nom, les deux entités qu'elle relie (A et B), des cardinalités $c_A$ et $c_B$, éventuellement des attributs.
		Chaque clé de A apparaît autant de fois qu'indiqué par $c_A$ dans la table de la relation.

		Toutes ces données se représentent comme ci-dessus.
		\textsl{Reprendre la notion de cardinalité lundi...}

\end{document}
