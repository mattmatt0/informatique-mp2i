\newcommand\PATH{/home/matthieu/Documents/Ecole/Informatique/cours/13-bases-de-donnees.latex}

\RequirePackage{fix-cm}
\documentclass[a4paper]{scrartcl}

% Fichier pour les dépendances

% Fontes Computer Modern
\usepackage{lmodern}

% Minted
\usepackage[cache=true,outputdir=\PATH/obj]{minted}

% Pour les symboles et outils mathématiques
\usepackage{amsmath}
\usepackage{amsfonts}
\usepackage{amssymb}
\usepackage{mathtools}
\usepackage{cancel}
\usepackage{mathrsfs}
\usepackage{stmaryrd}

% Pour les dessins / images
\usepackage{xcolor}
\usepackage{tikz}
\usepackage{graphicx}

% Tableaux et listes
\usepackage{tabularx}
\newcolumntype{Y}{>{\centering\arraybackslash}X}
\usepackage{booktabs}
\usepackage{multirow}
\usepackage{enumitem}
\usepackage{multicol}
\usepackage{adjustbox}

% Présentation de la page (bordures)
\usepackage[a4paper,top=1cm,bottom=1cm,includefoot,left=1.5cm,right=1.5cm,footskip=1cm]{geometry}
\usepackage{scrlayer-scrpage}
\rofoot*{\pagemark}
\cofoot*{}
\pagestyle{scrheadings}

% Autres
\usepackage{ulem} % Soulignage
\usepackage[french]{babel} % Babel, pour respecter les standards français
\frenchbsetup{StandardLists=true} % ... sauf pour les immondes listes à tirets
\usepackage[T1]{fontenc}

\usepackage{hyperref} % Pour les liens
\hypersetup{
    colorlinks=true,
    linkcolor=blue,
    filecolor=magenta,
    urlcolor=cyan,
}


%% Commandes personnalisées

% Code
\newenvironment{code}[1]
{\VerbatimEnvironment\begin{minted}[breaklines,frame=single,autogobble,linenos,bgcolor=codebg,tabsize=4,mathescape=true]{#1}}
{\end{minted}}
\definecolor{codebg}{gray}{1}
\newenvironment{algotext}
{\VerbatimEnvironment\begin{minted}[frame=single,autogobble,linenos,bgcolor=codebg,tabsize=4,escapeinside=||,mathescape=true]{text}}
{\end{minted}}


% Paragraphes spéciaux
\newcounter{compteurExos}
\setcounter{compteurExos}{0}
\newcommand{\plabel}[1]{{\parindent10pt \fontfamily{cmss}\selectfont\textbf{#1} \,}}

\newcommand{\prop}[1]{\plabel{Propriété :} \textsl{#1}}
\newcommand{\lemma}[1]{\plabel{Lemme :} \textsl{#1}}
\newcommand{\rem}{\plabel{Remarque :}}
\newcommand{\exemple}{\plabel{Exemple :}}
\newcommand{\exo}{\stepcounter{compteurExos}\plabel{Exercice \thecompteurExos :}}
\newcommand{\obs}{\plabel{Observation :} }

\newenvironment{demo}{\begin{itemize}[label=$\triangleright$]}{\end{itemize}}
\newcommand{\definition}{\parindent0pt }
\newcommand{\semidef}{\parindent0pt $\bullet$ \,}

% Outils
\newcommand{\corrpar}{\vspace{-20pt}}
\newcommand{\extraspace}{\vspace{5pt}}

% Configuration
\parskip5pt

% Annonce pour le développement:
\newcommand{\notecentrale}[1]{\begin{center}\textsl{#1}\end{center}}
\newcommand{\warnwip}{\notecentrale{Ce cours n'est pas encore bien relu}}
\newcommand{\warnwipnext}{\notecentrale{La suite de ce cours est encore en rédaction}\message}


\newcommand{\fpq}{\mathbb{F}_p(Q)}
\newcommand{\abins}{\mathscr{A}_B(\mathcal{S})}
\newcommand{\pb}[3]
{
	\textbf{#1}
	\left|\left|\begin{tabular}{l l}
		Entrée: & #2 \\
		\vspace{-10pt} & \\
		\hline
		\vspace{-10pt} & \\
		Sortie: & #3
	\end{tabular}\right.\right.
}
\newcommand{\pbinlist}[3]
{
	\textbf{#1} &  $\left|\left|\begin{tabular}{l l}
		Entrée: & #2\\ 
		\vspace{-10pt} & \\
		\hline
		\vspace{-10pt} & \\
		Sortie: & #3 
	\end{tabular}\right.\right.$ 
}
\newcommand{\val}{\text{val}}
\newcommand{\argmax}{\mathop{\text{argmax}}}
\newcommand{\argmin}{\mathop{\text{argmin}}}

\newcommand{\set}[1]{{\{#1\}}}
\newcommand{\intset}[1]{\llbracket #1 \rrbracket}
\newcommand{\intsete}[1]{\llbracket #1 \llbracket}
\newcommand{\inteset}[1]{\rrbracket #1 \rrbracket}
\newcommand{\intesete}[1]{\rrbracket #1 \llbracket}

\newcommand{\tq}{\, \big| \,}

\newcommand{\card}{\.\textmd{card}}
\newcommand{\Id}{\textsl{Id}}
\newcommand{\supp}{\text{supp}}

% Partie entière, arrondi supérieur

\DeclarePairedDelimiter\ceil{\lceil}{\rceil}
\DeclarePairedDelimiter\floor{\lfloor}{\rfloor}



\usepackage{lmodern}
\usetikzlibrary{calc}
\usetikzlibrary{positioning}
\usetikzlibrary{matrix}

\title{Bases de données}
\author{}
\date{}

\begin{document}
	\maketitle
	\section{Introduction}
		L'objet des bases de données (abrégé BD) est la gestion d'informations,
		qui peuvent se révéler nombreuses et variées dans des applications réelles.

		Il s'agit de trois choses:
		\begin{enumerate}
			\item Stocker les informations de manière persistente,
			\item Permettre l'ajout, la suppression et la modification de ces informations,
			\item Rendre facilement (de manière aisée pour des non-spécialistes) et efficacement accessible ces informations.
		\end{enumerate}

		On entend par facilement "de manière aisée pour des non-spécialistes". 
		On utilise pour cela une langage d'interface simple.

		Les systèmes de gestion de bases de données (abrégé SGBD) assurent le stockage physique des informations,
		et les opérations de création / ajout / modification et recherche, 
		pourvu qu'elles soient formulées dans un langage adéquat, appelé \textbf{langage de requête}.
		Pour cela, les informations doivent être organisées dans une base de données, 
		que l'on peut définir comme un ensemble d'informations structurées.

		La structure d'une base de données est définie par ses \textbf{tables}, et les \textbf{colonnes} de ces tables.
		Chaque table est identifiée par son nom, et contient un nombre fixé de colonnes.
		Chaque colonne est identifiée par son nom, et fixe un \textbf{domaine} de valeurs possibles:
		entiers, chaînes de caractères, ou flottants.
		Ce sont les lignes de ces tables, appelées \textbf{enregistrements}, qui contiennent ces informations.

		\rem La définition de base de données n'interdit pas la multiplicité des lignes. 
		En pratique, on s'interdira de stocker deux fois la même ligne dans le même tableau.

		Si l'on fixe l'ordre des $n$ colonnes d'une table, et que l'on note 
		$(D_i)_{i\in\intset{1,n}}$ les domaines de ces colonnes, chaque enregistrement est un $n$-uplet
		de $D_1\times...\times D_n$.

		Le nombre de lignes n'est pas limité, et leur ordre n'est pas significatif, de sorte qu'une table puisse être vue 
		comme un sous-ensemble de $D_1\times...\times D_n$, que l'on appelle aussi une relation sur $D_1\times...\times D_n$.
		
		\rem Techniquement, il ne s'agit pas réellement d'un sous-ensemble à cause de la multiplicité possible, mais plutôt d'un multi-ensemble.
		Mais on évitera toujours de telles redondances via l'identification d'une clé (voir plus loin).

		La gestion d'une base de données consiste donc en ces trois points:
		\begin{enumerate}
			\item la conception de la base de données, c'est à dire décider de la structure la plus adéquate pour une application donnée ;
			\item l'acquisition des données et leur enregistrement dans la base ;
			\item la valorisation des données, c'est à dire l'extraction d'informations pertinentes.
		\end{enumerate}

	\section{Le modèle entité/association (E/A)}
		Le modèle entité-association est un outil qui permet de représenter la structure d'une base de données. 
		Il est utile en phase de conception, ou pour visualiser une base de données existante.

		\exemple On cherche à enregistrer un groupe de personnes, l'endroit où elles habitent, la capitale du pays d'où elles viennent et leur âge.

		\begin{center}
		\begin{tabular}{|c|c|c|c|c|}
			\hline
			Nom & Prénom & Pays de résidence & Capitale & Date de naissance \\
			\hline
			DURAND & Paul & Pays-Bas & Amsterdam & 26/09/1981 \\
			\hline
			WILES & Andrew & Royaume-Uni & Londres & 11/04/1953 \\
			\hline
			... & ... & ... & ... & ... \\
			\hline
		\end{tabular}
		\end{center}

		\rem On a ici une redondance entre le pays de résidence et sa capitale: cela pose des problèmes d'espace occupé,
		mais surtout si on doit faire une correction (imaginons qu'un pays change de capitale!), il faut s'assurer que
		le changement soit fait partout. Il serait ici plus adapté de stocker dans une autre table les différents pays
		et leur capitale, et ne stocker alors que le pays dans la table précédente.

		\rem Il faut aussi éviter les doublons, qui pourront poser des problèmes de dénombrement, de cohérence et de mise à jour.
		Pour cela, il faut identifier ce qui caractérise un enregistrement, c'est à dire la \textbf{clé} de cet enregistrement,
		et imposer que deux enregistrement aient deux clés différentes.

		La clé d'une table est donc un sous-ensemble de ses colonnes, telle que deux lignes de la table ne coïncident pas sur ces colonnes.
		Il y a parfois plusieurs clés possibles. Celle qu'on indique au SGBD est appelée clé primaire.

		Une bonne clé est:
		\begin{itemize}
			\item composée de données stables dans le temps, peu sujettes à des modifications ;
			\item composée de données dont on dispose pour chaque enregistrement ;
			\item la plus petite possible.
		\end{itemize}
		
		Si parmi les données aucune ne satisfait ces contraintes, on ajoutera un identifiant artificiel qui servira de clé.
	
		\begin{center}	
		\begin{tikzpicture}
		\node[anchor=north west] (tableau1) at (0,0) {
			\begin{tabular}[t]{|c|}
				\hline
				\texttt{Personne}\\
				\hline
				\textcolor{orange}{\texttt{\textbf{id}: int}} \\
				\texttt{Nom : str} \\
				\texttt{Prénom : str } \\
				\texttt{Date de naissance : int} \\
				\texttt{Ville de naissance : int} \\
				\texttt{Pays : str} \\
				\hline
			\end{tabular}
		};
		\node[anchor=north west] (tableau2) at (12,0) {
		\begin{tabular}[t]{|c|}
			\hline
			\texttt{Pays}\\
			\hline
			\textcolor{orange}{\texttt{\textbf{Nom} : string}} \\
			\texttt{Capitale : string} \\
			\texttt{Nombre d'habitants : int} \\
			\hline
		\end{tabular}};
		\coordinate (main1) at ($(tableau1.north east) - (0,0.3)$);
		\coordinate (main2) at ($(tableau2.north west) - (0,0.3)$);
		\draw (main1) -- (main2) node[pos=0.1,above]{1..*} node[pos=0.5,above] {Habite} node[pos=0.5,below] {
			\begin{tabular}{|c|}
				\hline
				\texttt{Habite} \\
				\hline
				\textcolor{orange}{\texttt{\textbf{Date début}: int}} \\
				\texttt{Date fin : int} \\
				\hline
			\end{tabular}} node[pos=0.9,above] {1..1}; 
		\end{tikzpicture} \\
		\textsc{Représentation de deux entités et d'une association entre elles}. Les lignes en orange désignent les clés.
		\end{center}

		\rem Ici les éléments stockés en \textbf{ligne} dans la table sont réprésentés en \textbf{colonne}.

		Dans le modèle entité-association, on différencie les tables qui modélisent les objets de celles qui modélisent les liens entre ces objets.
		Les premiers sont appelés entités, et les seconds associations.

		On définit un type d'entité par la donnée de son nom, ses attributs avec leurs domaines, et sa clé,
		et une association binaire par son nom, les deux entités qu'elle relie (A et B), des cardinalités $c_A$ et $c_B$, éventuellement des attributs.
		Chaque clé de A apparaît autant de fois qu'indiqué par $c_A$ dans la table de la relation. 

		On notera la cardinalité $c_A$ sur la ligne liant les deux entités, à côté de A, et $c_B$ à côté de B (cf figure ci-dessus),
		de la manière suivante:
		\begin{itemize}
			\item $c_A = a..b$ signifie qu'il peut y avoir de $a$ à $b$ éléments de l'entité A en relation avec l'entité B.
			Si $a=b$, on s'autorisera à écrire $c_A=a$.
			\item $c_A = a..*$ signifie qu'il y a au moins $a$ éléments de l'entité A en relation avec l'entité B.
			Si $a=0$, on s'autorisera à noter $c_A=*$.
		\end{itemize}
	\section{Simplifications}
		Dans le cadre du programme, on adoptera les principes suivants:
		\begin{itemize}
			\item Éviter les relations ternaires ou plus.
			\item Éviter les associations 0..* vers 0..*
		\end{itemize}

		\exemple On cherche à gérer les différents emplois du temps des élèves et des professeurs dans un lycée. 
		Il faut donc organiser les séances, avec un horaire précis, une classe qui aura une certaine matière avec tel professeur:
		\begin{center}
		\begin{tikzpicture}
			\node[anchor=north west] (tableau1) 
			{
				\begin{tabular}[t]{|c|}
					\hline
					\texttt{Horaire} \\
					\hline
					\textcolor{orange}{\texttt{\textbf{id}}} \\
					\texttt{heure\_debut} \\
					\texttt{heure\_fin} \\
					\hline
				\end{tabular}		
			};

			\node[right=of tableau1.north east, anchor=north west] (tableau2)
			{
				\begin{tabular}[t]{|c|}
					\hline
					\texttt{Classe} \\
					\hline
					\textcolor{orange}{\texttt{\textbf{id}}}\\
					\texttt{nb\_eleves} \\
					\hline
				\end{tabular}
			};

			\node[right=of tableau2.north east, anchor=north west] (tableau3)
			{
				\begin{tabular}[t]{|c|}
					\hline
					\texttt{Salle} \\
					\hline
					\textcolor{orange}{\texttt{\textbf{id}}}\\
					\texttt{nb\_places} \\
					\hline
				\end{tabular}
			};

			\node[right=of tableau3.north east, anchor=north west] (tableau4)
			{
				\begin{tabular}[t]{|c|}
					\hline
					\texttt{Classe} \\
					\hline
					\textcolor{orange}{\texttt{\textbf{id}}}\\
					\texttt{professeur} \\
					\texttt{discipline} \\
					\hline
				\end{tabular}
			};
			\coordinate (main1) at (tableau1.north);
			\coordinate (top1) at ($ (tableau1.north) + (0,0.4) $);
			\coordinate (main2) at (tableau2.north);
			\coordinate (top2) at ($ (tableau2.north) + (0,0.4) $);
			\coordinate (main3) at (tableau3.north);
			\coordinate (top3) at ($ (tableau3.north) + (0,0.4) $);
			\coordinate (main4) at (tableau4.north);
			\coordinate (top4) at ($ (tableau4.north) + (0,0.4) $);
			\draw (main1) -- (top1);
			\draw (main2) -- (top2);
			\draw (main3) -- (top3);
			\draw (main4) -- (top4);
			\draw (top1) -- (top4) node[pos=0.5,below] (seance) {Séance};
			\node[above=of seance.center,anchor=center] (tableau5)
			{
				\begin{tabular}[t]{|c|}
					\hline
					\texttt{Séance} \\
					\hline
					\texttt{...} \\
					\hline
				\end{tabular}
			};
		\end{tikzpicture}\\
		\textsc{Représentation d'une séance comme association}. Les cardinalités n'ont pas été représentées.
		\end{center}
		On peut alors se ramener à utiliser uniquement des relation binaires plutôt qu'une relation quaternaire:
		\begin{center}
		\begin{tikzpicture}
			\matrix[column sep=2em] (lignes)
			{
				\node (tableau1) {\begin{tabular}[t]{|c|}\hline \texttt{\quad Horaire \quad} \\ \hline ... \\ \hline \end{tabular}}; &
				\node (tableau2) {\begin{tabular}[t]{|c|}\hline \texttt{\quad Classe \quad } \\ \hline ... \\ \hline \end{tabular}}; &
				\node (tableau3) {\begin{tabular}[t]{|c|}\hline \texttt{\quad Salle \quad} \\ \hline ... \\ \hline \end{tabular}}; &
				\node (tableau4) {\begin{tabular}[t]{|c|}\hline \texttt{\quad Cours \quad} \\ \hline ... \\ \hline \end{tabular}}; \\
			};
			\coordinate (main1) at (tableau1.north);
			\coordinate (main2) at (tableau2.north);
			\coordinate (main3) at (tableau3.north);
			\coordinate (main4) at (tableau4.north);

			\path let \p1=(lignes.east), \p2=(lignes.west), \n1={\x1-\x2-2*\pgfkeysvalueof{/pgf/inner xsep}-\pgflinewidth} in
			node[above=of lignes,text width = \n1] (relation) 
			{
				\begin{tabularx}{\n1}{|Y|}\hline \texttt{Séance} \\ \hline \texttt{...} \\ \hline \end{tabularx}
			};

			\coordinate (top1) at ($(main1) + (0,1.1)$);
			\coordinate (top2) at ($(main2) + (0,1.1)$);
			\coordinate (top3) at ($(main3) + (0,1.1)$);
			\coordinate (top4) at ($(main4) + (0,1.1)$);
			\draw (main1) -- (top1) node[pos=0.1,right] {1..1} node[pos=0.9,right] {0..*} node[pos=0.5,right] {a lieu à};
			\draw (main2) -- (top2) node[pos=0.1,right] {1..1} node[pos=0.9,right] {0..*} node[pos=0.5,right] {assurée par};
			\draw (main3) -- (top3) node[pos=0.1,right] {1..1} node[pos=0.9,right] {0..*} node[pos=0.5,right] {dans la salle};
			\draw (main4) -- (top4) node[pos=0.1,right] {1..1} node[pos=0.9,right] {0..*} node[pos=0.5,right] {est un cours de};
		\end{tikzpicture}\\
		\textsc{Représentation d'une séance comme entité}.\\
		\end{center}

		\exemple Se pose aussi la question des relations \texttt{0..*} - \texttt{0..*}. 
		Prenons l'exemple d'un musée qui embauche un guide, parlant plusieurs langues.
		On souhaite savoir s'il est possible d'organiser des visites pour les malvoyants:
		\begin{center}	
		\begin{tikzpicture}
			\node[anchor=north west] (tableau1) at (0,0) {
				\begin{tabular}[t]{|c|}
					\hline
					\texttt{Guide}\\
					\hline
					\textcolor{orange}{\texttt{\textbf{nom}}}\\
					\texttt{Prénom} \\
					\texttt{Parle français} \\
					\texttt{Parle anglais} \\
					\hline
				\end{tabular}
			};
			\node[anchor=north west] (tableau2) at (12,0) {
			\begin{tabular}[t]{|c|}
				\hline
				\texttt{Musée}\\
				\hline
				\textcolor{orange}{\texttt{\textbf{id}}} \\
				\texttt{Nom} \\
				\texttt{Thème} \\
				\hline
			\end{tabular}};
			\coordinate (main1) at ($(tableau1.north east) - (0,0.3)$);
			\coordinate (main2) at ($(tableau2.north west) - (0,0.3)$);
			\draw (main1) -- (main2) node[pos=0.05,above]{1..*} node[pos=0.5,above] {Propose une visite pour les malvoyants} node[pos=0.95,above] {0..*};
		\end{tikzpicture} \\
		\end{center}
		Pour éviter d'avoir à écrire cette relation, on va créer une nouvelle entité, \texttt{visite}:
		\begin{center}
		\begin{tikzpicture}[node distance=5em]
			\node[anchor=north west] (guide)
			{
				\begin{tabular}{|c|}
					\hline
					\texttt{Guide}\\
					\hline
					\textcolor{orange}{\texttt{\textbf{nom}}}\\
					\texttt{Prénom} \\
					\texttt{Parle français} \\
					\texttt{Parle anglais} \\
					\hline
				\end{tabular}
			};
			\node[right=of guide.north east, anchor=north west] (visite)
			{
				\begin{tabular}{|c|}
					\hline
					\texttt{Visite} \\
					\hline
					\textcolor{orange}{\texttt{\textbf{id}}} \\
					\texttt{date} \\
					\texttt{pour les malvoyants} \\
					\hline
				\end{tabular}
			};
			\node[right=of visite.north east, anchor=north west] (musee)
			{
				\begin{tabular}{|c|}
					\hline
					\texttt{Musée}\\
					\hline
					\textcolor{orange}{\texttt{\textbf{id}}} \\
					\texttt{Nom} \\
					\texttt{Thème} \\
					\hline
				\end{tabular}
			};
			\coordinate (main1) at ($(guide.north east) - (0,0.3)$);
			\coordinate (main2) at ($(visite.north west) - (0,0.3)$);
			\coordinate (main3) at ($(visite.north east) - (0,0.3)$);
			\coordinate (main4) at ($(musee.north west) - (0,0.3)$);
			\draw (main1) -- (main2) node[pos=0.1,above] {1} node[pos=0.9,above] {*};
			\draw (main3) -- (main4) node[pos=0.1,above] {*} node[pos=0.9,above] {1};
		\end{tikzpicture}
		\end{center}


	\section{Requêtes SQL}
		Une fois notre base de données créée, on va pouvoir travailler avec via le SQL. 
		\subsection{Sélection simple}
			\subsubsection{Agir sur les colonnes}
				\semidef Le code suivant permet de récupérer certaines colonnes d'une table dans notre base de données.
				\begin{code}{sql}
					SELECT <colonne1>,<colonne2>,<colonne3>,...,<colonneN>
					FROM <table1>
				\end{code}
				où \texttt{<table1>} est une table de la base et \texttt{<colonne1>,<colonne2>,...<colonneN>} sont des colonnes de \texttt{<table1>}.

				\rem Les mots-clés sont ici écrits en majuscules. Même si ça n'est pas obligatoire, c'est recommandé pour des raisons de lisibilité
				(les noms de colonnes/tables seront alors en minuscule).
				
				\rem Si il y a plusieurs tables dans notre base de données contenant des colonnes portant le même nom,
				on pourra les distinguer en écrivant \texttt{tableau1.colonneA} et \texttt{tableau2.colonneB}.
				
				\semidef On peut récupérer les colonnes sous un nom modifié grâce au mot-clé \texttt{as}:
				\begin{code}{sql} 
					SELECT <nom_colonne> as <nouveau_nom>
					FROM <table1>
				\end{code}
				récupérera la colonne \texttt{<nom\_colonne>} sous le nom \texttt{<nouveau\_nom>}.

				Il est aussi possible de faire certaines opérations sur les éléments d'une même ligne (en spécifiant les colonnes de chaque élemnt), avant de récupérer le résultat:
				\begin{code}{sql}
					SELECT <nom_colonne> + 20 AS colonnePlus20, fin-debut AS duree
				\end{code}

				\rem Si jamais les colonnes n'ont pas les mêmes tailles, les éléments manquants seront interprétés comme nuls (valeur \texttt{NULL}).
				Tout opérateur appliqué à \texttt{NULL} renvoie \texttt{NULL}.

				\exemple Si le jour de naissance de différents éléments est stocké dans la colonne \texttt{jour\_de\_naissance} de la table \texttt{personnes},
				(comme un entier, mettons le nombre de jours depuis une date donnée)
				et que l'on a stocké le jour de leur dernier anniversaire dans la colonne \texttt{dernier\_anniversaire},
				on peut connaître l'âge de la personne
				\begin{code}{sql}
					SELECT (dernier_anniversaire-jour_de_naissance)/365 AS age
					FROM personnes 
				\end{code}

				\rem On notera \texttt{AS} pour les colonnes, on aura tendance à l'omettre pour les tables, même si la présence ou l'absence du mot clé n'a pas d'importance.

				\rem Pour selectionner toutes les colonnes, on peut utiliser une astérisque (*).
				
				Le mot clé \texttt{WHERE} permet de sélectionner dans les colonnes les lignes respectant certaines conditions:
				\begin{code}{sql}
					SELECT <colonne1>,...,<colonneN>
					FROM <table1>
					WHERE <cond>
				\end{code}
				où \texttt{<cond>} est une expression booléenne sur les attributs construite grâce aux opérateurs suivants:
				\begin{multicols}{4}
				\begin{itemize}
					\item \texttt{=}
					\item \texttt{<>}
					\item \texttt{<=}
					\item \texttt{>=}
					\item \texttt{<}
					\item \texttt{>}
					\item \texttt{BETWEEN .. AND}
					\item \texttt{AND}
					\item \texttt{OR}
					\item \texttt{NOT}
					\item \texttt{IS NULL}
					\item \texttt{IS NOT NULL}
				\end{itemize}
				\end{multicols}

				On peut sélectionner des lignes ayant deux valeurs distinctes sur des colonnes différentes via le mot-clé \texttt{DISTINCT}
				(attention cela-dit: cela supprime certains éléments, on ne sait pas lesquels!). 
				La distinction se fera alors sur \textsl{toutes} les colonnes spécifiées, pas seulement la première.

				\exemple Si on a une table \texttt{personnes} contenant une colonne \texttt{prenom} et une colonne \texttt{pays}, 
				et que l'on veut lister - sans doublon - les noms des personnes en France:
				\begin{code}{sql}
					SELECT DISTINCT prenom
					FROM personnes
					WHERE pays == "France"
				\end{code}

				\exemple Si maintenant on veut TOUS les prénoms, sans distinction de pays. 
				Ici, deux personnes ayant le même prénom, même si elles ne sont pas nées dans le même pays, seront regroupées en un seul enregistrement contenant leur prénom.
				\begin{code}{sql}
					SELECT DISTINCT prenom
					FROM personnes
				\end{code}

				\exemple Ici, si deux personnes ont le même prénom mais un pays différent, elles ne seront pas regroupées dans le même enregistrement:
				\begin{code}{sql}
					SELECT DISTINCT prenom,pays
					FROM personnes
				\end{code}

			\subsubsection{Tronquer et trier les résultats}
				Le mot clé \texttt{ORDER BY} trie les lignes selon les valeurs dans la colonne juste après.
				En cas d'égalité, il utilise la colonne éventuellement spécifiée après pour trancher, et ainsi de suite...
				Il faut à chaque fois préciser si le tri se fait par ordre croissant ou décroissant, via le mot clé \texttt{ASC} ou \texttt{DESC}
				
				Pour limiter le nombre de lignes récupérées, on peut utiliser le mot-clé \texttt{OFFSET n} qui efface les \texttt{n}-premières lignes,
				ainsi que le mot-clé \texttt{LIMIT k} qui ne garde que les \texttt{k} premières lignes.

				\begin{code}{sql}
					SELECT <colonne1>,<colonne2>,...,<colonneN>
					FROM <table1>
					WHERE <condition>
					ORDER BY <colonneA> <ASC | DESC>, <colonneB> <ASC | DESC>
					LIMIT <nombre_limite>
					OFFSET <decalage>
				\end{code}

		\subsection{Agrégation}
			\semidef On peut être ammené à faire des calculs opérant sur les valeurs de différentes lignes d'une même colonne,
			par exemple le nombre de lignes, leur moyenne, ou leur somme.
			Si par exemple on veut compter le nombre de personnes s'appelant Paul dans la table \texttt{personnes}:			
			\begin{code}{sql}
				SELECT COUNT(id)
				FROM personnes
				WHERE prenom == "Paul"
			\end{code}
			Les principales fonctions d'agrégation sont \texttt{SUM}, \texttt{AVG}, \texttt{COUNT}, \texttt{MAX}, \texttt{MIN}.

			\rem Si l'on applique une fonction d'agrégation à une colonne, et que l'on sélectionne une autre colonne en même temps,
			la fonction d'agrégation s'appliquera séparément sur toutes les lignes coïncidant sur l'autre colonne.

			\exemple Si on a une table \texttt{achat} contenant les ventes d'une boutique:
			\begin{center}
			\begin{tabularx}{0.5\textwidth}{|Y|Y|Y|Y|}
				\hline identifiant & client & tarif & date \\
				\hline 1           & Pierre & 102   & 2012-10-23\\
				\hline 2           & Simon  & 47    & 2012-10-27\\
				\hline 3           & Marie  & 18    & 2012-11-05\\
				\hline 4           & Marie  & 20    & 2012-11-14\\
				\hline 5           & Pierre & 160   & 2012-12-03\\
				\hline
			\end{tabularx}
			\end{center}
			La requête SQL
			\begin{code}{sql}
				SELECT client, SUM(tarif)
				FROM achat
			\end{code}
			renverrra:
			\begin{center}
			\begin{tabularx}{0.5\textwidth}{|Y|Y|Y|}
				\hline client & tarif \\
				\hline pierre & 262   \\
				\hline simon  & 47    \\
				\hline marie  & 38    \\
				\hline marie  & 38    \\
				\hline pierre & 262   \\
				\hline
			\end{tabularx}
			\end{center}

			\semidef On voit sur l'exemple précédent qu'il y des doublons. 
			Pour les éviter, il faut "grouper" les éléments par nom, via la commande \texttt{GROUP BY} 
			(qui se place TOUJOURS après la commande \texttt{WHERE}):
			\begin{code}{sql}
				SELECT client, SUM(tarif)
				FROM achat
				GROUP BY client
			\end{code}
			Renverra:
			\begin{center}
			\begin{tabularx}{0.5\textwidth}{|Y|Y|}
				\hline client & tarif \\
				\hline pierre & 262   \\
				\hline simon  & 47    \\
				\hline marie  & 38    \\
				\hline
			\end{tabularx}
			\end{center}

			Si l'on veut alors appliquer encore appliquer un filtre, on ne peut pas utiliser \texttt{WHERE}.
			La commande à utiliser est \texttt{HAVING}, qui fonctionne comme \texttt{WHERE}.

		\subsection{Jointure}
			\semidef Reprenons l'exemple des personnes et de leur pays de résidence.
			Si l'on veut connaître la capitale du pays dans lequel vivent les différentes personnes, on procèdera comme suit:
			\begin{code}{sql}
				SELECT pers.nom, pers.prenom, pays.capitale
				FROM personne AS pers
				JOIN pays
				ON pers.pays = pays.capitale
			\end{code}

		\subsection{Jointure gauche}
			\semidef Imaginons que l'on gère une liste de clients enregistrés dans une table \texttt{clients}, 
			avec la date de leur dernier achat dans une autre table \texttt{achats}.
			Certains clients n'auront pas encore fait d'achats. 
			Si l'on veut cependant obtenir tous les clients en y joignant leurs achats, on ne récupérera pas ces clients!
			La solution est d'utiliser \texttt{LEFT JOIN}:
			\begin{code}{sql}
				SELECT *
				FROM clients
				LEFT JOIN achats ON clients.id == achats.id_client
			\end{code}

			Il est alors possible de récupérer tous les clients n'ayant pas effecué d'achats:
			\begin{code}{sql}
				SELECT clients.id, prenom, nom, achats.id_client
				FROM clients
				LEFT JOIN achats ON clients.id == achats.id_client
				WHERE achats.id_clients IS NULL
			\end{code}

			\rem On utilise bien ici \texttt{IS NULL} et non \texttt{== NULL}, car \texttt{NULL} désigne l'absence de valeur: il n'y a donc rien à comparer!
\end{document}
