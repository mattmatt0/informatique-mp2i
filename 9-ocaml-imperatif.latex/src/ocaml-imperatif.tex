\newcommand\PATH{/home/matthieu/Documents/Ecole/Informatique/cours/9-ocaml-imperatif.latex}
\RequirePackage{fix-cm}
\documentclass{scrartcl}

% Fichier pour les dépendances

% Fontes Computer Modern
\usepackage{lmodern}

% Minted
\usepackage[cache=true,outputdir=\PATH/obj]{minted}

% Pour les symboles et outils mathématiques
\usepackage{amsmath}
\usepackage{amsfonts}
\usepackage{amssymb}
\usepackage{mathtools}
\usepackage{cancel}
\usepackage{mathrsfs}
\usepackage{stmaryrd}

% Pour les dessins / images
\usepackage{xcolor}
\usepackage{tikz}
\usepackage{graphicx}

% Tableaux et listes
\usepackage{tabularx}
\newcolumntype{Y}{>{\centering\arraybackslash}X}
\usepackage{booktabs}
\usepackage{multirow}
\usepackage{enumitem}
\usepackage{multicol}
\usepackage{adjustbox}

% Présentation de la page (bordures)
\usepackage[a4paper,top=1cm,bottom=1cm,includefoot,left=1.5cm,right=1.5cm,footskip=1cm]{geometry}
\usepackage{scrlayer-scrpage}
\rofoot*{\pagemark}
\cofoot*{}
\pagestyle{scrheadings}

% Autres
\usepackage{ulem} % Soulignage
\usepackage[french]{babel} % Babel, pour respecter les standards français
\frenchbsetup{StandardLists=true} % ... sauf pour les immondes listes à tirets
\usepackage[T1]{fontenc}

\usepackage{hyperref} % Pour les liens
\hypersetup{
    colorlinks=true,
    linkcolor=blue,
    filecolor=magenta,
    urlcolor=cyan,
}


%% Commandes personnalisées

% Code
\newenvironment{code}[1]
{\VerbatimEnvironment\begin{minted}[breaklines,frame=single,autogobble,linenos,bgcolor=codebg,tabsize=4,mathescape=true]{#1}}
{\end{minted}}
\definecolor{codebg}{gray}{1}
\newenvironment{algotext}
{\VerbatimEnvironment\begin{minted}[frame=single,autogobble,linenos,bgcolor=codebg,tabsize=4,escapeinside=||,mathescape=true]{text}}
{\end{minted}}


% Paragraphes spéciaux
\newcounter{compteurExos}
\setcounter{compteurExos}{0}
\newcommand{\plabel}[1]{{\parindent10pt \fontfamily{cmss}\selectfont\textbf{#1} \,}}

\newcommand{\prop}[1]{\plabel{Propriété :} \textsl{#1}}
\newcommand{\lemma}[1]{\plabel{Lemme :} \textsl{#1}}
\newcommand{\rem}{\plabel{Remarque :}}
\newcommand{\exemple}{\plabel{Exemple :}}
\newcommand{\exo}{\stepcounter{compteurExos}\plabel{Exercice \thecompteurExos :}}
\newcommand{\obs}{\plabel{Observation :} }

\newenvironment{demo}{\begin{itemize}[label=$\triangleright$]}{\end{itemize}}
\newcommand{\definition}{\parindent0pt }
\newcommand{\semidef}{\parindent0pt $\bullet$ \,}

% Outils
\newcommand{\corrpar}{\vspace{-20pt}}
\newcommand{\extraspace}{\vspace{5pt}}

% Configuration
\parskip5pt

% Annonce pour le développement:
\newcommand{\notecentrale}[1]{\begin{center}\textsl{#1}\end{center}}
\newcommand{\warnwip}{\notecentrale{Ce cours n'est pas encore bien relu}}
\newcommand{\warnwipnext}{\notecentrale{La suite de ce cours est encore en rédaction}\message}


\newcommand{\set}[1]{{\{#1\}}}
\newcommand{\intset}[1]{\llbracket #1 \rrbracket}
\newcommand{\intsete}[1]{\llbracket #1 \llbracket}
\newcommand{\inteset}[1]{\rrbracket #1 \rrbracket}
\newcommand{\intesete}[1]{\rrbracket #1 \llbracket}

\newcommand{\tq}{\, \big| \,}

\newcommand{\card}{\.\textmd{card}}
\newcommand{\Id}{\textsl{Id}}
\newcommand{\supp}{\text{supp}}

% Partie entière, arrondi supérieur

\DeclarePairedDelimiter\ceil{\lceil}{\rceil}
\DeclarePairedDelimiter\floor{\lfloor}{\rfloor}




\usepackage{lmodern}

\newcommand{\ocaml}[1]{\mintinline{ocaml}{#1}}
\newcommand{\unit}{\textcolor{purple}{\texttt{unit}}\,}
\newcommand{\codetrue}{\textcolor{green}{\texttt{true}}}
\newcommand{\codefalse}{\textcolor{red}{\texttt{false}}}

\title{Programmation impérative en OCaml}
\author{}
\date{}

\begin{document}
	\maketitle
	\begin{center}\textsl{En Ocaml, ce que l'on appellera instructions sont des expressions de type \texttt{unit}.}\end{center}
	\section{Expression de type \unit}
		On rappelle que le type \texttt{unit} a une unique valeur, \texttt{()}. 
		Les expressions de type \unit ne portent donc aucune information mais sont intéressantes pour l'effet que produit leur évaluation,
		appelé \textbf{effet de bord}.

		\exemple
		\begin{code}{ocaml} 
			assert (1 == 2);
			print_string "Hello world!";
		\end{code}

		\rem \texttt{let} n'est pas une expression de type \unit.

		\rem On ne peut pas juxtaposer des expressions de type \unit. Il faut les séparer par points-virgules.

		\exemple \texttt{e1;e2} est une expression de type et de valeur qui sont ceux de \texttt{e2}.
		L'évaluation de cette expression consiste en l'évaluation de \texttt{e1} puis l'évaluation de \texttt{e2}. 
		C'est équivalent à \ocaml{let _ = e1 in e2}. Le ";" est appelé \textbf{séquence}.

		\rem Attention: si \texttt{e1} n'est pas du type \unit, cela déclenche un avertissement qui indique que la valeur de e1 a été perdue.
	
		\rem Bien sûr, on peut généraliser à plus de 2 expressions...

		Pour gagner en lisibilité, il est possible d'utiliser les mots-clés \ocaml{begin} et \ocaml{end} à la place
		des parenthèses.

	\section{Références}
		Jusque là, nous avions appelé (à tort) \textsl{variables} des \textbf{valeurs nommées} dont la valeur ne changeait pas au cours du temps.
		En Ocaml, lorsque l'on veut une "vraie" variable, c'est à dire une case mémoire dont on va pouvoir modifier la valeur, il faut le préciser
		grâce au mot-clé \ocaml{ref}.	
		
		\exemple 
		\begin{code}{ocaml}
			let i = ref 1 in while (!i<3) do (print_int !i; i := !i+1) done;;
		\end{code}

		\verb|ref e|, où \verb|e| est une expression de type \verb|t| de valeur \verb|v|, est une expression de type \verb|t ref|,
		et de valeur \verb|{contents = v}|, c'est à dire une case mémoire contenant \verb|v| la valeur de \verb|e|.

		L'évalulation de cette expression consiste en:
		\begin{itemize}
			\item L'évaluation de \verb|e| en une valeur \verb|v|.
			\item La réservation d'une case mémoire pouvant stocker une donnée de type \verb|t|.
			\item L'initialisation de la donnée stockée dans cette case mémoire à \verb|v|.
		\end{itemize}

		\rem On ne peut pas déclarer/créer une variable sans lui donner de valeur! La création d'une nouvelle variable se fait comme suit:
		
		\plabel{Déclaration d'une variable:} Pour déclarer une variable en Ocaml, la syntaxe est la suivante:
		\begin{code}{ocaml}
			let var = ref e
		\end{code}
		où \verb|var| est un indentificateur et \verb|e| est une expression de type \verb|t| et de valeur \verb|v|

		\plabel{Accès à la valeur d'une variable:} Attention, dans l'expression précédente, \verb|var| est un identificateur! Si l'on s'intéresse
		à la valeur identifiée, il faut utiliser l'opérateur \texttt{![nom de la variable]}, équivalent au \texttt{*[nom de la variable]} en C:
		\begin{code}{ocaml}
			!var
		\end{code}
		où \verb|var| est une expression de type \verb|t ref|, est une expression de type \verb|t| et de valeur la valeur enregistrée dans la référence \verb|var|.

		\plabel{Modification de la valeur d'une variable:} Il faut utiliser \texttt{:=} à la place de \texttt{=}
		\begin{code}{ocaml}
			var := e
		\end{code}
		où \verb|var| est une expression du type \verb|t ref|, \verb|e| est une expression de type \verb|t|, est une expression de type \unit (et donc de valeur \verb|()|).
		Son évaluation consiste en l'évaluation de \verb|e| en une valeur \verb|v|,
		puis la modification de la valeur contenue dans la référence \verb|var| en \verb|v|. 

		\rem Attention, n'écrivez pas \verb|a := a+2|.

		\rem Attention, pour copier la valeur d'une variable dans une nouvelle variable, il faut faire une création de variable avec \verb|ref|.
		Le "\verb|=|" teste l'égalité entre valeurs, le "\verb|==|" teste l'égalité entre adresses mémoires. 

		\begin{code}{ocaml}
			var1 = var2
		\end{code}
		où \verb|var1| et \verb|var2| sont des expressions de type \verb|t ref| est une expression de type bool 
		et de valeur \verb|true| si \verb|var1| et \verb|var2| contiennent la même valeur (i.e \verb|(!var1) = (!var2)|)

	\section{Boucles}
		\subsection{Boucle \texttt{while}}
			\begin{code}{ocaml}
				while CONDITION do
					CORPS
				done
			\end{code}
			où \verb|CONDITION| est de type \verb|bool|, \verb|CORPS| est une expression de type \unit.
			Son évaluation consiste en l'évaluation de \verb|CONDITION|, puis:
			\begin{itemize}
				\item Si la valeur obtenue est \codefalse, l'évaluation est terminée
				\item sinon, on évalue \verb|CORPS|, et on recommence.
			\end{itemize}

			En pratique:
			\begin{itemize}
				\item On crée d'abord une référence (au moins), disons \verb|x|.
				\item \verb|CONDITION| fait apparaître \verb|!x|.
				\item \verb|CORPS| modifie la valeur de la référence \verb|x| (via \texttt{x:=...}).
			\end{itemize}
		\subsection{Boucle \texttt{for}}
			\begin{code}{ocaml}
				for i = INIT to FINAL do
					CORPS
				done
			\end{code}
			où \verb|i| est un identificateur (\textbf{attention}: et pas une référence),
			\verb|INIT| et \verb|FINAL| sont de type \textcolor{purple}{\texttt{int}}, et \verb|CORPS| une expression de type \unit, dans laquelle peut apparaître \verb|i|.
			Son évaluation consiste en l'évaluation de \verb|INIT| et \verb|FINAL| en $d \in \mathbb{Z}$ et $f \in \mathbb{Z}$,
			puis pour chaque $k \in \intset{d,f}$, l'évaluation de \verb|CORPS| où \verb|i| est évaluée à $k$.

			\rem Pour un comptage décroissant, on pourra utiliser \textcolor{teal}{\texttt{downto}} à la place de \textcolor{teal}{\texttt{to}}.

		\subsection{Tableaux}
			Les \texttt{array} sont en Ocaml la même chose que les tableaux en C. Ils permettent de stocker un nombre fini d'éléments de même type
			\plabel{Création d'un tableau à une dimension:}
			\begin{code}{ocaml}
				Array.make <taille du tableau> <valeur>
			\end{code}

			\plabel{Création d'un tableau à deux dimensions}
			\begin{code}{ocaml}
				Array.make_matrix <nombre de lignes/tableaux> <nombre de colonnes/taille des tableaux> <valeur>
			\end{code}
\end{document}
